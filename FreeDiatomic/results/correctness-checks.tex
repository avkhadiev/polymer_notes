%%%%%%%%%%%%%%%%%%%%%%%%%%%%%%%%%%%%%%%%%%%%%%%%%%%%%%%%%%%%%%%%%
%			      	SHAKE -> CORRECTNESS CHECKS		            %
%%%%%%%%%%%%%%%%%%%%%%%%%%%%%%%%%%%%%%%%%%%%%%%%%%%%%%%%%%%%%%%%%
% If you are creating an additional file in this or another folder,
% copy this header an make appropriate changes.
\subsection{Correctness Checks}
\label{sec:correctness-tolerance-checks}
%----------------------------------------------------------------
% Add text directly below! Add an \input statement in intro/main
% where appropriate
%----------------------------------------------------------------
\par Of course, the fictitious forces should not change the observed physics: they are merely a way to ensure that molecular constraints are respected. In particular, the  contribution to the virial from the fictitious forces should be zero on average over all molecules at each time step. This constraint virial is set to zero at each time step before iterating over all molecules; if a given bond length is corrected and the corresponding fictitious force is calculated, the constraint virial is updated as:
\begin{tcolorbox}
\begin{equation}
\label{eq:constraint-virial}
	\mathcal{W}_G \rightarrow \mathcal{W}_G + \frac{1}{3} \frac{1}{\delta t^2} g_{AB} d^2_{AB}
\end{equation}
\end{tcolorbox}
\par The divison by 3 and the square of the time step is done in the end, after iterating over all molecules.