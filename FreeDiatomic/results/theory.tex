%%%%%%%%%%%%%%%%%%%%%%%%%%%%%%%%%%%%%%%%%%%%%%%%%%%%%%%%%%%%%%%%%
%			      	  SHAKE -> INTRODUCTION				        %
%%%%%%%%%%%%%%%%%%%%%%%%%%%%%%%%%%%%%%%%%%%%%%%%%%%%%%%%%%%%%%%%%
% If you are creating an additional file in this or another folder,
% copy this header an make appropriate changes.				
\subsection{Theory}
\label{sec:shake-theory}
%----------------------------------------------------------------
% Add text directly below! Add an \input statement in intro/main
% where appropriate
%----------------------------------------------------------------
\par First, let us write the atomic position vectors at time $t + \delta t$, $\mathbf{r}_{a}(t + \delta t)$, as 
\begin{equation}
\label{eq:shake-setup}
\mathbf{r}_a(t + \delta t) = \mathbf{r}^{(0)}_a(t + \delta t) + \delta \mathbf{r}_a
\end{equation} 
where  $\mathbf{r}^{(0)}_a(t + \delta t)$ are the atomic position vectors that would have been in the absence of constraints at the time $t + \delta t$, and $\delta \mathbf{r}_a$ is the correction due to constraint forces.
\par The problem at hand is, as usual, to solve the equations of motion with constraints:
\begin{tcolorbox}
	\begin{align}
	\label{eq:shake-eoms}
	m_{a} \ddot{\mathbf{r}}_a &= \mathbf{F}_a + \mathbf{G}_a 				 	\\
	\label{eq:constraint-forces}
	\mathbf{G}_a 	 &= - \sum_{k = 1}^l \lambda_k(t) \nabla_{a} \sigma_k(t)	\\
	\label{eq:constraints}
	\sigma_k(t) 	 &= (\mathbf{r}_B(t) - \mathbf{r}_A(t))^2 - d_{AB}^2 = 0
	\end{align}
\end{tcolorbox}
Here, $\mathbf{F}_a(t)$ is the force on atom $a$ at time $t$ due to ``intermolecular interactions and those intramolecular effects explicitly included in the potential''\cite{allen}. Those effects do not include the fictitious constraint forces, $\mathbf{G}_a(t)$. The magnitude of the constraint forces is determined by the time-dependend Lagrange multipliers $\lambda_k(t)$. The number of these multipliers is equal to the number of constraints $\sigma_k(t)$ that keep the bond lenghts fixed.
\par As Ryckaert,  Ciccotti, and Berendsen \cite{Shake} show, the constraints \ref{eq:constraints} ``are satisfied by adding displacement vectors $\delta \mathbf{r}_a$ to the vectors \ldots which resulted from a nonconstrained timestep'', and the displacement vectors for the Verlet algorithm are given by 
\begin{tcolorbox}
	\begin{equation}
	\label{eq:shake-verlet-displacement}
	\begin{aligned}
	\delta \mathbf{r}_a &= - \frac{1}{m_a}(\delta t)^2 \sum_{k = 1}^l \gamma_k ( \nabla_a \sigma_k )_{t} 	\\
	&= - \frac{2(\delta t)^2}{m_a} \sum_{k = 1}^{l} \gamma_k \mathbf{r}_{AB}(t)
	\end{aligned}
	\end{equation} 
\end{tcolorbox}
Where $\gamma_k$ are constants equal to $\lambda_k(t)$ up to order $\mathcal{O}(\delta t^2)$.
\par We can now define
\begin{equation}
\label{eq:theory-gab}
g_{AB} = - 2 (\delta t)^2 \gamma_k
\end{equation}
So that $\frac{g_{AB} \mathbf{r}_{AB}(t)}{m_A}$ is the contribution of the $k$th constraint to the displacement $\delta \mathbf{r}_A$. Summing over all such contributions, 
\begin{equation}
\label{eq:shake-full-displacement}
\delta \mathbf{r}_A = \sum_{B} \frac{g_{AB} \mathbf{r}_{AB}(t)}{m_A}
\end{equation}
\par The idea of $\textsf{SHAKE}$ is to consider these constraints one-by-one. The correction to the position vector of atom $A$ due to $k$th constraint on the bond between $A$ and $B$ is given by
\begin{equation*}
\begin{aligned}
\delta^k \mathbf{r}_A &= \frac{g_{AB}}{m_A} \mathbf{r}_{AB}(t)		\\
\delta^k \mathbf{r}_B &= -\frac{g_{AB}}{m_B} \mathbf{r}_{AB}(t)
\end{aligned}
\end{equation*}
\par Consequently, the unconstrained position $\mathbf{r}^{(0)}_a(t + \delta t)$ is corrected by $\sum_{k} \delta^k \mathbf{r}_a$. Since $\textsf{SHAKE}$ considers all constraints in succession, it is convenient to define 
\begin{equation*}
\begin{aligned}
\mathbf{r}^{(i + 1)}_{AB}(t + \delta t) &= \mathbf{r}^{(i)}_{A}(t + \delta t) + \sum_{k^\prime < k} \delta^{k^\prime} \mathbf{r}_A - (\mathbf{r}^{(i)}_{B}(t + \delta t) + \sum_{k^\prime < k} \delta^{k^\prime} \mathbf{r}_B) \\
\delta \mathbf{r}_{AB} &= \delta^{k} \mathbf{r}_{A} - \delta^{k} \mathbf{r}_{B}
\end{aligned}
\end{equation*}
\par We can now re-write the constraint equation \ref{eq:constraints} as 
\begin{equation*}
(\mathbf{r}^{(i)}_{AB}(t + \delta t) + \delta \mathbf{r}_{AB})^2 - d_{AB}^2 = 0
\end{equation*}
with
\begin{equation*}
\delta \mathbf{r}_{AB} = \left( \frac{1}{m_A} + \frac{1}{m_B} \right) g_{AB} \mathbf{r}_{AB}(t)
\end{equation*}
which yields the equation
\begin{tcolorbox}
	\begin{equation}
	\label{eq:shake-main}
	2\left( \frac{1}{m_A} + \frac{1}{m_B} \right) g_{AB} (\mathbf{r}_{AB}(t) \cdot \mathbf{r}^{(i)}_{AB}(t + \delta t)) + \left( \frac{1}{m_A} + \frac{1}{m_B} \right)^2 g_{AB}^2 \mathbf{r}_{AB}(t)^2 = d_{AB}^2 - \mathbf{r}^{(i)}_{AB}(t + \delta t)^2 
	\end{equation}
\end{tcolorbox}
\par The computation of $k$th constraint partially destroyes that of the previous constraints, so the procedure is iterated until every constraint is satisfied to within the specified tolerance. For computational efficiency, \ref{eq:shake-main} is solved only to first order. This is not bad, since the quadratic term is of order $\mathcal{O}(\delta t^4)$. Moreover, ``the iterative nature of the procedure assures that each quadratic equation is finally also solved to within the specified tolerance'' \cite{Shake}.