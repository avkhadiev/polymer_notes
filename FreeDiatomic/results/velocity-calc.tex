%%%%%%%%%%%%%%%%%%%%%%%%%%%%%%%%%%%%%%%%%%%%%%%%%%%%%%%%%%%%%%%%%
%			      	SHAKE -> VELOCITY CALCULATION               %
%%%%%%%%%%%%%%%%%%%%%%%%%%%%%%%%%%%%%%%%%%%%%%%%%%%%%%%%%%%%%%%%%
% If you are creating an additional file in this or another folder,
% copy this header an make appropriate changes.
\subsection{Calculation of Velocities}
\label{sec:velocity-calc}
%----------------------------------------------------------------
% Add text directly below! Add an \input statement in intro/main
% where appropriate
%----------------------------------------------------------------
\par Once atomic position vectors at time $t + \delta t$ have been fully corrected, it is possible to calculate the atomic velocities $\mathbf{v}_a(t + \delta t)$ according to the prescription of the Verlet scheme:
\begin{equation}
\label{eq:calc-velocities}
	\mathbf{v}_a(t + \delta t) = \frac{\mathbf{r}_a(t + \delta t) - \mathbf{r}_a(t - \delta t)}{2 \delta t}
\end{equation}
This way one can also keep track of the kinetic energy of the system at each time step.