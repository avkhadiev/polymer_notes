%%%%%%%%%%%%%%%%%%%%%%%%%%%%%%%%%%%%%%%%%%%%%%%%%%%%%%%%%%%%%%%%%
%			      	    SHAKE -> VERLET STEP			        %
%%%%%%%%%%%%%%%%%%%%%%%%%%%%%%%%%%%%%%%%%%%%%%%%%%%%%%%%%%%%%%%%%
% If you are creating an additional file in this or another folder,
% copy this header an make appropriate changes.
\subsection{Verlet Step}
\label{sec:shake-verlet-step}
%---------- ------------------------------------------------------
% Add text directly below! Add an \input statement in intro/main
% where appropriate
%----------------------------------------------------------------
\par The positions at the next timestep in the absence of constraints are calculated for each atom $a$ in the usual way, using the  Verlet scheme:
\begin{equation}
\label{eq:shake-verlet}
\mathbf{r^{(0)}}_a(t + \delta t) = 2 \mathbf{r}_a(t) - \mathbf{r}_a(t - \delta t) + \frac{\delta t^2}{m_a} \mathbf{F}_a(t)
\end{equation}
where $\mathbf{F}_a(t)$ is the force on atom $a$ at time $t$ due to ``intermolecular interactions and those intramolecular effects explicitly included in the potential''\cite{allen} (those effects do not include the fictitious constraint forces).
\par These positions of atoms calculated in the absence of constraint forces can be thought of as the zeroeth step in the iterative correction procedure described above --- hence the notation.
