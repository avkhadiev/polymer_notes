%%%%%%%%%%%%%%%%%%%%%%%%%%%%%%%%%%%%%%%%%%%%%%%%%%%%%%%%%%%%%%%%%%%%%%%%%%%%%%%%
%					 	    KINETIC ENERGY OF CM -> KINETIC ENERGY					               %
%%%%%%%%%%%%%%%%%%%%%%%%%%%%%%%%%%%%%%%%%%%%%%%%%%%%%%%%%%%%%%%%%%%%%%%%%%%%%%%%
% Subsection names and labels
\emph{Kinetic Energy}
\label{sec:calculation-kinetic-energy-cm-ke}
%-------------------------------------------------------------------------------
% Add text directly into the main file, or use \input statements
%------------------------------------------------------------------------------
  \par Calculating $\vcmsq = \vcm \bigcdot \vcm$, we find
  %-------------------------------------------------------------------------------
% Add math environment directly below. Include a label.
%-------------------------------------------------------------------------------
\begin{equation}
\label{eq:vcmsq}
\begin{aligned}
    \vcmsq
      &= \left(\vcmtargetrep\right) \bigcdot \left(\vcmtargetrep\right) \\
      &= \vel^2_{0}
        + 2 \frac{\bondlength}{N+1}\sum_{i=1}^{N}(N + 1 - i)
          \left(
            \vel_{0} \bigcdot \bondvel_{i}
          \right) \\
      &+ \left(\frac{\bondlength }{N+1}\right)^2
          \sum_{i=1}^{N} (N + 1 - i)^2 \bondvel^2_{i} \\
      &+ 2 \left(\frac{\bondlength}{N+1}\right)^2
          \sum_{i=1}^{N-1}\sum_{j=i+1}^{N}
            (N + 1 - i)(N + 1 - j)
          \left(
            \bondvel_{i} \bigcdot \bondvel_{j}
          \right) \\
\end{aligned}
\end{equation}
%-------------------------------------------------------------------------------

  \par We can re-write the second term in the equation above using a linear form, $\frac{\bondlength}{N+1} A_{i} (\vel_{0} \bigcdot \bondvel^{i})$ (using Einstein notation and with the index running from $1$ to $N$)
  \begin{tcolorbox}
    \begin{equation}
    \label{eq:vector-a-ke-cm}
      A_{i} = (N + 1 - i)
    \end{equation}
  \end{tcolorbox}
  \par Similarly, we can re-package the last two terms in the equation above into a quadratic form, $\left(\frac{\bondlength}{N+1}\right)^2 B_{ij} \bondvel^{i} \bondvel^{j}$ (using Einstein notation and with the indices running from $1$ to $N$), where
  \begin{tcolorbox}
    \begin{equation}
    \label{eq:matrix-ke-cm}
      B_{ij} = (N + 1 - i) (N + 1 - j)
    \end{equation}
  \end{tcolorbox}
  Note that $i$, $j$ are indexing 3-vectors.
  \par For a polymer of any given number of links $N$, the vector $\vec{A}$ from equation \ref{eq:vector-a-ke-cm} can be precomputed; matrix $B$ can be calculated as an outer product $B = \vec{A} \otimes \vec{A}$. These quantities can then be made readily available whenever kinetic energy needs to be computed from the state of the simulation.
  \par Using the above definitions, we may finally write down the kinetic energy of the center of mass as
  \begin{tcolorbox}
    \begin{equation}
    \label{eq:k-cm}
        K_{\textrm{CM}} = \onehalf (N+1)m
          \left(
            \vel^2_{0}
              + \frac{\bondlength}{N+1}
                  A_{i} \left(\vel_{0} \bigcdot \bondvel^{i}\right)
              + \left(\frac{\bondlength}{N+1}\right)^2
                B_{ij} \left(\bondvel^{i} \bondvel^{j}\right)
          \right)
    \end{equation}
  \end{tcolorbox}
%-------------------------------------------------------------------------------
