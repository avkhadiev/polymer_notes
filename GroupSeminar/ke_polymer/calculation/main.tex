%%%%%%%%%%%%%%%%%%%%%%%%%%%%%%%%%%%%%%%%%%%%%%%%%%%%%%%%%%%%%%%%%%%%%%%%%%%%%%%%
%					 	                         SETUP								                     %
%%%%%%%%%%%%%%%%%%%%%%%%%%%%%%%%%%%%%%%%%%%%%%%%%%%%%%%%%%%%%%%%%%%%%%%%%%%%%%%%
% Section names and labels
\subsection{Calculation}
\label{sec:kePolymer_alculation}
  %-----------------------------------------------------------------------------
  % Add text directly into the main file, or use \input statements
  %-----------------------------------------------------------------------------
  %\par We will now convert equation \ref{eq:kinetic-energy-of-system} of the kinetic energy for a classical system from the initial representation $\initialrep$ to the desired representation $\targetrep$. We will do so in two steps: first, find the expression for the kinetic energy of the center of mass; second, find the expression of the ``internal'' kinetic energy corresponding to the motion relative to the center of mass.
  %-----------------------------------------------------------------------------
  \documentclass[%
    reprint,
    superscriptaddress,
    %groupedaddress,
    %unsortedaddress,
    %runinaddress,
    %frontmatterverbose,
    %preprint,
    longbibliography,
    bibnotes,
    %showpacs,preprintnumbers,
    %nofootinbib,
    %nobibnotes,
    %bibnotes,
     amsmath,amssymb,
     %aps,
     aip,
     jcp,                                       % J. Chem. Phys
    %pra,
    %prb,
    %rmp,
    %prstab,
    %prstper,
    %floatfix,
    ]{revtex4-1}
%%%%%%%%%%%%%%%%%%%%%%%%%%%%%%%%%%%%%%%%%%%%%%%%%%%%%%%%%%%%%%%%%%%%%%%%%%%%%%%%
%					 	                       LIBRARIES							                     %
%%%%%%%%%%%%%%%%%%%%%%%%%%%%%%%%%%%%%%%%%%%%%%%%%%%%%%%%%%%%%%%%%%%%%%%%%%%%%%%%
% Add whatever libraries you need here. Add a description if you want
\usepackage{graphicx}			 % Include figure files
\usepackage{dcolumn}			 % Align table columns on decimal point
\usepackage{tcolorbox}		 % use tcolorbox environment to box equations
\usepackage[USenglish]{babel}
\usepackage[useregional]{datetime2}
\usepackage[utf8]{inputenc}
\usepackage[T1]{fontenc}
\usepackage{enumerate}
\usepackage{caption}
\usepackage{subcaption}
\usepackage{physics}
\usepackage{hyperref}
\usepackage{chemformula}
\usepackage{booktabs}
\usepackage{makecell}
\usepackage{fullpage}
\usepackage{natbib}
\usepackage{setspace}
\usepackage{amsmath}
\usepackage{bm}              % enchanced bold math symbols
\usepackage{amssymb}
\usepackage{listings} 			 % For code citations
\usepackage{amsfonts}
\usepackage{mathtools}
\usepackage{commath}
\usepackage{fancyhdr}
\usepackage{lastpage}
\usepackage{tikz}
\usepackage{graphicx}
	\graphicspath{ {images/} }
\usepackage{feynmp-auto}
\usepackage[toc,page]{appendix}
%%%%%%%%%%%%%%%%%%%%%%%%%%%%%%%%%%%%%%%%%%%%%%%%%%%%%%%%%%%%%%%%%%%%%%%%%%%%%%%%
%					 	                      NEW COMMANDS							                   %
%%%%%%%%%%%%%%%%%%%%%%%%%%%%%%%%%%%%%%%%%%%%%%%%%%%%%%%%%%%%%%%%%%%%%%%%%%%%%%%%
% Redefine commands if necessary
\renewcommand\vec{\mathbf}                      % use boldface for vectors
% New commands
%------------------------------------Date---------------------------------------
\newcommand{\seminardate}{\DTMdisplaydate{2017}{7}{25}{-1}}
\newcommand{\revisiondate}{\DTMdisplaydate{2017}{7}{26}{-1}}
%-------------------------------------------------------------------------------
\newcommand*\Del{\vec{\nabla}}                  % del operator
\newcommand*\diff{\mathop{}\!\mathrm{d}}		    % differential d
\newcommand*\Diff[1]{\mathop{}\!\mathrm{d^#1}}	% d^(...)
\newcommand*{\scalarnorm}[1]                    % scalar norm |*|
  {\left\lvert#1\right\rvert}
\newcommand*{\vectornorm}[1]                    % vector norm ||*||
  {\left\lVert#1\right\rVert}
\newcommand*\onehalf{\frac{1}{2}}               % fraction 1/2
%------------------------------------Time---------------------------------------
\newcommand*\dlt[1]{\delta #1}
\newcommand*\timestep{\dlt{t}}              % timestep dt
\newcommand*\halfstep                       % half timestep 1/2 dt
  {\onehalf \timestep}
\newcommand*\timeInHalfStep{\left(t + \halfstep\right)}
\newcommand*\timeInFullStep{\left(t + \timestep\right)}
%-------------------------------------------------------------------------------
%%%%%%%%%%%%%%%%%%%%%%%%%%%%%%%%%%%%%%%%%%%%%%%%%%%%%%%%%%%%%%%%%%%%%%%%%%%%%%%%
%					 	                    DEFINITIONS								                     %
%%%%%%%%%%%%%%%%%%%%%%%%%%%%%%%%%%%%%%%%%%%%%%%%%%%%%%%%%%%%%%%%%%%%%%%%%%%%%%%%
%-------------------------------------------------------------------------------
\newcommand*\rattle{\textsf{RATTLE} }       % display name of algorithm
%-------------------------------------------------------------------------------
\newcommand*\rattlePos{\vec{r}}             % position vector
\newcommand*\rattleVel{\vec{v}}             % velocity vector
\newcommand*\rattleAcc{\vec{a}}             % acceleration vector
%-------------------------------------------------------------------------------
\newcommand*\rattleMassNoIndex{m}
\newcommand*\rattleForceNoIndex{\vec{F}}
\newcommand*\rattleConstraintForceNoIndex{\vec{G}}
\newcommand*\rattleConstraintNoIndex{\sigma}
\newcommand*\rattleLagrangeNoIndex{\lambda}
\newcommand*\rattleLagrangeApproxNoIndex{\gamma}
\newcommand*\rattleCorrectiveValueNoIndex{g}
%---------------------Definitions for General Formulation ----------------------
%%%%%%%%%%%%%%%%%%%%%%%%%%%%%%%%%%%%%%%%%%%%%%%%%%%%%%%%%%%%%%%%%%%%%%%%%%%%%%%%
%					 	                    DEFINITIONS								                     %
%%%%%%%%%%%%%%%%%%%%%%%%%%%%%%%%%%%%%%%%%%%%%%%%%%%%%%%%%%%%%%%%%%%%%%%%%%%%%%%%
%-------------------------------------------------------------------------------
\newcommand*\rattle{\textsf{RATTLE} }       % display name of algorithm
%-------------------------------------------------------------------------------
\newcommand*\rattlePos{\vec{r}}             % position vector
\newcommand*\rattleVel{\vec{v}}             % velocity vector
\newcommand*\rattleAcc{\vec{a}}             % acceleration vector
%-------------------------------------------------------------------------------
\newcommand*\rattleMassNoIndex{m}
\newcommand*\rattleForceNoIndex{\vec{F}}
\newcommand*\rattleConstraintForceNoIndex{\vec{G}}
\newcommand*\rattleConstraintNoIndex{\sigma}
\newcommand*\rattleLagrangeNoIndex{\lambda}
\newcommand*\rattleLagrangeApproxNoIndex{\gamma}
\newcommand*\rattleCorrectiveValueNoIndex{g}
%---------------------Definitions for General Formulation ----------------------
\input{rattle/general/definitions}
%-----------------------------------Indexing------------------------------------
\newcommand*\rattleAtomIndex{a}               % solo atom index
\newcommand*\rattleAtomIndexFirst{a}          % first index in atomic pair
\newcommand*\rattleAtomIndexSecond{b}         % second index in atomic pair
\newcommand*\rattleNAtoms{N}                  % number of atoms
\newcommand*\rattleConstraintSet{K}           % set of constraints
\newcommand*\rattleConstraintIndex            % index for constraints
  {\left(
    \rattleAtomIndexFirst,\,\rattleAtomIndexSecond
  \right) \in \rattleConstraintSet}
\newcommand*\rattleConstraintIndexSimple      % simplified index for constraints
  {\rattleAtomIndexFirst\rattleAtomIndexSecond}
\newcommand*\rattleCorrectionIndex{i}         % index for iterative correction
\newcommand*\rattleCorrectionIndexThis        % current for iterative correction
  {\rattleCorrectionIndex}
\newcommand*\rattleCorrectionIndexNext        % next for iterative correction
    {\rattleCorrectionIndex + 1}
\newcommand*\rattleCorrectionIndexLast{m}     % last step of correction
\newcommand*\rattleNCorrections{M}            % number of iterative corrections
%-----------------------------------Dynamics------------------------------------
\newcommand*\rattleMass                       % solo atomic mass
  {\rattleMassNoIndex_{\rattleAtomIndex}}
\newcommand*\rattleMassFirst                  % first atomic mass in pair
    {\rattleMassNoIndex_{\rattleAtomIndexFirst}}
\newcommand*\rattleMassSecond                 % second atomic mass in pair
    {\rattleMassNoIndex_{\rattleAtomIndexSecond}}
\newcommand*\rattleForce
  {\rattleForceNoIndex_{\rattleAtomIndex}}
\newcommand*\rattleConstraintForce            % solo constraint force
  {\rattleConstraintForceNoIndex_{\rattleAtomIndex}}
\newcommand*\rattleConstraintForceFirst       % first constraint force in pair
    {\rattleConstraintForceNoIndex_{\rattleAtomIndexFirst}}
\newcommand*\rattleConstraintForceSecond      % second constraint force in pair
    {\rattleConstraintForceNoIndex_{\rattleAtomIndexSecond}}
\newcommand*\rattleConstraintForceBond        % constraint force along bond
        {\rattleConstraintForceNoIndex_{\rattleConstraintIndexSimple}}
\newcommand*\rattleConstraint
  {\rattleConstraintNoIndex_{\rattleConstraintIndexSimple}}
  \newcommand*\rattleConstraintDot
    {\dot{\rattleConstraintNoIndex}_{\rattleConstraintIndexSimple}}
\newcommand*\rattleLagrange
    {\rattleLagrangeNoIndex_{\rattleConstraintIndexSimple}}
\newcommand*\rattleLagrangeApprox
        {\rattleLagrangeApproxNoIndex_{\rattleConstraintIndexSimple}}
\newcommand*\rattleCorrectiveValue
  {\rattleCorrectiveValueNoIndex_{\rattleConstraintIndexSimple}}
%-----------------------------IterativeCorrection-------------------------------
\newcommand*\iteration[2]{#1^{(#2)}}            % iterative correction steps
\newcommand*\iterationThis[1]                   % arg^{i}
  {\iteration{#1}{\rattleCorrectionIndexThis}}
\newcommand*\iterationNext[1]                   % arg^{i+1}
  {\iteration{#1}{\rattleCorrectionIndexNext}}
%----------------------------------Vectors--------------------------------------
\newcommand*\rattleAtomPos{\rattlePos_{\rattleAtomIndex}}
\newcommand*\rattleAtomPosFirst{\rattlePos_{\rattleAtomIndexFirst}}
\newcommand*\rattleAtomPosSecond{\rattlePos_{\rattleAtomIndexSecond}}
\newcommand*\rattleAtomVel{\rattleVel_{\rattleAtomIndex}}
\newcommand*\rattleAtomVelFirst{\rattleVel_{\rattleAtomIndexFirst}}
\newcommand*\rattleAtomVelSecond{\rattleVel_{\rattleAtomIndexSecond}}
\newcommand*\rattlePosUnconstrained             % r^{(0)}
  {\iteration{\rattlePos}{0}}
\newcommand*\rattleVelUnconstrained             % v^{(0)}
  {\iteration{\rattleVel}{0}}
\newcommand*\rattleAtomPosUnconstrained         % r^{(0)}
  {\iteration{\rattleAtomPos}{0}}
\newcommand*\rattleAtomVelUnconstrained         % v^{(0)}
  {\iteration{\rattleAtomVel}{0}}
\newcommand*\rattleBondPos{\rattlePos_{\rattleConstraintIndexSimple}}
\newcommand*\rattleBondPosUnit{\hat{\rattlePos}_{\rattleConstraintIndexSimple}}
\newcommand*\rattleBondVel{\rattleVel_{\rattleConstraintIndexSimple}}
\newcommand*\rattleBondVelUnit{\hat{\rattleVel}_{\rattleConstraintIndexSimple}}
\newcommand*\rattleBondPosUnconstrained
  {\iteration{\rattleBondPos}{0}}
\newcommand*\rattleBondVelUnconstrained
  {\iteration{\rattleBondVel}{0}}
\newcommand*\rattleAtomPosUnconstrainedFirst
    {\iteration{\rattleAtomPosFirst}{0}}
\newcommand*\rattleAtomPosUnconstrainedSecond
    {\iteration{\rattleAtomPosSecond}{0}}
\newcommand*\rattleAtomVelUnconstrainedFirst
    {\iteration{\rattleAtomVelFirst}{0}}
\newcommand*\rattleAtomVelUnconstrainedSecond
    {\iteration{\rattleAtomVelSecond}{0}}
%---------------------------------Constants-------------------------------------
\newcommand*\rattleBondlength                   % fixed bond length
  {d_{\rattleConstraintIndexSimple}}
\newcommand*\rattleTol{\xi}                     % d-r/d tolerance
\newcommand*\rattleRRTol{\varepsilon}           % r(t) * r(t+dt) tolerance
\newcommand*\rattleRVTol{\rattleTol}            % v(t)/d tolerance
%-------------------------------------------------------------------------------

%-----------------------------------Indexing------------------------------------
\newcommand*\rattleAtomIndex{a}               % solo atom index
\newcommand*\rattleAtomIndexFirst{a}          % first index in atomic pair
\newcommand*\rattleAtomIndexSecond{b}         % second index in atomic pair
\newcommand*\rattleNAtoms{N}                  % number of atoms
\newcommand*\rattleConstraintSet{K}           % set of constraints
\newcommand*\rattleConstraintIndex            % index for constraints
  {\left(
    \rattleAtomIndexFirst,\,\rattleAtomIndexSecond
  \right) \in \rattleConstraintSet}
\newcommand*\rattleConstraintIndexSimple      % simplified index for constraints
  {\rattleAtomIndexFirst\rattleAtomIndexSecond}
\newcommand*\rattleCorrectionIndex{i}         % index for iterative correction
\newcommand*\rattleCorrectionIndexThis        % current for iterative correction
  {\rattleCorrectionIndex}
\newcommand*\rattleCorrectionIndexNext        % next for iterative correction
    {\rattleCorrectionIndex + 1}
\newcommand*\rattleCorrectionIndexLast{m}     % last step of correction
\newcommand*\rattleNCorrections{M}            % number of iterative corrections
%-----------------------------------Dynamics------------------------------------
\newcommand*\rattleMass                       % solo atomic mass
  {\rattleMassNoIndex_{\rattleAtomIndex}}
\newcommand*\rattleMassFirst                  % first atomic mass in pair
    {\rattleMassNoIndex_{\rattleAtomIndexFirst}}
\newcommand*\rattleMassSecond                 % second atomic mass in pair
    {\rattleMassNoIndex_{\rattleAtomIndexSecond}}
\newcommand*\rattleForce
  {\rattleForceNoIndex_{\rattleAtomIndex}}
\newcommand*\rattleConstraintForce            % solo constraint force
  {\rattleConstraintForceNoIndex_{\rattleAtomIndex}}
\newcommand*\rattleConstraintForceFirst       % first constraint force in pair
    {\rattleConstraintForceNoIndex_{\rattleAtomIndexFirst}}
\newcommand*\rattleConstraintForceSecond      % second constraint force in pair
    {\rattleConstraintForceNoIndex_{\rattleAtomIndexSecond}}
\newcommand*\rattleConstraintForceBond        % constraint force along bond
        {\rattleConstraintForceNoIndex_{\rattleConstraintIndexSimple}}
\newcommand*\rattleConstraint
  {\rattleConstraintNoIndex_{\rattleConstraintIndexSimple}}
  \newcommand*\rattleConstraintDot
    {\dot{\rattleConstraintNoIndex}_{\rattleConstraintIndexSimple}}
\newcommand*\rattleLagrange
    {\rattleLagrangeNoIndex_{\rattleConstraintIndexSimple}}
\newcommand*\rattleLagrangeApprox
        {\rattleLagrangeApproxNoIndex_{\rattleConstraintIndexSimple}}
\newcommand*\rattleCorrectiveValue
  {\rattleCorrectiveValueNoIndex_{\rattleConstraintIndexSimple}}
%-----------------------------IterativeCorrection-------------------------------
\newcommand*\iteration[2]{#1^{(#2)}}            % iterative correction steps
\newcommand*\iterationThis[1]                   % arg^{i}
  {\iteration{#1}{\rattleCorrectionIndexThis}}
\newcommand*\iterationNext[1]                   % arg^{i+1}
  {\iteration{#1}{\rattleCorrectionIndexNext}}
%----------------------------------Vectors--------------------------------------
\newcommand*\rattleAtomPos{\rattlePos_{\rattleAtomIndex}}
\newcommand*\rattleAtomPosFirst{\rattlePos_{\rattleAtomIndexFirst}}
\newcommand*\rattleAtomPosSecond{\rattlePos_{\rattleAtomIndexSecond}}
\newcommand*\rattleAtomVel{\rattleVel_{\rattleAtomIndex}}
\newcommand*\rattleAtomVelFirst{\rattleVel_{\rattleAtomIndexFirst}}
\newcommand*\rattleAtomVelSecond{\rattleVel_{\rattleAtomIndexSecond}}
\newcommand*\rattlePosUnconstrained             % r^{(0)}
  {\iteration{\rattlePos}{0}}
\newcommand*\rattleVelUnconstrained             % v^{(0)}
  {\iteration{\rattleVel}{0}}
\newcommand*\rattleAtomPosUnconstrained         % r^{(0)}
  {\iteration{\rattleAtomPos}{0}}
\newcommand*\rattleAtomVelUnconstrained         % v^{(0)}
  {\iteration{\rattleAtomVel}{0}}
\newcommand*\rattleBondPos{\rattlePos_{\rattleConstraintIndexSimple}}
\newcommand*\rattleBondPosUnit{\hat{\rattlePos}_{\rattleConstraintIndexSimple}}
\newcommand*\rattleBondVel{\rattleVel_{\rattleConstraintIndexSimple}}
\newcommand*\rattleBondVelUnit{\hat{\rattleVel}_{\rattleConstraintIndexSimple}}
\newcommand*\rattleBondPosUnconstrained
  {\iteration{\rattleBondPos}{0}}
\newcommand*\rattleBondVelUnconstrained
  {\iteration{\rattleBondVel}{0}}
\newcommand*\rattleAtomPosUnconstrainedFirst
    {\iteration{\rattleAtomPosFirst}{0}}
\newcommand*\rattleAtomPosUnconstrainedSecond
    {\iteration{\rattleAtomPosSecond}{0}}
\newcommand*\rattleAtomVelUnconstrainedFirst
    {\iteration{\rattleAtomVelFirst}{0}}
\newcommand*\rattleAtomVelUnconstrainedSecond
    {\iteration{\rattleAtomVelSecond}{0}}
%---------------------------------Constants-------------------------------------
\newcommand*\rattleBondlength                   % fixed bond length
  {d_{\rattleConstraintIndexSimple}}
\newcommand*\rattleTol{\xi}                     % d-r/d tolerance
\newcommand*\rattleRRTol{\varepsilon}           % r(t) * r(t+dt) tolerance
\newcommand*\rattleRVTol{\rattleTol}            % v(t)/d tolerance
%-------------------------------------------------------------------------------

%%%%%%%%%%%%%%%%%%%%%%%%%%%%%%%%%%%%%%%%%%%%%%%%%%%%%%%%%%%%%%%%%%%%%%%%%%%%%%%%
%					 	                    DEFINITIONS								                     %
%%%%%%%%%%%%%%%%%%%%%%%%%%%%%%%%%%%%%%%%%%%%%%%%%%%%%%%%%%%%%%%%%%%%%%%%%%%%%%%%
%-------------------------------------------------------------------------------
\newcommand*\rattle{\textsf{RATTLE} }       % display name of algorithm
%-------------------------------------------------------------------------------
\newcommand*\rattlePos{\vec{r}}             % position vector
\newcommand*\rattleVel{\vec{v}}             % velocity vector
\newcommand*\rattleAcc{\vec{a}}             % acceleration vector
%-------------------------------------------------------------------------------
\newcommand*\rattleMassNoIndex{m}
\newcommand*\rattleForceNoIndex{\vec{F}}
\newcommand*\rattleConstraintForceNoIndex{\vec{G}}
\newcommand*\rattleConstraintNoIndex{\sigma}
\newcommand*\rattleLagrangeNoIndex{\lambda}
\newcommand*\rattleLagrangeApproxNoIndex{\gamma}
\newcommand*\rattleCorrectiveValueNoIndex{g}
%---------------------Definitions for General Formulation ----------------------
%%%%%%%%%%%%%%%%%%%%%%%%%%%%%%%%%%%%%%%%%%%%%%%%%%%%%%%%%%%%%%%%%%%%%%%%%%%%%%%%
%					 	                    DEFINITIONS								                     %
%%%%%%%%%%%%%%%%%%%%%%%%%%%%%%%%%%%%%%%%%%%%%%%%%%%%%%%%%%%%%%%%%%%%%%%%%%%%%%%%
%-------------------------------------------------------------------------------
\newcommand*\rattle{\textsf{RATTLE} }       % display name of algorithm
%-------------------------------------------------------------------------------
\newcommand*\rattlePos{\vec{r}}             % position vector
\newcommand*\rattleVel{\vec{v}}             % velocity vector
\newcommand*\rattleAcc{\vec{a}}             % acceleration vector
%-------------------------------------------------------------------------------
\newcommand*\rattleMassNoIndex{m}
\newcommand*\rattleForceNoIndex{\vec{F}}
\newcommand*\rattleConstraintForceNoIndex{\vec{G}}
\newcommand*\rattleConstraintNoIndex{\sigma}
\newcommand*\rattleLagrangeNoIndex{\lambda}
\newcommand*\rattleLagrangeApproxNoIndex{\gamma}
\newcommand*\rattleCorrectiveValueNoIndex{g}
%---------------------Definitions for General Formulation ----------------------
\input{rattle/general/definitions}
%-----------------------------------Indexing------------------------------------
\newcommand*\rattleAtomIndex{a}               % solo atom index
\newcommand*\rattleAtomIndexFirst{a}          % first index in atomic pair
\newcommand*\rattleAtomIndexSecond{b}         % second index in atomic pair
\newcommand*\rattleNAtoms{N}                  % number of atoms
\newcommand*\rattleConstraintSet{K}           % set of constraints
\newcommand*\rattleConstraintIndex            % index for constraints
  {\left(
    \rattleAtomIndexFirst,\,\rattleAtomIndexSecond
  \right) \in \rattleConstraintSet}
\newcommand*\rattleConstraintIndexSimple      % simplified index for constraints
  {\rattleAtomIndexFirst\rattleAtomIndexSecond}
\newcommand*\rattleCorrectionIndex{i}         % index for iterative correction
\newcommand*\rattleCorrectionIndexThis        % current for iterative correction
  {\rattleCorrectionIndex}
\newcommand*\rattleCorrectionIndexNext        % next for iterative correction
    {\rattleCorrectionIndex + 1}
\newcommand*\rattleCorrectionIndexLast{m}     % last step of correction
\newcommand*\rattleNCorrections{M}            % number of iterative corrections
%-----------------------------------Dynamics------------------------------------
\newcommand*\rattleMass                       % solo atomic mass
  {\rattleMassNoIndex_{\rattleAtomIndex}}
\newcommand*\rattleMassFirst                  % first atomic mass in pair
    {\rattleMassNoIndex_{\rattleAtomIndexFirst}}
\newcommand*\rattleMassSecond                 % second atomic mass in pair
    {\rattleMassNoIndex_{\rattleAtomIndexSecond}}
\newcommand*\rattleForce
  {\rattleForceNoIndex_{\rattleAtomIndex}}
\newcommand*\rattleConstraintForce            % solo constraint force
  {\rattleConstraintForceNoIndex_{\rattleAtomIndex}}
\newcommand*\rattleConstraintForceFirst       % first constraint force in pair
    {\rattleConstraintForceNoIndex_{\rattleAtomIndexFirst}}
\newcommand*\rattleConstraintForceSecond      % second constraint force in pair
    {\rattleConstraintForceNoIndex_{\rattleAtomIndexSecond}}
\newcommand*\rattleConstraintForceBond        % constraint force along bond
        {\rattleConstraintForceNoIndex_{\rattleConstraintIndexSimple}}
\newcommand*\rattleConstraint
  {\rattleConstraintNoIndex_{\rattleConstraintIndexSimple}}
  \newcommand*\rattleConstraintDot
    {\dot{\rattleConstraintNoIndex}_{\rattleConstraintIndexSimple}}
\newcommand*\rattleLagrange
    {\rattleLagrangeNoIndex_{\rattleConstraintIndexSimple}}
\newcommand*\rattleLagrangeApprox
        {\rattleLagrangeApproxNoIndex_{\rattleConstraintIndexSimple}}
\newcommand*\rattleCorrectiveValue
  {\rattleCorrectiveValueNoIndex_{\rattleConstraintIndexSimple}}
%-----------------------------IterativeCorrection-------------------------------
\newcommand*\iteration[2]{#1^{(#2)}}            % iterative correction steps
\newcommand*\iterationThis[1]                   % arg^{i}
  {\iteration{#1}{\rattleCorrectionIndexThis}}
\newcommand*\iterationNext[1]                   % arg^{i+1}
  {\iteration{#1}{\rattleCorrectionIndexNext}}
%----------------------------------Vectors--------------------------------------
\newcommand*\rattleAtomPos{\rattlePos_{\rattleAtomIndex}}
\newcommand*\rattleAtomPosFirst{\rattlePos_{\rattleAtomIndexFirst}}
\newcommand*\rattleAtomPosSecond{\rattlePos_{\rattleAtomIndexSecond}}
\newcommand*\rattleAtomVel{\rattleVel_{\rattleAtomIndex}}
\newcommand*\rattleAtomVelFirst{\rattleVel_{\rattleAtomIndexFirst}}
\newcommand*\rattleAtomVelSecond{\rattleVel_{\rattleAtomIndexSecond}}
\newcommand*\rattlePosUnconstrained             % r^{(0)}
  {\iteration{\rattlePos}{0}}
\newcommand*\rattleVelUnconstrained             % v^{(0)}
  {\iteration{\rattleVel}{0}}
\newcommand*\rattleAtomPosUnconstrained         % r^{(0)}
  {\iteration{\rattleAtomPos}{0}}
\newcommand*\rattleAtomVelUnconstrained         % v^{(0)}
  {\iteration{\rattleAtomVel}{0}}
\newcommand*\rattleBondPos{\rattlePos_{\rattleConstraintIndexSimple}}
\newcommand*\rattleBondPosUnit{\hat{\rattlePos}_{\rattleConstraintIndexSimple}}
\newcommand*\rattleBondVel{\rattleVel_{\rattleConstraintIndexSimple}}
\newcommand*\rattleBondVelUnit{\hat{\rattleVel}_{\rattleConstraintIndexSimple}}
\newcommand*\rattleBondPosUnconstrained
  {\iteration{\rattleBondPos}{0}}
\newcommand*\rattleBondVelUnconstrained
  {\iteration{\rattleBondVel}{0}}
\newcommand*\rattleAtomPosUnconstrainedFirst
    {\iteration{\rattleAtomPosFirst}{0}}
\newcommand*\rattleAtomPosUnconstrainedSecond
    {\iteration{\rattleAtomPosSecond}{0}}
\newcommand*\rattleAtomVelUnconstrainedFirst
    {\iteration{\rattleAtomVelFirst}{0}}
\newcommand*\rattleAtomVelUnconstrainedSecond
    {\iteration{\rattleAtomVelSecond}{0}}
%---------------------------------Constants-------------------------------------
\newcommand*\rattleBondlength                   % fixed bond length
  {d_{\rattleConstraintIndexSimple}}
\newcommand*\rattleTol{\xi}                     % d-r/d tolerance
\newcommand*\rattleRRTol{\varepsilon}           % r(t) * r(t+dt) tolerance
\newcommand*\rattleRVTol{\rattleTol}            % v(t)/d tolerance
%-------------------------------------------------------------------------------

%-----------------------------------Indexing------------------------------------
\newcommand*\rattleAtomIndex{a}               % solo atom index
\newcommand*\rattleAtomIndexFirst{a}          % first index in atomic pair
\newcommand*\rattleAtomIndexSecond{b}         % second index in atomic pair
\newcommand*\rattleNAtoms{N}                  % number of atoms
\newcommand*\rattleConstraintSet{K}           % set of constraints
\newcommand*\rattleConstraintIndex            % index for constraints
  {\left(
    \rattleAtomIndexFirst,\,\rattleAtomIndexSecond
  \right) \in \rattleConstraintSet}
\newcommand*\rattleConstraintIndexSimple      % simplified index for constraints
  {\rattleAtomIndexFirst\rattleAtomIndexSecond}
\newcommand*\rattleCorrectionIndex{i}         % index for iterative correction
\newcommand*\rattleCorrectionIndexThis        % current for iterative correction
  {\rattleCorrectionIndex}
\newcommand*\rattleCorrectionIndexNext        % next for iterative correction
    {\rattleCorrectionIndex + 1}
\newcommand*\rattleCorrectionIndexLast{m}     % last step of correction
\newcommand*\rattleNCorrections{M}            % number of iterative corrections
%-----------------------------------Dynamics------------------------------------
\newcommand*\rattleMass                       % solo atomic mass
  {\rattleMassNoIndex_{\rattleAtomIndex}}
\newcommand*\rattleMassFirst                  % first atomic mass in pair
    {\rattleMassNoIndex_{\rattleAtomIndexFirst}}
\newcommand*\rattleMassSecond                 % second atomic mass in pair
    {\rattleMassNoIndex_{\rattleAtomIndexSecond}}
\newcommand*\rattleForce
  {\rattleForceNoIndex_{\rattleAtomIndex}}
\newcommand*\rattleConstraintForce            % solo constraint force
  {\rattleConstraintForceNoIndex_{\rattleAtomIndex}}
\newcommand*\rattleConstraintForceFirst       % first constraint force in pair
    {\rattleConstraintForceNoIndex_{\rattleAtomIndexFirst}}
\newcommand*\rattleConstraintForceSecond      % second constraint force in pair
    {\rattleConstraintForceNoIndex_{\rattleAtomIndexSecond}}
\newcommand*\rattleConstraintForceBond        % constraint force along bond
        {\rattleConstraintForceNoIndex_{\rattleConstraintIndexSimple}}
\newcommand*\rattleConstraint
  {\rattleConstraintNoIndex_{\rattleConstraintIndexSimple}}
  \newcommand*\rattleConstraintDot
    {\dot{\rattleConstraintNoIndex}_{\rattleConstraintIndexSimple}}
\newcommand*\rattleLagrange
    {\rattleLagrangeNoIndex_{\rattleConstraintIndexSimple}}
\newcommand*\rattleLagrangeApprox
        {\rattleLagrangeApproxNoIndex_{\rattleConstraintIndexSimple}}
\newcommand*\rattleCorrectiveValue
  {\rattleCorrectiveValueNoIndex_{\rattleConstraintIndexSimple}}
%-----------------------------IterativeCorrection-------------------------------
\newcommand*\iteration[2]{#1^{(#2)}}            % iterative correction steps
\newcommand*\iterationThis[1]                   % arg^{i}
  {\iteration{#1}{\rattleCorrectionIndexThis}}
\newcommand*\iterationNext[1]                   % arg^{i+1}
  {\iteration{#1}{\rattleCorrectionIndexNext}}
%----------------------------------Vectors--------------------------------------
\newcommand*\rattleAtomPos{\rattlePos_{\rattleAtomIndex}}
\newcommand*\rattleAtomPosFirst{\rattlePos_{\rattleAtomIndexFirst}}
\newcommand*\rattleAtomPosSecond{\rattlePos_{\rattleAtomIndexSecond}}
\newcommand*\rattleAtomVel{\rattleVel_{\rattleAtomIndex}}
\newcommand*\rattleAtomVelFirst{\rattleVel_{\rattleAtomIndexFirst}}
\newcommand*\rattleAtomVelSecond{\rattleVel_{\rattleAtomIndexSecond}}
\newcommand*\rattlePosUnconstrained             % r^{(0)}
  {\iteration{\rattlePos}{0}}
\newcommand*\rattleVelUnconstrained             % v^{(0)}
  {\iteration{\rattleVel}{0}}
\newcommand*\rattleAtomPosUnconstrained         % r^{(0)}
  {\iteration{\rattleAtomPos}{0}}
\newcommand*\rattleAtomVelUnconstrained         % v^{(0)}
  {\iteration{\rattleAtomVel}{0}}
\newcommand*\rattleBondPos{\rattlePos_{\rattleConstraintIndexSimple}}
\newcommand*\rattleBondPosUnit{\hat{\rattlePos}_{\rattleConstraintIndexSimple}}
\newcommand*\rattleBondVel{\rattleVel_{\rattleConstraintIndexSimple}}
\newcommand*\rattleBondVelUnit{\hat{\rattleVel}_{\rattleConstraintIndexSimple}}
\newcommand*\rattleBondPosUnconstrained
  {\iteration{\rattleBondPos}{0}}
\newcommand*\rattleBondVelUnconstrained
  {\iteration{\rattleBondVel}{0}}
\newcommand*\rattleAtomPosUnconstrainedFirst
    {\iteration{\rattleAtomPosFirst}{0}}
\newcommand*\rattleAtomPosUnconstrainedSecond
    {\iteration{\rattleAtomPosSecond}{0}}
\newcommand*\rattleAtomVelUnconstrainedFirst
    {\iteration{\rattleAtomVelFirst}{0}}
\newcommand*\rattleAtomVelUnconstrainedSecond
    {\iteration{\rattleAtomVelSecond}{0}}
%---------------------------------Constants-------------------------------------
\newcommand*\rattleBondlength                   % fixed bond length
  {d_{\rattleConstraintIndexSimple}}
\newcommand*\rattleTol{\xi}                     % d-r/d tolerance
\newcommand*\rattleRRTol{\varepsilon}           % r(t) * r(t+dt) tolerance
\newcommand*\rattleRVTol{\rattleTol}            % v(t)/d tolerance
%-------------------------------------------------------------------------------

%-------------------------------------------------------------------------------
%\setlength{\parindent}{0pt}						          % do not indent
\numberwithin{equation}{section}                % number eqs within sections
\begin{document}
\lstset{language=Python}
%%%%%%%%%%%%%%%%%%%%%%%%%%%%%%%%%%%%%%%%%%%%%%%%%%%%%%%%%%%%%%%%%%%%%%%%%%%%%%%%
%					 	                        TITLE							                         %
%%%%%%%%%%%%%%%%%%%%%%%%%%%%%%%%%%%%%%%%%%%%%%%%%%%%%%%%%%%%%%%%%%%%%%%%%%%%%%%%
\title{Group Seminar (revised: \revisiondate)}
\thanks{Professor Stratt, Vale Cofer-Shabica, Yan Zhao, Andrew Ton, Mansheej Paul, Evan Coleman}
\author{Artur Avkhadiev}
\email{artur\_avkhadiev@brown.edu}
\affiliation{Department of Physics, Brown University, Providence RI 02912, USA}
\date{\seminardate}
%%%%%%%%%%%%%%%%%%%%%%%%%%%%%%%%%%%%%%%%%%%%%%%%%%%%%%%%%%%%%%%%%%%%%%%%%%%%%%%%
%					 	                       ABSTRACT							                       %
%%%%%%%%%%%%%%%%%%%%%%%%%%%%%%%%%%%%%%%%%%%%%%%%%%%%%%%%%%%%%%%%%%%%%%%%%%%%%%%%
\begin{abstract}
  This report consist of two parts:
  \begin{enumerate}
      \item A discussion of the constraint dynamics algorithm based on velocity Verlet numerical integration scheme, \rattle.
      \item A calculation of the kinetic energy of a freely-jointed polymer chain using the bond-vector representation in the CM frame of the molecule.
    \end{enumerate}
\end{abstract}
\maketitle
\tableofcontents						 % hide table of contents
%%%%%%%%%%%%%%%%%%%%%%%%%%%%%%%%%%%%%%%%%%%%%%%%%%%%%%%%%%%%%%%%%%%%%%%%%%%%%%%%
%					 	                        BODY							                         %
%%%%%%%%%%%%%%%%%%%%%%%%%%%%%%%%%%%%%%%%%%%%%%%%%%%%%%%%%%%%%%%%%%%%%%%%%%%%%%%%
% Add include statements below. All the text should be in the folders.
\documentclass[%
    reprint,
    superscriptaddress,
    %groupedaddress,
    %unsortedaddress,
    %runinaddress,
    %frontmatterverbose,
    %preprint,
    longbibliography,
    bibnotes,
    %showpacs,preprintnumbers,
    %nofootinbib,
    %nobibnotes,
    %bibnotes,
     amsmath,amssymb,
     %aps,
     aip,
     jcp,                                       % J. Chem. Phys
    %pra,
    %prb,
    %rmp,
    %prstab,
    %prstper,
    %floatfix,
    ]{revtex4-1}
%%%%%%%%%%%%%%%%%%%%%%%%%%%%%%%%%%%%%%%%%%%%%%%%%%%%%%%%%%%%%%%%%%%%%%%%%%%%%%%%
%					 	                       LIBRARIES							                     %
%%%%%%%%%%%%%%%%%%%%%%%%%%%%%%%%%%%%%%%%%%%%%%%%%%%%%%%%%%%%%%%%%%%%%%%%%%%%%%%%
% Add whatever libraries you need here. Add a description if you want
\usepackage{graphicx}			 % Include figure files
\usepackage{dcolumn}			 % Align table columns on decimal point
\usepackage{tcolorbox}		 % use tcolorbox environment to box equations
\usepackage[USenglish]{babel}
\usepackage[useregional]{datetime2}
\usepackage[utf8]{inputenc}
\usepackage[T1]{fontenc}
\usepackage{enumerate}
\usepackage{caption}
\usepackage{subcaption}
\usepackage{physics}
\usepackage{hyperref}
\usepackage{chemformula}
\usepackage{booktabs}
\usepackage{makecell}
\usepackage{fullpage}
\usepackage{natbib}
\usepackage{setspace}
\usepackage{amsmath}
\usepackage{bm}              % enchanced bold math symbols
\usepackage{amssymb}
\usepackage{listings} 			 % For code citations
\usepackage{amsfonts}
\usepackage{mathtools}
\usepackage{commath}
\usepackage{fancyhdr}
\usepackage{lastpage}
\usepackage{tikz}
\usepackage{graphicx}
	\graphicspath{ {images/} }
\usepackage{feynmp-auto}
\usepackage[toc,page]{appendix}
%%%%%%%%%%%%%%%%%%%%%%%%%%%%%%%%%%%%%%%%%%%%%%%%%%%%%%%%%%%%%%%%%%%%%%%%%%%%%%%%
%					 	                      NEW COMMANDS							                   %
%%%%%%%%%%%%%%%%%%%%%%%%%%%%%%%%%%%%%%%%%%%%%%%%%%%%%%%%%%%%%%%%%%%%%%%%%%%%%%%%
% Redefine commands if necessary
\renewcommand\vec{\mathbf}                      % use boldface for vectors
% New commands
%------------------------------------Date---------------------------------------
\newcommand{\seminardate}{\DTMdisplaydate{2017}{7}{25}{-1}}
\newcommand{\revisiondate}{\DTMdisplaydate{2017}{7}{26}{-1}}
%-------------------------------------------------------------------------------
\newcommand*\Del{\vec{\nabla}}                  % del operator
\newcommand*\diff{\mathop{}\!\mathrm{d}}		    % differential d
\newcommand*\Diff[1]{\mathop{}\!\mathrm{d^#1}}	% d^(...)
\newcommand*{\scalarnorm}[1]                    % scalar norm |*|
  {\left\lvert#1\right\rvert}
\newcommand*{\vectornorm}[1]                    % vector norm ||*||
  {\left\lVert#1\right\rVert}
\newcommand*\onehalf{\frac{1}{2}}               % fraction 1/2
%------------------------------------Time---------------------------------------
\newcommand*\dlt[1]{\delta #1}
\newcommand*\timestep{\dlt{t}}              % timestep dt
\newcommand*\halfstep                       % half timestep 1/2 dt
  {\onehalf \timestep}
\newcommand*\timeInHalfStep{\left(t + \halfstep\right)}
\newcommand*\timeInFullStep{\left(t + \timestep\right)}
%-------------------------------------------------------------------------------
%%%%%%%%%%%%%%%%%%%%%%%%%%%%%%%%%%%%%%%%%%%%%%%%%%%%%%%%%%%%%%%%%%%%%%%%%%%%%%%%
%					 	                    DEFINITIONS								                     %
%%%%%%%%%%%%%%%%%%%%%%%%%%%%%%%%%%%%%%%%%%%%%%%%%%%%%%%%%%%%%%%%%%%%%%%%%%%%%%%%
%-------------------------------------------------------------------------------
\newcommand*\rattle{\textsf{RATTLE} }       % display name of algorithm
%-------------------------------------------------------------------------------
\newcommand*\rattlePos{\vec{r}}             % position vector
\newcommand*\rattleVel{\vec{v}}             % velocity vector
\newcommand*\rattleAcc{\vec{a}}             % acceleration vector
%-------------------------------------------------------------------------------
\newcommand*\rattleMassNoIndex{m}
\newcommand*\rattleForceNoIndex{\vec{F}}
\newcommand*\rattleConstraintForceNoIndex{\vec{G}}
\newcommand*\rattleConstraintNoIndex{\sigma}
\newcommand*\rattleLagrangeNoIndex{\lambda}
\newcommand*\rattleLagrangeApproxNoIndex{\gamma}
\newcommand*\rattleCorrectiveValueNoIndex{g}
%---------------------Definitions for General Formulation ----------------------
\input{rattle/general/definitions}
%-----------------------------------Indexing------------------------------------
\newcommand*\rattleAtomIndex{a}               % solo atom index
\newcommand*\rattleAtomIndexFirst{a}          % first index in atomic pair
\newcommand*\rattleAtomIndexSecond{b}         % second index in atomic pair
\newcommand*\rattleNAtoms{N}                  % number of atoms
\newcommand*\rattleConstraintSet{K}           % set of constraints
\newcommand*\rattleConstraintIndex            % index for constraints
  {\left(
    \rattleAtomIndexFirst,\,\rattleAtomIndexSecond
  \right) \in \rattleConstraintSet}
\newcommand*\rattleConstraintIndexSimple      % simplified index for constraints
  {\rattleAtomIndexFirst\rattleAtomIndexSecond}
\newcommand*\rattleCorrectionIndex{i}         % index for iterative correction
\newcommand*\rattleCorrectionIndexThis        % current for iterative correction
  {\rattleCorrectionIndex}
\newcommand*\rattleCorrectionIndexNext        % next for iterative correction
    {\rattleCorrectionIndex + 1}
\newcommand*\rattleCorrectionIndexLast{m}     % last step of correction
\newcommand*\rattleNCorrections{M}            % number of iterative corrections
%-----------------------------------Dynamics------------------------------------
\newcommand*\rattleMass                       % solo atomic mass
  {\rattleMassNoIndex_{\rattleAtomIndex}}
\newcommand*\rattleMassFirst                  % first atomic mass in pair
    {\rattleMassNoIndex_{\rattleAtomIndexFirst}}
\newcommand*\rattleMassSecond                 % second atomic mass in pair
    {\rattleMassNoIndex_{\rattleAtomIndexSecond}}
\newcommand*\rattleForce
  {\rattleForceNoIndex_{\rattleAtomIndex}}
\newcommand*\rattleConstraintForce            % solo constraint force
  {\rattleConstraintForceNoIndex_{\rattleAtomIndex}}
\newcommand*\rattleConstraintForceFirst       % first constraint force in pair
    {\rattleConstraintForceNoIndex_{\rattleAtomIndexFirst}}
\newcommand*\rattleConstraintForceSecond      % second constraint force in pair
    {\rattleConstraintForceNoIndex_{\rattleAtomIndexSecond}}
\newcommand*\rattleConstraintForceBond        % constraint force along bond
        {\rattleConstraintForceNoIndex_{\rattleConstraintIndexSimple}}
\newcommand*\rattleConstraint
  {\rattleConstraintNoIndex_{\rattleConstraintIndexSimple}}
  \newcommand*\rattleConstraintDot
    {\dot{\rattleConstraintNoIndex}_{\rattleConstraintIndexSimple}}
\newcommand*\rattleLagrange
    {\rattleLagrangeNoIndex_{\rattleConstraintIndexSimple}}
\newcommand*\rattleLagrangeApprox
        {\rattleLagrangeApproxNoIndex_{\rattleConstraintIndexSimple}}
\newcommand*\rattleCorrectiveValue
  {\rattleCorrectiveValueNoIndex_{\rattleConstraintIndexSimple}}
%-----------------------------IterativeCorrection-------------------------------
\newcommand*\iteration[2]{#1^{(#2)}}            % iterative correction steps
\newcommand*\iterationThis[1]                   % arg^{i}
  {\iteration{#1}{\rattleCorrectionIndexThis}}
\newcommand*\iterationNext[1]                   % arg^{i+1}
  {\iteration{#1}{\rattleCorrectionIndexNext}}
%----------------------------------Vectors--------------------------------------
\newcommand*\rattleAtomPos{\rattlePos_{\rattleAtomIndex}}
\newcommand*\rattleAtomPosFirst{\rattlePos_{\rattleAtomIndexFirst}}
\newcommand*\rattleAtomPosSecond{\rattlePos_{\rattleAtomIndexSecond}}
\newcommand*\rattleAtomVel{\rattleVel_{\rattleAtomIndex}}
\newcommand*\rattleAtomVelFirst{\rattleVel_{\rattleAtomIndexFirst}}
\newcommand*\rattleAtomVelSecond{\rattleVel_{\rattleAtomIndexSecond}}
\newcommand*\rattlePosUnconstrained             % r^{(0)}
  {\iteration{\rattlePos}{0}}
\newcommand*\rattleVelUnconstrained             % v^{(0)}
  {\iteration{\rattleVel}{0}}
\newcommand*\rattleAtomPosUnconstrained         % r^{(0)}
  {\iteration{\rattleAtomPos}{0}}
\newcommand*\rattleAtomVelUnconstrained         % v^{(0)}
  {\iteration{\rattleAtomVel}{0}}
\newcommand*\rattleBondPos{\rattlePos_{\rattleConstraintIndexSimple}}
\newcommand*\rattleBondPosUnit{\hat{\rattlePos}_{\rattleConstraintIndexSimple}}
\newcommand*\rattleBondVel{\rattleVel_{\rattleConstraintIndexSimple}}
\newcommand*\rattleBondVelUnit{\hat{\rattleVel}_{\rattleConstraintIndexSimple}}
\newcommand*\rattleBondPosUnconstrained
  {\iteration{\rattleBondPos}{0}}
\newcommand*\rattleBondVelUnconstrained
  {\iteration{\rattleBondVel}{0}}
\newcommand*\rattleAtomPosUnconstrainedFirst
    {\iteration{\rattleAtomPosFirst}{0}}
\newcommand*\rattleAtomPosUnconstrainedSecond
    {\iteration{\rattleAtomPosSecond}{0}}
\newcommand*\rattleAtomVelUnconstrainedFirst
    {\iteration{\rattleAtomVelFirst}{0}}
\newcommand*\rattleAtomVelUnconstrainedSecond
    {\iteration{\rattleAtomVelSecond}{0}}
%---------------------------------Constants-------------------------------------
\newcommand*\rattleBondlength                   % fixed bond length
  {d_{\rattleConstraintIndexSimple}}
\newcommand*\rattleTol{\xi}                     % d-r/d tolerance
\newcommand*\rattleRRTol{\varepsilon}           % r(t) * r(t+dt) tolerance
\newcommand*\rattleRVTol{\rattleTol}            % v(t)/d tolerance
%-------------------------------------------------------------------------------

%%%%%%%%%%%%%%%%%%%%%%%%%%%%%%%%%%%%%%%%%%%%%%%%%%%%%%%%%%%%%%%%%%%%%%%%%%%%%%%%
%					 	                    DEFINITIONS								                     %
%%%%%%%%%%%%%%%%%%%%%%%%%%%%%%%%%%%%%%%%%%%%%%%%%%%%%%%%%%%%%%%%%%%%%%%%%%%%%%%%
%-------------------------------------------------------------------------------
\newcommand*\rattle{\textsf{RATTLE} }       % display name of algorithm
%-------------------------------------------------------------------------------
\newcommand*\rattlePos{\vec{r}}             % position vector
\newcommand*\rattleVel{\vec{v}}             % velocity vector
\newcommand*\rattleAcc{\vec{a}}             % acceleration vector
%-------------------------------------------------------------------------------
\newcommand*\rattleMassNoIndex{m}
\newcommand*\rattleForceNoIndex{\vec{F}}
\newcommand*\rattleConstraintForceNoIndex{\vec{G}}
\newcommand*\rattleConstraintNoIndex{\sigma}
\newcommand*\rattleLagrangeNoIndex{\lambda}
\newcommand*\rattleLagrangeApproxNoIndex{\gamma}
\newcommand*\rattleCorrectiveValueNoIndex{g}
%---------------------Definitions for General Formulation ----------------------
\input{rattle/general/definitions}
%-----------------------------------Indexing------------------------------------
\newcommand*\rattleAtomIndex{a}               % solo atom index
\newcommand*\rattleAtomIndexFirst{a}          % first index in atomic pair
\newcommand*\rattleAtomIndexSecond{b}         % second index in atomic pair
\newcommand*\rattleNAtoms{N}                  % number of atoms
\newcommand*\rattleConstraintSet{K}           % set of constraints
\newcommand*\rattleConstraintIndex            % index for constraints
  {\left(
    \rattleAtomIndexFirst,\,\rattleAtomIndexSecond
  \right) \in \rattleConstraintSet}
\newcommand*\rattleConstraintIndexSimple      % simplified index for constraints
  {\rattleAtomIndexFirst\rattleAtomIndexSecond}
\newcommand*\rattleCorrectionIndex{i}         % index for iterative correction
\newcommand*\rattleCorrectionIndexThis        % current for iterative correction
  {\rattleCorrectionIndex}
\newcommand*\rattleCorrectionIndexNext        % next for iterative correction
    {\rattleCorrectionIndex + 1}
\newcommand*\rattleCorrectionIndexLast{m}     % last step of correction
\newcommand*\rattleNCorrections{M}            % number of iterative corrections
%-----------------------------------Dynamics------------------------------------
\newcommand*\rattleMass                       % solo atomic mass
  {\rattleMassNoIndex_{\rattleAtomIndex}}
\newcommand*\rattleMassFirst                  % first atomic mass in pair
    {\rattleMassNoIndex_{\rattleAtomIndexFirst}}
\newcommand*\rattleMassSecond                 % second atomic mass in pair
    {\rattleMassNoIndex_{\rattleAtomIndexSecond}}
\newcommand*\rattleForce
  {\rattleForceNoIndex_{\rattleAtomIndex}}
\newcommand*\rattleConstraintForce            % solo constraint force
  {\rattleConstraintForceNoIndex_{\rattleAtomIndex}}
\newcommand*\rattleConstraintForceFirst       % first constraint force in pair
    {\rattleConstraintForceNoIndex_{\rattleAtomIndexFirst}}
\newcommand*\rattleConstraintForceSecond      % second constraint force in pair
    {\rattleConstraintForceNoIndex_{\rattleAtomIndexSecond}}
\newcommand*\rattleConstraintForceBond        % constraint force along bond
        {\rattleConstraintForceNoIndex_{\rattleConstraintIndexSimple}}
\newcommand*\rattleConstraint
  {\rattleConstraintNoIndex_{\rattleConstraintIndexSimple}}
  \newcommand*\rattleConstraintDot
    {\dot{\rattleConstraintNoIndex}_{\rattleConstraintIndexSimple}}
\newcommand*\rattleLagrange
    {\rattleLagrangeNoIndex_{\rattleConstraintIndexSimple}}
\newcommand*\rattleLagrangeApprox
        {\rattleLagrangeApproxNoIndex_{\rattleConstraintIndexSimple}}
\newcommand*\rattleCorrectiveValue
  {\rattleCorrectiveValueNoIndex_{\rattleConstraintIndexSimple}}
%-----------------------------IterativeCorrection-------------------------------
\newcommand*\iteration[2]{#1^{(#2)}}            % iterative correction steps
\newcommand*\iterationThis[1]                   % arg^{i}
  {\iteration{#1}{\rattleCorrectionIndexThis}}
\newcommand*\iterationNext[1]                   % arg^{i+1}
  {\iteration{#1}{\rattleCorrectionIndexNext}}
%----------------------------------Vectors--------------------------------------
\newcommand*\rattleAtomPos{\rattlePos_{\rattleAtomIndex}}
\newcommand*\rattleAtomPosFirst{\rattlePos_{\rattleAtomIndexFirst}}
\newcommand*\rattleAtomPosSecond{\rattlePos_{\rattleAtomIndexSecond}}
\newcommand*\rattleAtomVel{\rattleVel_{\rattleAtomIndex}}
\newcommand*\rattleAtomVelFirst{\rattleVel_{\rattleAtomIndexFirst}}
\newcommand*\rattleAtomVelSecond{\rattleVel_{\rattleAtomIndexSecond}}
\newcommand*\rattlePosUnconstrained             % r^{(0)}
  {\iteration{\rattlePos}{0}}
\newcommand*\rattleVelUnconstrained             % v^{(0)}
  {\iteration{\rattleVel}{0}}
\newcommand*\rattleAtomPosUnconstrained         % r^{(0)}
  {\iteration{\rattleAtomPos}{0}}
\newcommand*\rattleAtomVelUnconstrained         % v^{(0)}
  {\iteration{\rattleAtomVel}{0}}
\newcommand*\rattleBondPos{\rattlePos_{\rattleConstraintIndexSimple}}
\newcommand*\rattleBondPosUnit{\hat{\rattlePos}_{\rattleConstraintIndexSimple}}
\newcommand*\rattleBondVel{\rattleVel_{\rattleConstraintIndexSimple}}
\newcommand*\rattleBondVelUnit{\hat{\rattleVel}_{\rattleConstraintIndexSimple}}
\newcommand*\rattleBondPosUnconstrained
  {\iteration{\rattleBondPos}{0}}
\newcommand*\rattleBondVelUnconstrained
  {\iteration{\rattleBondVel}{0}}
\newcommand*\rattleAtomPosUnconstrainedFirst
    {\iteration{\rattleAtomPosFirst}{0}}
\newcommand*\rattleAtomPosUnconstrainedSecond
    {\iteration{\rattleAtomPosSecond}{0}}
\newcommand*\rattleAtomVelUnconstrainedFirst
    {\iteration{\rattleAtomVelFirst}{0}}
\newcommand*\rattleAtomVelUnconstrainedSecond
    {\iteration{\rattleAtomVelSecond}{0}}
%---------------------------------Constants-------------------------------------
\newcommand*\rattleBondlength                   % fixed bond length
  {d_{\rattleConstraintIndexSimple}}
\newcommand*\rattleTol{\xi}                     % d-r/d tolerance
\newcommand*\rattleRRTol{\varepsilon}           % r(t) * r(t+dt) tolerance
\newcommand*\rattleRVTol{\rattleTol}            % v(t)/d tolerance
%-------------------------------------------------------------------------------

%-------------------------------------------------------------------------------
%\setlength{\parindent}{0pt}						          % do not indent
\numberwithin{equation}{section}                % number eqs within sections
\begin{document}
\lstset{language=Python}
%%%%%%%%%%%%%%%%%%%%%%%%%%%%%%%%%%%%%%%%%%%%%%%%%%%%%%%%%%%%%%%%%%%%%%%%%%%%%%%%
%					 	                        TITLE							                         %
%%%%%%%%%%%%%%%%%%%%%%%%%%%%%%%%%%%%%%%%%%%%%%%%%%%%%%%%%%%%%%%%%%%%%%%%%%%%%%%%
\title{Group Seminar (revised: \revisiondate)}
\thanks{Professor Stratt, Vale Cofer-Shabica, Yan Zhao, Andrew Ton, Mansheej Paul, Evan Coleman}
\author{Artur Avkhadiev}
\email{artur\_avkhadiev@brown.edu}
\affiliation{Department of Physics, Brown University, Providence RI 02912, USA}
\date{\seminardate}
%%%%%%%%%%%%%%%%%%%%%%%%%%%%%%%%%%%%%%%%%%%%%%%%%%%%%%%%%%%%%%%%%%%%%%%%%%%%%%%%
%					 	                       ABSTRACT							                       %
%%%%%%%%%%%%%%%%%%%%%%%%%%%%%%%%%%%%%%%%%%%%%%%%%%%%%%%%%%%%%%%%%%%%%%%%%%%%%%%%
\begin{abstract}
  This report consist of two parts:
  \begin{enumerate}
      \item A discussion of the constraint dynamics algorithm based on velocity Verlet numerical integration scheme, \rattle.
      \item A calculation of the kinetic energy of a freely-jointed polymer chain using the bond-vector representation in the CM frame of the molecule.
    \end{enumerate}
\end{abstract}
\maketitle
\tableofcontents						 % hide table of contents
%%%%%%%%%%%%%%%%%%%%%%%%%%%%%%%%%%%%%%%%%%%%%%%%%%%%%%%%%%%%%%%%%%%%%%%%%%%%%%%%
%					 	                        BODY							                         %
%%%%%%%%%%%%%%%%%%%%%%%%%%%%%%%%%%%%%%%%%%%%%%%%%%%%%%%%%%%%%%%%%%%%%%%%%%%%%%%%
% Add include statements below. All the text should be in the folders.
\documentclass[%
    reprint,
    superscriptaddress,
    %groupedaddress,
    %unsortedaddress,
    %runinaddress,
    %frontmatterverbose,
    %preprint,
    longbibliography,
    bibnotes,
    %showpacs,preprintnumbers,
    %nofootinbib,
    %nobibnotes,
    %bibnotes,
     amsmath,amssymb,
     %aps,
     aip,
     jcp,                                       % J. Chem. Phys
    %pra,
    %prb,
    %rmp,
    %prstab,
    %prstper,
    %floatfix,
    ]{revtex4-1}
%%%%%%%%%%%%%%%%%%%%%%%%%%%%%%%%%%%%%%%%%%%%%%%%%%%%%%%%%%%%%%%%%%%%%%%%%%%%%%%%
%					 	                       LIBRARIES							                     %
%%%%%%%%%%%%%%%%%%%%%%%%%%%%%%%%%%%%%%%%%%%%%%%%%%%%%%%%%%%%%%%%%%%%%%%%%%%%%%%%
% Add whatever libraries you need here. Add a description if you want
\usepackage{graphicx}			 % Include figure files
\usepackage{dcolumn}			 % Align table columns on decimal point
\usepackage{tcolorbox}		 % use tcolorbox environment to box equations
\usepackage[USenglish]{babel}
\usepackage[useregional]{datetime2}
\usepackage[utf8]{inputenc}
\usepackage[T1]{fontenc}
\usepackage{enumerate}
\usepackage{caption}
\usepackage{subcaption}
\usepackage{physics}
\usepackage{hyperref}
\usepackage{chemformula}
\usepackage{booktabs}
\usepackage{makecell}
\usepackage{fullpage}
\usepackage{natbib}
\usepackage{setspace}
\usepackage{amsmath}
\usepackage{bm}              % enchanced bold math symbols
\usepackage{amssymb}
\usepackage{listings} 			 % For code citations
\usepackage{amsfonts}
\usepackage{mathtools}
\usepackage{commath}
\usepackage{fancyhdr}
\usepackage{lastpage}
\usepackage{tikz}
\usepackage{graphicx}
	\graphicspath{ {images/} }
\usepackage{feynmp-auto}
\usepackage[toc,page]{appendix}
%%%%%%%%%%%%%%%%%%%%%%%%%%%%%%%%%%%%%%%%%%%%%%%%%%%%%%%%%%%%%%%%%%%%%%%%%%%%%%%%
%					 	                      NEW COMMANDS							                   %
%%%%%%%%%%%%%%%%%%%%%%%%%%%%%%%%%%%%%%%%%%%%%%%%%%%%%%%%%%%%%%%%%%%%%%%%%%%%%%%%
% Redefine commands if necessary
\renewcommand\vec{\mathbf}                      % use boldface for vectors
% New commands
%------------------------------------Date---------------------------------------
\newcommand{\seminardate}{\DTMdisplaydate{2017}{7}{25}{-1}}
\newcommand{\revisiondate}{\DTMdisplaydate{2017}{7}{26}{-1}}
%-------------------------------------------------------------------------------
\newcommand*\Del{\vec{\nabla}}                  % del operator
\newcommand*\diff{\mathop{}\!\mathrm{d}}		    % differential d
\newcommand*\Diff[1]{\mathop{}\!\mathrm{d^#1}}	% d^(...)
\newcommand*{\scalarnorm}[1]                    % scalar norm |*|
  {\left\lvert#1\right\rvert}
\newcommand*{\vectornorm}[1]                    % vector norm ||*||
  {\left\lVert#1\right\rVert}
\newcommand*\onehalf{\frac{1}{2}}               % fraction 1/2
%------------------------------------Time---------------------------------------
\newcommand*\dlt[1]{\delta #1}
\newcommand*\timestep{\dlt{t}}              % timestep dt
\newcommand*\halfstep                       % half timestep 1/2 dt
  {\onehalf \timestep}
\newcommand*\timeInHalfStep{\left(t + \halfstep\right)}
\newcommand*\timeInFullStep{\left(t + \timestep\right)}
%-------------------------------------------------------------------------------
\input{rattle/definitions}
\input{ke_polymer/definitions}
%-------------------------------------------------------------------------------
%\setlength{\parindent}{0pt}						          % do not indent
\numberwithin{equation}{section}                % number eqs within sections
\begin{document}
\lstset{language=Python}
%%%%%%%%%%%%%%%%%%%%%%%%%%%%%%%%%%%%%%%%%%%%%%%%%%%%%%%%%%%%%%%%%%%%%%%%%%%%%%%%
%					 	                        TITLE							                         %
%%%%%%%%%%%%%%%%%%%%%%%%%%%%%%%%%%%%%%%%%%%%%%%%%%%%%%%%%%%%%%%%%%%%%%%%%%%%%%%%
\title{Group Seminar (revised: \revisiondate)}
\thanks{Professor Stratt, Vale Cofer-Shabica, Yan Zhao, Andrew Ton, Mansheej Paul, Evan Coleman}
\author{Artur Avkhadiev}
\email{artur\_avkhadiev@brown.edu}
\affiliation{Department of Physics, Brown University, Providence RI 02912, USA}
\date{\seminardate}
%%%%%%%%%%%%%%%%%%%%%%%%%%%%%%%%%%%%%%%%%%%%%%%%%%%%%%%%%%%%%%%%%%%%%%%%%%%%%%%%
%					 	                       ABSTRACT							                       %
%%%%%%%%%%%%%%%%%%%%%%%%%%%%%%%%%%%%%%%%%%%%%%%%%%%%%%%%%%%%%%%%%%%%%%%%%%%%%%%%
\begin{abstract}
  This report consist of two parts:
  \begin{enumerate}
      \item A discussion of the constraint dynamics algorithm based on velocity Verlet numerical integration scheme, \rattle.
      \item A calculation of the kinetic energy of a freely-jointed polymer chain using the bond-vector representation in the CM frame of the molecule.
    \end{enumerate}
\end{abstract}
\maketitle
\tableofcontents						 % hide table of contents
%%%%%%%%%%%%%%%%%%%%%%%%%%%%%%%%%%%%%%%%%%%%%%%%%%%%%%%%%%%%%%%%%%%%%%%%%%%%%%%%
%					 	                        BODY							                         %
%%%%%%%%%%%%%%%%%%%%%%%%%%%%%%%%%%%%%%%%%%%%%%%%%%%%%%%%%%%%%%%%%%%%%%%%%%%%%%%%
% Add include statements below. All the text should be in the folders.
\input{rattle/main}
\input{ke_polymer/main}
%-------------------------------------------------------------------------------
%%%%%%%%%%%%%%%%%%%%%%%%%%%%%%%%%%%%%%%%%%%%%%%%%%%%%%%%%%%%%%%%%%%%%%%%%%%%%%%%
%					                        BIBLIOGRAPHY							                   %
%%%%%%%%%%%%%%%%%%%%%%%%%%%%%%%%%%%%%%%%%%%%%%%%%%%%%%%%%%%%%%%%%%%%%%%%%%%%%%%%
% The bibliography file is reference,bib. If you need to add a
% reference, make sure all the entries look good (e.g., ensure
% proper capitalization by putting {} around the letters to be
% capitalized. Reference the books by chapters!
\nocite{*}
\bibliographystyle{apalike}
\bibliography{reference}
%-----------------------------
%%%%%%%%%%%%%%%%%%%%%%%%%%%%%%%%%%%%%%%%%%%%%%%%%%%%%%%%%%%%%%%%%%%%%%%%%%%%%%%%
%					 	                      APPENDICES							                     %
%%%%%%%%%%%%%%%%%%%%%%%%%%%%%%%%%%%%%%%%%%%%%%%%%%%%%%%%%%%%%%%%%%%%%%%%%%%%%%%%
% Add all appendices in appendices/main.tex with include statements.
% see appendices/sub-example-app.tex for an example.
\onecolumngrid
\input{appendices/main}
%-------------------------------------------------------------------------------
\end{document}

\documentclass[%
    reprint,
    superscriptaddress,
    %groupedaddress,
    %unsortedaddress,
    %runinaddress,
    %frontmatterverbose,
    %preprint,
    longbibliography,
    bibnotes,
    %showpacs,preprintnumbers,
    %nofootinbib,
    %nobibnotes,
    %bibnotes,
     amsmath,amssymb,
     %aps,
     aip,
     jcp,                                       % J. Chem. Phys
    %pra,
    %prb,
    %rmp,
    %prstab,
    %prstper,
    %floatfix,
    ]{revtex4-1}
%%%%%%%%%%%%%%%%%%%%%%%%%%%%%%%%%%%%%%%%%%%%%%%%%%%%%%%%%%%%%%%%%%%%%%%%%%%%%%%%
%					 	                       LIBRARIES							                     %
%%%%%%%%%%%%%%%%%%%%%%%%%%%%%%%%%%%%%%%%%%%%%%%%%%%%%%%%%%%%%%%%%%%%%%%%%%%%%%%%
% Add whatever libraries you need here. Add a description if you want
\usepackage{graphicx}			 % Include figure files
\usepackage{dcolumn}			 % Align table columns on decimal point
\usepackage{tcolorbox}		 % use tcolorbox environment to box equations
\usepackage[USenglish]{babel}
\usepackage[useregional]{datetime2}
\usepackage[utf8]{inputenc}
\usepackage[T1]{fontenc}
\usepackage{enumerate}
\usepackage{caption}
\usepackage{subcaption}
\usepackage{physics}
\usepackage{hyperref}
\usepackage{chemformula}
\usepackage{booktabs}
\usepackage{makecell}
\usepackage{fullpage}
\usepackage{natbib}
\usepackage{setspace}
\usepackage{amsmath}
\usepackage{bm}              % enchanced bold math symbols
\usepackage{amssymb}
\usepackage{listings} 			 % For code citations
\usepackage{amsfonts}
\usepackage{mathtools}
\usepackage{commath}
\usepackage{fancyhdr}
\usepackage{lastpage}
\usepackage{tikz}
\usepackage{graphicx}
	\graphicspath{ {images/} }
\usepackage{feynmp-auto}
\usepackage[toc,page]{appendix}
%%%%%%%%%%%%%%%%%%%%%%%%%%%%%%%%%%%%%%%%%%%%%%%%%%%%%%%%%%%%%%%%%%%%%%%%%%%%%%%%
%					 	                      NEW COMMANDS							                   %
%%%%%%%%%%%%%%%%%%%%%%%%%%%%%%%%%%%%%%%%%%%%%%%%%%%%%%%%%%%%%%%%%%%%%%%%%%%%%%%%
% Redefine commands if necessary
\renewcommand\vec{\mathbf}                      % use boldface for vectors
% New commands
%------------------------------------Date---------------------------------------
\newcommand{\seminardate}{\DTMdisplaydate{2017}{7}{25}{-1}}
\newcommand{\revisiondate}{\DTMdisplaydate{2017}{7}{26}{-1}}
%-------------------------------------------------------------------------------
\newcommand*\Del{\vec{\nabla}}                  % del operator
\newcommand*\diff{\mathop{}\!\mathrm{d}}		    % differential d
\newcommand*\Diff[1]{\mathop{}\!\mathrm{d^#1}}	% d^(...)
\newcommand*{\scalarnorm}[1]                    % scalar norm |*|
  {\left\lvert#1\right\rvert}
\newcommand*{\vectornorm}[1]                    % vector norm ||*||
  {\left\lVert#1\right\rVert}
\newcommand*\onehalf{\frac{1}{2}}               % fraction 1/2
%------------------------------------Time---------------------------------------
\newcommand*\dlt[1]{\delta #1}
\newcommand*\timestep{\dlt{t}}              % timestep dt
\newcommand*\halfstep                       % half timestep 1/2 dt
  {\onehalf \timestep}
\newcommand*\timeInHalfStep{\left(t + \halfstep\right)}
\newcommand*\timeInFullStep{\left(t + \timestep\right)}
%-------------------------------------------------------------------------------
\input{rattle/definitions}
\input{ke_polymer/definitions}
%-------------------------------------------------------------------------------
%\setlength{\parindent}{0pt}						          % do not indent
\numberwithin{equation}{section}                % number eqs within sections
\begin{document}
\lstset{language=Python}
%%%%%%%%%%%%%%%%%%%%%%%%%%%%%%%%%%%%%%%%%%%%%%%%%%%%%%%%%%%%%%%%%%%%%%%%%%%%%%%%
%					 	                        TITLE							                         %
%%%%%%%%%%%%%%%%%%%%%%%%%%%%%%%%%%%%%%%%%%%%%%%%%%%%%%%%%%%%%%%%%%%%%%%%%%%%%%%%
\title{Group Seminar (revised: \revisiondate)}
\thanks{Professor Stratt, Vale Cofer-Shabica, Yan Zhao, Andrew Ton, Mansheej Paul, Evan Coleman}
\author{Artur Avkhadiev}
\email{artur\_avkhadiev@brown.edu}
\affiliation{Department of Physics, Brown University, Providence RI 02912, USA}
\date{\seminardate}
%%%%%%%%%%%%%%%%%%%%%%%%%%%%%%%%%%%%%%%%%%%%%%%%%%%%%%%%%%%%%%%%%%%%%%%%%%%%%%%%
%					 	                       ABSTRACT							                       %
%%%%%%%%%%%%%%%%%%%%%%%%%%%%%%%%%%%%%%%%%%%%%%%%%%%%%%%%%%%%%%%%%%%%%%%%%%%%%%%%
\begin{abstract}
  This report consist of two parts:
  \begin{enumerate}
      \item A discussion of the constraint dynamics algorithm based on velocity Verlet numerical integration scheme, \rattle.
      \item A calculation of the kinetic energy of a freely-jointed polymer chain using the bond-vector representation in the CM frame of the molecule.
    \end{enumerate}
\end{abstract}
\maketitle
\tableofcontents						 % hide table of contents
%%%%%%%%%%%%%%%%%%%%%%%%%%%%%%%%%%%%%%%%%%%%%%%%%%%%%%%%%%%%%%%%%%%%%%%%%%%%%%%%
%					 	                        BODY							                         %
%%%%%%%%%%%%%%%%%%%%%%%%%%%%%%%%%%%%%%%%%%%%%%%%%%%%%%%%%%%%%%%%%%%%%%%%%%%%%%%%
% Add include statements below. All the text should be in the folders.
\input{rattle/main}
\input{ke_polymer/main}
%-------------------------------------------------------------------------------
%%%%%%%%%%%%%%%%%%%%%%%%%%%%%%%%%%%%%%%%%%%%%%%%%%%%%%%%%%%%%%%%%%%%%%%%%%%%%%%%
%					                        BIBLIOGRAPHY							                   %
%%%%%%%%%%%%%%%%%%%%%%%%%%%%%%%%%%%%%%%%%%%%%%%%%%%%%%%%%%%%%%%%%%%%%%%%%%%%%%%%
% The bibliography file is reference,bib. If you need to add a
% reference, make sure all the entries look good (e.g., ensure
% proper capitalization by putting {} around the letters to be
% capitalized. Reference the books by chapters!
\nocite{*}
\bibliographystyle{apalike}
\bibliography{reference}
%-----------------------------
%%%%%%%%%%%%%%%%%%%%%%%%%%%%%%%%%%%%%%%%%%%%%%%%%%%%%%%%%%%%%%%%%%%%%%%%%%%%%%%%
%					 	                      APPENDICES							                     %
%%%%%%%%%%%%%%%%%%%%%%%%%%%%%%%%%%%%%%%%%%%%%%%%%%%%%%%%%%%%%%%%%%%%%%%%%%%%%%%%
% Add all appendices in appendices/main.tex with include statements.
% see appendices/sub-example-app.tex for an example.
\onecolumngrid
\input{appendices/main}
%-------------------------------------------------------------------------------
\end{document}

%-------------------------------------------------------------------------------
%%%%%%%%%%%%%%%%%%%%%%%%%%%%%%%%%%%%%%%%%%%%%%%%%%%%%%%%%%%%%%%%%%%%%%%%%%%%%%%%
%					                        BIBLIOGRAPHY							                   %
%%%%%%%%%%%%%%%%%%%%%%%%%%%%%%%%%%%%%%%%%%%%%%%%%%%%%%%%%%%%%%%%%%%%%%%%%%%%%%%%
% The bibliography file is reference,bib. If you need to add a
% reference, make sure all the entries look good (e.g., ensure
% proper capitalization by putting {} around the letters to be
% capitalized. Reference the books by chapters!
\nocite{*}
\bibliographystyle{apalike}
\bibliography{reference}
%-----------------------------
%%%%%%%%%%%%%%%%%%%%%%%%%%%%%%%%%%%%%%%%%%%%%%%%%%%%%%%%%%%%%%%%%%%%%%%%%%%%%%%%
%					 	                      APPENDICES							                     %
%%%%%%%%%%%%%%%%%%%%%%%%%%%%%%%%%%%%%%%%%%%%%%%%%%%%%%%%%%%%%%%%%%%%%%%%%%%%%%%%
% Add all appendices in appendices/main.tex with include statements.
% see appendices/sub-example-app.tex for an example.
\onecolumngrid
\documentclass[%
    reprint,
    superscriptaddress,
    %groupedaddress,
    %unsortedaddress,
    %runinaddress,
    %frontmatterverbose,
    %preprint,
    longbibliography,
    bibnotes,
    %showpacs,preprintnumbers,
    %nofootinbib,
    %nobibnotes,
    %bibnotes,
     amsmath,amssymb,
     %aps,
     aip,
     jcp,                                       % J. Chem. Phys
    %pra,
    %prb,
    %rmp,
    %prstab,
    %prstper,
    %floatfix,
    ]{revtex4-1}
%%%%%%%%%%%%%%%%%%%%%%%%%%%%%%%%%%%%%%%%%%%%%%%%%%%%%%%%%%%%%%%%%%%%%%%%%%%%%%%%
%					 	                       LIBRARIES							                     %
%%%%%%%%%%%%%%%%%%%%%%%%%%%%%%%%%%%%%%%%%%%%%%%%%%%%%%%%%%%%%%%%%%%%%%%%%%%%%%%%
% Add whatever libraries you need here. Add a description if you want
\usepackage{graphicx}			 % Include figure files
\usepackage{dcolumn}			 % Align table columns on decimal point
\usepackage{tcolorbox}		 % use tcolorbox environment to box equations
\usepackage[USenglish]{babel}
\usepackage[useregional]{datetime2}
\usepackage[utf8]{inputenc}
\usepackage[T1]{fontenc}
\usepackage{enumerate}
\usepackage{caption}
\usepackage{subcaption}
\usepackage{physics}
\usepackage{hyperref}
\usepackage{chemformula}
\usepackage{booktabs}
\usepackage{makecell}
\usepackage{fullpage}
\usepackage{natbib}
\usepackage{setspace}
\usepackage{amsmath}
\usepackage{bm}              % enchanced bold math symbols
\usepackage{amssymb}
\usepackage{listings} 			 % For code citations
\usepackage{amsfonts}
\usepackage{mathtools}
\usepackage{commath}
\usepackage{fancyhdr}
\usepackage{lastpage}
\usepackage{tikz}
\usepackage{graphicx}
	\graphicspath{ {images/} }
\usepackage{feynmp-auto}
\usepackage[toc,page]{appendix}
%%%%%%%%%%%%%%%%%%%%%%%%%%%%%%%%%%%%%%%%%%%%%%%%%%%%%%%%%%%%%%%%%%%%%%%%%%%%%%%%
%					 	                      NEW COMMANDS							                   %
%%%%%%%%%%%%%%%%%%%%%%%%%%%%%%%%%%%%%%%%%%%%%%%%%%%%%%%%%%%%%%%%%%%%%%%%%%%%%%%%
% Redefine commands if necessary
\renewcommand\vec{\mathbf}                      % use boldface for vectors
% New commands
%------------------------------------Date---------------------------------------
\newcommand{\seminardate}{\DTMdisplaydate{2017}{7}{25}{-1}}
\newcommand{\revisiondate}{\DTMdisplaydate{2017}{7}{26}{-1}}
%-------------------------------------------------------------------------------
\newcommand*\Del{\vec{\nabla}}                  % del operator
\newcommand*\diff{\mathop{}\!\mathrm{d}}		    % differential d
\newcommand*\Diff[1]{\mathop{}\!\mathrm{d^#1}}	% d^(...)
\newcommand*{\scalarnorm}[1]                    % scalar norm |*|
  {\left\lvert#1\right\rvert}
\newcommand*{\vectornorm}[1]                    % vector norm ||*||
  {\left\lVert#1\right\rVert}
\newcommand*\onehalf{\frac{1}{2}}               % fraction 1/2
%------------------------------------Time---------------------------------------
\newcommand*\dlt[1]{\delta #1}
\newcommand*\timestep{\dlt{t}}              % timestep dt
\newcommand*\halfstep                       % half timestep 1/2 dt
  {\onehalf \timestep}
\newcommand*\timeInHalfStep{\left(t + \halfstep\right)}
\newcommand*\timeInFullStep{\left(t + \timestep\right)}
%-------------------------------------------------------------------------------
\input{rattle/definitions}
\input{ke_polymer/definitions}
%-------------------------------------------------------------------------------
%\setlength{\parindent}{0pt}						          % do not indent
\numberwithin{equation}{section}                % number eqs within sections
\begin{document}
\lstset{language=Python}
%%%%%%%%%%%%%%%%%%%%%%%%%%%%%%%%%%%%%%%%%%%%%%%%%%%%%%%%%%%%%%%%%%%%%%%%%%%%%%%%
%					 	                        TITLE							                         %
%%%%%%%%%%%%%%%%%%%%%%%%%%%%%%%%%%%%%%%%%%%%%%%%%%%%%%%%%%%%%%%%%%%%%%%%%%%%%%%%
\title{Group Seminar (revised: \revisiondate)}
\thanks{Professor Stratt, Vale Cofer-Shabica, Yan Zhao, Andrew Ton, Mansheej Paul, Evan Coleman}
\author{Artur Avkhadiev}
\email{artur\_avkhadiev@brown.edu}
\affiliation{Department of Physics, Brown University, Providence RI 02912, USA}
\date{\seminardate}
%%%%%%%%%%%%%%%%%%%%%%%%%%%%%%%%%%%%%%%%%%%%%%%%%%%%%%%%%%%%%%%%%%%%%%%%%%%%%%%%
%					 	                       ABSTRACT							                       %
%%%%%%%%%%%%%%%%%%%%%%%%%%%%%%%%%%%%%%%%%%%%%%%%%%%%%%%%%%%%%%%%%%%%%%%%%%%%%%%%
\begin{abstract}
  This report consist of two parts:
  \begin{enumerate}
      \item A discussion of the constraint dynamics algorithm based on velocity Verlet numerical integration scheme, \rattle.
      \item A calculation of the kinetic energy of a freely-jointed polymer chain using the bond-vector representation in the CM frame of the molecule.
    \end{enumerate}
\end{abstract}
\maketitle
\tableofcontents						 % hide table of contents
%%%%%%%%%%%%%%%%%%%%%%%%%%%%%%%%%%%%%%%%%%%%%%%%%%%%%%%%%%%%%%%%%%%%%%%%%%%%%%%%
%					 	                        BODY							                         %
%%%%%%%%%%%%%%%%%%%%%%%%%%%%%%%%%%%%%%%%%%%%%%%%%%%%%%%%%%%%%%%%%%%%%%%%%%%%%%%%
% Add include statements below. All the text should be in the folders.
\input{rattle/main}
\input{ke_polymer/main}
%-------------------------------------------------------------------------------
%%%%%%%%%%%%%%%%%%%%%%%%%%%%%%%%%%%%%%%%%%%%%%%%%%%%%%%%%%%%%%%%%%%%%%%%%%%%%%%%
%					                        BIBLIOGRAPHY							                   %
%%%%%%%%%%%%%%%%%%%%%%%%%%%%%%%%%%%%%%%%%%%%%%%%%%%%%%%%%%%%%%%%%%%%%%%%%%%%%%%%
% The bibliography file is reference,bib. If you need to add a
% reference, make sure all the entries look good (e.g., ensure
% proper capitalization by putting {} around the letters to be
% capitalized. Reference the books by chapters!
\nocite{*}
\bibliographystyle{apalike}
\bibliography{reference}
%-----------------------------
%%%%%%%%%%%%%%%%%%%%%%%%%%%%%%%%%%%%%%%%%%%%%%%%%%%%%%%%%%%%%%%%%%%%%%%%%%%%%%%%
%					 	                      APPENDICES							                     %
%%%%%%%%%%%%%%%%%%%%%%%%%%%%%%%%%%%%%%%%%%%%%%%%%%%%%%%%%%%%%%%%%%%%%%%%%%%%%%%%
% Add all appendices in appendices/main.tex with include statements.
% see appendices/sub-example-app.tex for an example.
\onecolumngrid
\input{appendices/main}
%-------------------------------------------------------------------------------
\end{document}

%-------------------------------------------------------------------------------
\end{document}

\documentclass[%
    reprint,
    superscriptaddress,
    %groupedaddress,
    %unsortedaddress,
    %runinaddress,
    %frontmatterverbose,
    %preprint,
    longbibliography,
    bibnotes,
    %showpacs,preprintnumbers,
    %nofootinbib,
    %nobibnotes,
    %bibnotes,
     amsmath,amssymb,
     %aps,
     aip,
     jcp,                                       % J. Chem. Phys
    %pra,
    %prb,
    %rmp,
    %prstab,
    %prstper,
    %floatfix,
    ]{revtex4-1}
%%%%%%%%%%%%%%%%%%%%%%%%%%%%%%%%%%%%%%%%%%%%%%%%%%%%%%%%%%%%%%%%%%%%%%%%%%%%%%%%
%					 	                       LIBRARIES							                     %
%%%%%%%%%%%%%%%%%%%%%%%%%%%%%%%%%%%%%%%%%%%%%%%%%%%%%%%%%%%%%%%%%%%%%%%%%%%%%%%%
% Add whatever libraries you need here. Add a description if you want
\usepackage{graphicx}			 % Include figure files
\usepackage{dcolumn}			 % Align table columns on decimal point
\usepackage{tcolorbox}		 % use tcolorbox environment to box equations
\usepackage[USenglish]{babel}
\usepackage[useregional]{datetime2}
\usepackage[utf8]{inputenc}
\usepackage[T1]{fontenc}
\usepackage{enumerate}
\usepackage{caption}
\usepackage{subcaption}
\usepackage{physics}
\usepackage{hyperref}
\usepackage{chemformula}
\usepackage{booktabs}
\usepackage{makecell}
\usepackage{fullpage}
\usepackage{natbib}
\usepackage{setspace}
\usepackage{amsmath}
\usepackage{bm}              % enchanced bold math symbols
\usepackage{amssymb}
\usepackage{listings} 			 % For code citations
\usepackage{amsfonts}
\usepackage{mathtools}
\usepackage{commath}
\usepackage{fancyhdr}
\usepackage{lastpage}
\usepackage{tikz}
\usepackage{graphicx}
	\graphicspath{ {images/} }
\usepackage{feynmp-auto}
\usepackage[toc,page]{appendix}
%%%%%%%%%%%%%%%%%%%%%%%%%%%%%%%%%%%%%%%%%%%%%%%%%%%%%%%%%%%%%%%%%%%%%%%%%%%%%%%%
%					 	                      NEW COMMANDS							                   %
%%%%%%%%%%%%%%%%%%%%%%%%%%%%%%%%%%%%%%%%%%%%%%%%%%%%%%%%%%%%%%%%%%%%%%%%%%%%%%%%
% Redefine commands if necessary
\renewcommand\vec{\mathbf}                      % use boldface for vectors
% New commands
%------------------------------------Date---------------------------------------
\newcommand{\seminardate}{\DTMdisplaydate{2017}{7}{25}{-1}}
\newcommand{\revisiondate}{\DTMdisplaydate{2017}{7}{26}{-1}}
%-------------------------------------------------------------------------------
\newcommand*\Del{\vec{\nabla}}                  % del operator
\newcommand*\diff{\mathop{}\!\mathrm{d}}		    % differential d
\newcommand*\Diff[1]{\mathop{}\!\mathrm{d^#1}}	% d^(...)
\newcommand*{\scalarnorm}[1]                    % scalar norm |*|
  {\left\lvert#1\right\rvert}
\newcommand*{\vectornorm}[1]                    % vector norm ||*||
  {\left\lVert#1\right\rVert}
\newcommand*\onehalf{\frac{1}{2}}               % fraction 1/2
%------------------------------------Time---------------------------------------
\newcommand*\dlt[1]{\delta #1}
\newcommand*\timestep{\dlt{t}}              % timestep dt
\newcommand*\halfstep                       % half timestep 1/2 dt
  {\onehalf \timestep}
\newcommand*\timeInHalfStep{\left(t + \halfstep\right)}
\newcommand*\timeInFullStep{\left(t + \timestep\right)}
%-------------------------------------------------------------------------------
%%%%%%%%%%%%%%%%%%%%%%%%%%%%%%%%%%%%%%%%%%%%%%%%%%%%%%%%%%%%%%%%%%%%%%%%%%%%%%%%
%					 	                    DEFINITIONS								                     %
%%%%%%%%%%%%%%%%%%%%%%%%%%%%%%%%%%%%%%%%%%%%%%%%%%%%%%%%%%%%%%%%%%%%%%%%%%%%%%%%
%-------------------------------------------------------------------------------
\newcommand*\rattle{\textsf{RATTLE} }       % display name of algorithm
%-------------------------------------------------------------------------------
\newcommand*\rattlePos{\vec{r}}             % position vector
\newcommand*\rattleVel{\vec{v}}             % velocity vector
\newcommand*\rattleAcc{\vec{a}}             % acceleration vector
%-------------------------------------------------------------------------------
\newcommand*\rattleMassNoIndex{m}
\newcommand*\rattleForceNoIndex{\vec{F}}
\newcommand*\rattleConstraintForceNoIndex{\vec{G}}
\newcommand*\rattleConstraintNoIndex{\sigma}
\newcommand*\rattleLagrangeNoIndex{\lambda}
\newcommand*\rattleLagrangeApproxNoIndex{\gamma}
\newcommand*\rattleCorrectiveValueNoIndex{g}
%---------------------Definitions for General Formulation ----------------------
\input{rattle/general/definitions}
%-----------------------------------Indexing------------------------------------
\newcommand*\rattleAtomIndex{a}               % solo atom index
\newcommand*\rattleAtomIndexFirst{a}          % first index in atomic pair
\newcommand*\rattleAtomIndexSecond{b}         % second index in atomic pair
\newcommand*\rattleNAtoms{N}                  % number of atoms
\newcommand*\rattleConstraintSet{K}           % set of constraints
\newcommand*\rattleConstraintIndex            % index for constraints
  {\left(
    \rattleAtomIndexFirst,\,\rattleAtomIndexSecond
  \right) \in \rattleConstraintSet}
\newcommand*\rattleConstraintIndexSimple      % simplified index for constraints
  {\rattleAtomIndexFirst\rattleAtomIndexSecond}
\newcommand*\rattleCorrectionIndex{i}         % index for iterative correction
\newcommand*\rattleCorrectionIndexThis        % current for iterative correction
  {\rattleCorrectionIndex}
\newcommand*\rattleCorrectionIndexNext        % next for iterative correction
    {\rattleCorrectionIndex + 1}
\newcommand*\rattleCorrectionIndexLast{m}     % last step of correction
\newcommand*\rattleNCorrections{M}            % number of iterative corrections
%-----------------------------------Dynamics------------------------------------
\newcommand*\rattleMass                       % solo atomic mass
  {\rattleMassNoIndex_{\rattleAtomIndex}}
\newcommand*\rattleMassFirst                  % first atomic mass in pair
    {\rattleMassNoIndex_{\rattleAtomIndexFirst}}
\newcommand*\rattleMassSecond                 % second atomic mass in pair
    {\rattleMassNoIndex_{\rattleAtomIndexSecond}}
\newcommand*\rattleForce
  {\rattleForceNoIndex_{\rattleAtomIndex}}
\newcommand*\rattleConstraintForce            % solo constraint force
  {\rattleConstraintForceNoIndex_{\rattleAtomIndex}}
\newcommand*\rattleConstraintForceFirst       % first constraint force in pair
    {\rattleConstraintForceNoIndex_{\rattleAtomIndexFirst}}
\newcommand*\rattleConstraintForceSecond      % second constraint force in pair
    {\rattleConstraintForceNoIndex_{\rattleAtomIndexSecond}}
\newcommand*\rattleConstraintForceBond        % constraint force along bond
        {\rattleConstraintForceNoIndex_{\rattleConstraintIndexSimple}}
\newcommand*\rattleConstraint
  {\rattleConstraintNoIndex_{\rattleConstraintIndexSimple}}
  \newcommand*\rattleConstraintDot
    {\dot{\rattleConstraintNoIndex}_{\rattleConstraintIndexSimple}}
\newcommand*\rattleLagrange
    {\rattleLagrangeNoIndex_{\rattleConstraintIndexSimple}}
\newcommand*\rattleLagrangeApprox
        {\rattleLagrangeApproxNoIndex_{\rattleConstraintIndexSimple}}
\newcommand*\rattleCorrectiveValue
  {\rattleCorrectiveValueNoIndex_{\rattleConstraintIndexSimple}}
%-----------------------------IterativeCorrection-------------------------------
\newcommand*\iteration[2]{#1^{(#2)}}            % iterative correction steps
\newcommand*\iterationThis[1]                   % arg^{i}
  {\iteration{#1}{\rattleCorrectionIndexThis}}
\newcommand*\iterationNext[1]                   % arg^{i+1}
  {\iteration{#1}{\rattleCorrectionIndexNext}}
%----------------------------------Vectors--------------------------------------
\newcommand*\rattleAtomPos{\rattlePos_{\rattleAtomIndex}}
\newcommand*\rattleAtomPosFirst{\rattlePos_{\rattleAtomIndexFirst}}
\newcommand*\rattleAtomPosSecond{\rattlePos_{\rattleAtomIndexSecond}}
\newcommand*\rattleAtomVel{\rattleVel_{\rattleAtomIndex}}
\newcommand*\rattleAtomVelFirst{\rattleVel_{\rattleAtomIndexFirst}}
\newcommand*\rattleAtomVelSecond{\rattleVel_{\rattleAtomIndexSecond}}
\newcommand*\rattlePosUnconstrained             % r^{(0)}
  {\iteration{\rattlePos}{0}}
\newcommand*\rattleVelUnconstrained             % v^{(0)}
  {\iteration{\rattleVel}{0}}
\newcommand*\rattleAtomPosUnconstrained         % r^{(0)}
  {\iteration{\rattleAtomPos}{0}}
\newcommand*\rattleAtomVelUnconstrained         % v^{(0)}
  {\iteration{\rattleAtomVel}{0}}
\newcommand*\rattleBondPos{\rattlePos_{\rattleConstraintIndexSimple}}
\newcommand*\rattleBondPosUnit{\hat{\rattlePos}_{\rattleConstraintIndexSimple}}
\newcommand*\rattleBondVel{\rattleVel_{\rattleConstraintIndexSimple}}
\newcommand*\rattleBondVelUnit{\hat{\rattleVel}_{\rattleConstraintIndexSimple}}
\newcommand*\rattleBondPosUnconstrained
  {\iteration{\rattleBondPos}{0}}
\newcommand*\rattleBondVelUnconstrained
  {\iteration{\rattleBondVel}{0}}
\newcommand*\rattleAtomPosUnconstrainedFirst
    {\iteration{\rattleAtomPosFirst}{0}}
\newcommand*\rattleAtomPosUnconstrainedSecond
    {\iteration{\rattleAtomPosSecond}{0}}
\newcommand*\rattleAtomVelUnconstrainedFirst
    {\iteration{\rattleAtomVelFirst}{0}}
\newcommand*\rattleAtomVelUnconstrainedSecond
    {\iteration{\rattleAtomVelSecond}{0}}
%---------------------------------Constants-------------------------------------
\newcommand*\rattleBondlength                   % fixed bond length
  {d_{\rattleConstraintIndexSimple}}
\newcommand*\rattleTol{\xi}                     % d-r/d tolerance
\newcommand*\rattleRRTol{\varepsilon}           % r(t) * r(t+dt) tolerance
\newcommand*\rattleRVTol{\rattleTol}            % v(t)/d tolerance
%-------------------------------------------------------------------------------

%%%%%%%%%%%%%%%%%%%%%%%%%%%%%%%%%%%%%%%%%%%%%%%%%%%%%%%%%%%%%%%%%%%%%%%%%%%%%%%%
%					 	                    DEFINITIONS								                     %
%%%%%%%%%%%%%%%%%%%%%%%%%%%%%%%%%%%%%%%%%%%%%%%%%%%%%%%%%%%%%%%%%%%%%%%%%%%%%%%%
%-------------------------------------------------------------------------------
\newcommand*\rattle{\textsf{RATTLE} }       % display name of algorithm
%-------------------------------------------------------------------------------
\newcommand*\rattlePos{\vec{r}}             % position vector
\newcommand*\rattleVel{\vec{v}}             % velocity vector
\newcommand*\rattleAcc{\vec{a}}             % acceleration vector
%-------------------------------------------------------------------------------
\newcommand*\rattleMassNoIndex{m}
\newcommand*\rattleForceNoIndex{\vec{F}}
\newcommand*\rattleConstraintForceNoIndex{\vec{G}}
\newcommand*\rattleConstraintNoIndex{\sigma}
\newcommand*\rattleLagrangeNoIndex{\lambda}
\newcommand*\rattleLagrangeApproxNoIndex{\gamma}
\newcommand*\rattleCorrectiveValueNoIndex{g}
%---------------------Definitions for General Formulation ----------------------
\input{rattle/general/definitions}
%-----------------------------------Indexing------------------------------------
\newcommand*\rattleAtomIndex{a}               % solo atom index
\newcommand*\rattleAtomIndexFirst{a}          % first index in atomic pair
\newcommand*\rattleAtomIndexSecond{b}         % second index in atomic pair
\newcommand*\rattleNAtoms{N}                  % number of atoms
\newcommand*\rattleConstraintSet{K}           % set of constraints
\newcommand*\rattleConstraintIndex            % index for constraints
  {\left(
    \rattleAtomIndexFirst,\,\rattleAtomIndexSecond
  \right) \in \rattleConstraintSet}
\newcommand*\rattleConstraintIndexSimple      % simplified index for constraints
  {\rattleAtomIndexFirst\rattleAtomIndexSecond}
\newcommand*\rattleCorrectionIndex{i}         % index for iterative correction
\newcommand*\rattleCorrectionIndexThis        % current for iterative correction
  {\rattleCorrectionIndex}
\newcommand*\rattleCorrectionIndexNext        % next for iterative correction
    {\rattleCorrectionIndex + 1}
\newcommand*\rattleCorrectionIndexLast{m}     % last step of correction
\newcommand*\rattleNCorrections{M}            % number of iterative corrections
%-----------------------------------Dynamics------------------------------------
\newcommand*\rattleMass                       % solo atomic mass
  {\rattleMassNoIndex_{\rattleAtomIndex}}
\newcommand*\rattleMassFirst                  % first atomic mass in pair
    {\rattleMassNoIndex_{\rattleAtomIndexFirst}}
\newcommand*\rattleMassSecond                 % second atomic mass in pair
    {\rattleMassNoIndex_{\rattleAtomIndexSecond}}
\newcommand*\rattleForce
  {\rattleForceNoIndex_{\rattleAtomIndex}}
\newcommand*\rattleConstraintForce            % solo constraint force
  {\rattleConstraintForceNoIndex_{\rattleAtomIndex}}
\newcommand*\rattleConstraintForceFirst       % first constraint force in pair
    {\rattleConstraintForceNoIndex_{\rattleAtomIndexFirst}}
\newcommand*\rattleConstraintForceSecond      % second constraint force in pair
    {\rattleConstraintForceNoIndex_{\rattleAtomIndexSecond}}
\newcommand*\rattleConstraintForceBond        % constraint force along bond
        {\rattleConstraintForceNoIndex_{\rattleConstraintIndexSimple}}
\newcommand*\rattleConstraint
  {\rattleConstraintNoIndex_{\rattleConstraintIndexSimple}}
  \newcommand*\rattleConstraintDot
    {\dot{\rattleConstraintNoIndex}_{\rattleConstraintIndexSimple}}
\newcommand*\rattleLagrange
    {\rattleLagrangeNoIndex_{\rattleConstraintIndexSimple}}
\newcommand*\rattleLagrangeApprox
        {\rattleLagrangeApproxNoIndex_{\rattleConstraintIndexSimple}}
\newcommand*\rattleCorrectiveValue
  {\rattleCorrectiveValueNoIndex_{\rattleConstraintIndexSimple}}
%-----------------------------IterativeCorrection-------------------------------
\newcommand*\iteration[2]{#1^{(#2)}}            % iterative correction steps
\newcommand*\iterationThis[1]                   % arg^{i}
  {\iteration{#1}{\rattleCorrectionIndexThis}}
\newcommand*\iterationNext[1]                   % arg^{i+1}
  {\iteration{#1}{\rattleCorrectionIndexNext}}
%----------------------------------Vectors--------------------------------------
\newcommand*\rattleAtomPos{\rattlePos_{\rattleAtomIndex}}
\newcommand*\rattleAtomPosFirst{\rattlePos_{\rattleAtomIndexFirst}}
\newcommand*\rattleAtomPosSecond{\rattlePos_{\rattleAtomIndexSecond}}
\newcommand*\rattleAtomVel{\rattleVel_{\rattleAtomIndex}}
\newcommand*\rattleAtomVelFirst{\rattleVel_{\rattleAtomIndexFirst}}
\newcommand*\rattleAtomVelSecond{\rattleVel_{\rattleAtomIndexSecond}}
\newcommand*\rattlePosUnconstrained             % r^{(0)}
  {\iteration{\rattlePos}{0}}
\newcommand*\rattleVelUnconstrained             % v^{(0)}
  {\iteration{\rattleVel}{0}}
\newcommand*\rattleAtomPosUnconstrained         % r^{(0)}
  {\iteration{\rattleAtomPos}{0}}
\newcommand*\rattleAtomVelUnconstrained         % v^{(0)}
  {\iteration{\rattleAtomVel}{0}}
\newcommand*\rattleBondPos{\rattlePos_{\rattleConstraintIndexSimple}}
\newcommand*\rattleBondPosUnit{\hat{\rattlePos}_{\rattleConstraintIndexSimple}}
\newcommand*\rattleBondVel{\rattleVel_{\rattleConstraintIndexSimple}}
\newcommand*\rattleBondVelUnit{\hat{\rattleVel}_{\rattleConstraintIndexSimple}}
\newcommand*\rattleBondPosUnconstrained
  {\iteration{\rattleBondPos}{0}}
\newcommand*\rattleBondVelUnconstrained
  {\iteration{\rattleBondVel}{0}}
\newcommand*\rattleAtomPosUnconstrainedFirst
    {\iteration{\rattleAtomPosFirst}{0}}
\newcommand*\rattleAtomPosUnconstrainedSecond
    {\iteration{\rattleAtomPosSecond}{0}}
\newcommand*\rattleAtomVelUnconstrainedFirst
    {\iteration{\rattleAtomVelFirst}{0}}
\newcommand*\rattleAtomVelUnconstrainedSecond
    {\iteration{\rattleAtomVelSecond}{0}}
%---------------------------------Constants-------------------------------------
\newcommand*\rattleBondlength                   % fixed bond length
  {d_{\rattleConstraintIndexSimple}}
\newcommand*\rattleTol{\xi}                     % d-r/d tolerance
\newcommand*\rattleRRTol{\varepsilon}           % r(t) * r(t+dt) tolerance
\newcommand*\rattleRVTol{\rattleTol}            % v(t)/d tolerance
%-------------------------------------------------------------------------------

%-------------------------------------------------------------------------------
%\setlength{\parindent}{0pt}						          % do not indent
\numberwithin{equation}{section}                % number eqs within sections
\begin{document}
\lstset{language=Python}
%%%%%%%%%%%%%%%%%%%%%%%%%%%%%%%%%%%%%%%%%%%%%%%%%%%%%%%%%%%%%%%%%%%%%%%%%%%%%%%%
%					 	                        TITLE							                         %
%%%%%%%%%%%%%%%%%%%%%%%%%%%%%%%%%%%%%%%%%%%%%%%%%%%%%%%%%%%%%%%%%%%%%%%%%%%%%%%%
\title{Group Seminar (revised: \revisiondate)}
\thanks{Professor Stratt, Vale Cofer-Shabica, Yan Zhao, Andrew Ton, Mansheej Paul, Evan Coleman}
\author{Artur Avkhadiev}
\email{artur\_avkhadiev@brown.edu}
\affiliation{Department of Physics, Brown University, Providence RI 02912, USA}
\date{\seminardate}
%%%%%%%%%%%%%%%%%%%%%%%%%%%%%%%%%%%%%%%%%%%%%%%%%%%%%%%%%%%%%%%%%%%%%%%%%%%%%%%%
%					 	                       ABSTRACT							                       %
%%%%%%%%%%%%%%%%%%%%%%%%%%%%%%%%%%%%%%%%%%%%%%%%%%%%%%%%%%%%%%%%%%%%%%%%%%%%%%%%
\begin{abstract}
  This report consist of two parts:
  \begin{enumerate}
      \item A discussion of the constraint dynamics algorithm based on velocity Verlet numerical integration scheme, \rattle.
      \item A calculation of the kinetic energy of a freely-jointed polymer chain using the bond-vector representation in the CM frame of the molecule.
    \end{enumerate}
\end{abstract}
\maketitle
\tableofcontents						 % hide table of contents
%%%%%%%%%%%%%%%%%%%%%%%%%%%%%%%%%%%%%%%%%%%%%%%%%%%%%%%%%%%%%%%%%%%%%%%%%%%%%%%%
%					 	                        BODY							                         %
%%%%%%%%%%%%%%%%%%%%%%%%%%%%%%%%%%%%%%%%%%%%%%%%%%%%%%%%%%%%%%%%%%%%%%%%%%%%%%%%
% Add include statements below. All the text should be in the folders.
\documentclass[%
    reprint,
    superscriptaddress,
    %groupedaddress,
    %unsortedaddress,
    %runinaddress,
    %frontmatterverbose,
    %preprint,
    longbibliography,
    bibnotes,
    %showpacs,preprintnumbers,
    %nofootinbib,
    %nobibnotes,
    %bibnotes,
     amsmath,amssymb,
     %aps,
     aip,
     jcp,                                       % J. Chem. Phys
    %pra,
    %prb,
    %rmp,
    %prstab,
    %prstper,
    %floatfix,
    ]{revtex4-1}
%%%%%%%%%%%%%%%%%%%%%%%%%%%%%%%%%%%%%%%%%%%%%%%%%%%%%%%%%%%%%%%%%%%%%%%%%%%%%%%%
%					 	                       LIBRARIES							                     %
%%%%%%%%%%%%%%%%%%%%%%%%%%%%%%%%%%%%%%%%%%%%%%%%%%%%%%%%%%%%%%%%%%%%%%%%%%%%%%%%
% Add whatever libraries you need here. Add a description if you want
\usepackage{graphicx}			 % Include figure files
\usepackage{dcolumn}			 % Align table columns on decimal point
\usepackage{tcolorbox}		 % use tcolorbox environment to box equations
\usepackage[USenglish]{babel}
\usepackage[useregional]{datetime2}
\usepackage[utf8]{inputenc}
\usepackage[T1]{fontenc}
\usepackage{enumerate}
\usepackage{caption}
\usepackage{subcaption}
\usepackage{physics}
\usepackage{hyperref}
\usepackage{chemformula}
\usepackage{booktabs}
\usepackage{makecell}
\usepackage{fullpage}
\usepackage{natbib}
\usepackage{setspace}
\usepackage{amsmath}
\usepackage{bm}              % enchanced bold math symbols
\usepackage{amssymb}
\usepackage{listings} 			 % For code citations
\usepackage{amsfonts}
\usepackage{mathtools}
\usepackage{commath}
\usepackage{fancyhdr}
\usepackage{lastpage}
\usepackage{tikz}
\usepackage{graphicx}
	\graphicspath{ {images/} }
\usepackage{feynmp-auto}
\usepackage[toc,page]{appendix}
%%%%%%%%%%%%%%%%%%%%%%%%%%%%%%%%%%%%%%%%%%%%%%%%%%%%%%%%%%%%%%%%%%%%%%%%%%%%%%%%
%					 	                      NEW COMMANDS							                   %
%%%%%%%%%%%%%%%%%%%%%%%%%%%%%%%%%%%%%%%%%%%%%%%%%%%%%%%%%%%%%%%%%%%%%%%%%%%%%%%%
% Redefine commands if necessary
\renewcommand\vec{\mathbf}                      % use boldface for vectors
% New commands
%------------------------------------Date---------------------------------------
\newcommand{\seminardate}{\DTMdisplaydate{2017}{7}{25}{-1}}
\newcommand{\revisiondate}{\DTMdisplaydate{2017}{7}{26}{-1}}
%-------------------------------------------------------------------------------
\newcommand*\Del{\vec{\nabla}}                  % del operator
\newcommand*\diff{\mathop{}\!\mathrm{d}}		    % differential d
\newcommand*\Diff[1]{\mathop{}\!\mathrm{d^#1}}	% d^(...)
\newcommand*{\scalarnorm}[1]                    % scalar norm |*|
  {\left\lvert#1\right\rvert}
\newcommand*{\vectornorm}[1]                    % vector norm ||*||
  {\left\lVert#1\right\rVert}
\newcommand*\onehalf{\frac{1}{2}}               % fraction 1/2
%------------------------------------Time---------------------------------------
\newcommand*\dlt[1]{\delta #1}
\newcommand*\timestep{\dlt{t}}              % timestep dt
\newcommand*\halfstep                       % half timestep 1/2 dt
  {\onehalf \timestep}
\newcommand*\timeInHalfStep{\left(t + \halfstep\right)}
\newcommand*\timeInFullStep{\left(t + \timestep\right)}
%-------------------------------------------------------------------------------
\input{rattle/definitions}
\input{ke_polymer/definitions}
%-------------------------------------------------------------------------------
%\setlength{\parindent}{0pt}						          % do not indent
\numberwithin{equation}{section}                % number eqs within sections
\begin{document}
\lstset{language=Python}
%%%%%%%%%%%%%%%%%%%%%%%%%%%%%%%%%%%%%%%%%%%%%%%%%%%%%%%%%%%%%%%%%%%%%%%%%%%%%%%%
%					 	                        TITLE							                         %
%%%%%%%%%%%%%%%%%%%%%%%%%%%%%%%%%%%%%%%%%%%%%%%%%%%%%%%%%%%%%%%%%%%%%%%%%%%%%%%%
\title{Group Seminar (revised: \revisiondate)}
\thanks{Professor Stratt, Vale Cofer-Shabica, Yan Zhao, Andrew Ton, Mansheej Paul, Evan Coleman}
\author{Artur Avkhadiev}
\email{artur\_avkhadiev@brown.edu}
\affiliation{Department of Physics, Brown University, Providence RI 02912, USA}
\date{\seminardate}
%%%%%%%%%%%%%%%%%%%%%%%%%%%%%%%%%%%%%%%%%%%%%%%%%%%%%%%%%%%%%%%%%%%%%%%%%%%%%%%%
%					 	                       ABSTRACT							                       %
%%%%%%%%%%%%%%%%%%%%%%%%%%%%%%%%%%%%%%%%%%%%%%%%%%%%%%%%%%%%%%%%%%%%%%%%%%%%%%%%
\begin{abstract}
  This report consist of two parts:
  \begin{enumerate}
      \item A discussion of the constraint dynamics algorithm based on velocity Verlet numerical integration scheme, \rattle.
      \item A calculation of the kinetic energy of a freely-jointed polymer chain using the bond-vector representation in the CM frame of the molecule.
    \end{enumerate}
\end{abstract}
\maketitle
\tableofcontents						 % hide table of contents
%%%%%%%%%%%%%%%%%%%%%%%%%%%%%%%%%%%%%%%%%%%%%%%%%%%%%%%%%%%%%%%%%%%%%%%%%%%%%%%%
%					 	                        BODY							                         %
%%%%%%%%%%%%%%%%%%%%%%%%%%%%%%%%%%%%%%%%%%%%%%%%%%%%%%%%%%%%%%%%%%%%%%%%%%%%%%%%
% Add include statements below. All the text should be in the folders.
\input{rattle/main}
\input{ke_polymer/main}
%-------------------------------------------------------------------------------
%%%%%%%%%%%%%%%%%%%%%%%%%%%%%%%%%%%%%%%%%%%%%%%%%%%%%%%%%%%%%%%%%%%%%%%%%%%%%%%%
%					                        BIBLIOGRAPHY							                   %
%%%%%%%%%%%%%%%%%%%%%%%%%%%%%%%%%%%%%%%%%%%%%%%%%%%%%%%%%%%%%%%%%%%%%%%%%%%%%%%%
% The bibliography file is reference,bib. If you need to add a
% reference, make sure all the entries look good (e.g., ensure
% proper capitalization by putting {} around the letters to be
% capitalized. Reference the books by chapters!
\nocite{*}
\bibliographystyle{apalike}
\bibliography{reference}
%-----------------------------
%%%%%%%%%%%%%%%%%%%%%%%%%%%%%%%%%%%%%%%%%%%%%%%%%%%%%%%%%%%%%%%%%%%%%%%%%%%%%%%%
%					 	                      APPENDICES							                     %
%%%%%%%%%%%%%%%%%%%%%%%%%%%%%%%%%%%%%%%%%%%%%%%%%%%%%%%%%%%%%%%%%%%%%%%%%%%%%%%%
% Add all appendices in appendices/main.tex with include statements.
% see appendices/sub-example-app.tex for an example.
\onecolumngrid
\input{appendices/main}
%-------------------------------------------------------------------------------
\end{document}

\documentclass[%
    reprint,
    superscriptaddress,
    %groupedaddress,
    %unsortedaddress,
    %runinaddress,
    %frontmatterverbose,
    %preprint,
    longbibliography,
    bibnotes,
    %showpacs,preprintnumbers,
    %nofootinbib,
    %nobibnotes,
    %bibnotes,
     amsmath,amssymb,
     %aps,
     aip,
     jcp,                                       % J. Chem. Phys
    %pra,
    %prb,
    %rmp,
    %prstab,
    %prstper,
    %floatfix,
    ]{revtex4-1}
%%%%%%%%%%%%%%%%%%%%%%%%%%%%%%%%%%%%%%%%%%%%%%%%%%%%%%%%%%%%%%%%%%%%%%%%%%%%%%%%
%					 	                       LIBRARIES							                     %
%%%%%%%%%%%%%%%%%%%%%%%%%%%%%%%%%%%%%%%%%%%%%%%%%%%%%%%%%%%%%%%%%%%%%%%%%%%%%%%%
% Add whatever libraries you need here. Add a description if you want
\usepackage{graphicx}			 % Include figure files
\usepackage{dcolumn}			 % Align table columns on decimal point
\usepackage{tcolorbox}		 % use tcolorbox environment to box equations
\usepackage[USenglish]{babel}
\usepackage[useregional]{datetime2}
\usepackage[utf8]{inputenc}
\usepackage[T1]{fontenc}
\usepackage{enumerate}
\usepackage{caption}
\usepackage{subcaption}
\usepackage{physics}
\usepackage{hyperref}
\usepackage{chemformula}
\usepackage{booktabs}
\usepackage{makecell}
\usepackage{fullpage}
\usepackage{natbib}
\usepackage{setspace}
\usepackage{amsmath}
\usepackage{bm}              % enchanced bold math symbols
\usepackage{amssymb}
\usepackage{listings} 			 % For code citations
\usepackage{amsfonts}
\usepackage{mathtools}
\usepackage{commath}
\usepackage{fancyhdr}
\usepackage{lastpage}
\usepackage{tikz}
\usepackage{graphicx}
	\graphicspath{ {images/} }
\usepackage{feynmp-auto}
\usepackage[toc,page]{appendix}
%%%%%%%%%%%%%%%%%%%%%%%%%%%%%%%%%%%%%%%%%%%%%%%%%%%%%%%%%%%%%%%%%%%%%%%%%%%%%%%%
%					 	                      NEW COMMANDS							                   %
%%%%%%%%%%%%%%%%%%%%%%%%%%%%%%%%%%%%%%%%%%%%%%%%%%%%%%%%%%%%%%%%%%%%%%%%%%%%%%%%
% Redefine commands if necessary
\renewcommand\vec{\mathbf}                      % use boldface for vectors
% New commands
%------------------------------------Date---------------------------------------
\newcommand{\seminardate}{\DTMdisplaydate{2017}{7}{25}{-1}}
\newcommand{\revisiondate}{\DTMdisplaydate{2017}{7}{26}{-1}}
%-------------------------------------------------------------------------------
\newcommand*\Del{\vec{\nabla}}                  % del operator
\newcommand*\diff{\mathop{}\!\mathrm{d}}		    % differential d
\newcommand*\Diff[1]{\mathop{}\!\mathrm{d^#1}}	% d^(...)
\newcommand*{\scalarnorm}[1]                    % scalar norm |*|
  {\left\lvert#1\right\rvert}
\newcommand*{\vectornorm}[1]                    % vector norm ||*||
  {\left\lVert#1\right\rVert}
\newcommand*\onehalf{\frac{1}{2}}               % fraction 1/2
%------------------------------------Time---------------------------------------
\newcommand*\dlt[1]{\delta #1}
\newcommand*\timestep{\dlt{t}}              % timestep dt
\newcommand*\halfstep                       % half timestep 1/2 dt
  {\onehalf \timestep}
\newcommand*\timeInHalfStep{\left(t + \halfstep\right)}
\newcommand*\timeInFullStep{\left(t + \timestep\right)}
%-------------------------------------------------------------------------------
\input{rattle/definitions}
\input{ke_polymer/definitions}
%-------------------------------------------------------------------------------
%\setlength{\parindent}{0pt}						          % do not indent
\numberwithin{equation}{section}                % number eqs within sections
\begin{document}
\lstset{language=Python}
%%%%%%%%%%%%%%%%%%%%%%%%%%%%%%%%%%%%%%%%%%%%%%%%%%%%%%%%%%%%%%%%%%%%%%%%%%%%%%%%
%					 	                        TITLE							                         %
%%%%%%%%%%%%%%%%%%%%%%%%%%%%%%%%%%%%%%%%%%%%%%%%%%%%%%%%%%%%%%%%%%%%%%%%%%%%%%%%
\title{Group Seminar (revised: \revisiondate)}
\thanks{Professor Stratt, Vale Cofer-Shabica, Yan Zhao, Andrew Ton, Mansheej Paul, Evan Coleman}
\author{Artur Avkhadiev}
\email{artur\_avkhadiev@brown.edu}
\affiliation{Department of Physics, Brown University, Providence RI 02912, USA}
\date{\seminardate}
%%%%%%%%%%%%%%%%%%%%%%%%%%%%%%%%%%%%%%%%%%%%%%%%%%%%%%%%%%%%%%%%%%%%%%%%%%%%%%%%
%					 	                       ABSTRACT							                       %
%%%%%%%%%%%%%%%%%%%%%%%%%%%%%%%%%%%%%%%%%%%%%%%%%%%%%%%%%%%%%%%%%%%%%%%%%%%%%%%%
\begin{abstract}
  This report consist of two parts:
  \begin{enumerate}
      \item A discussion of the constraint dynamics algorithm based on velocity Verlet numerical integration scheme, \rattle.
      \item A calculation of the kinetic energy of a freely-jointed polymer chain using the bond-vector representation in the CM frame of the molecule.
    \end{enumerate}
\end{abstract}
\maketitle
\tableofcontents						 % hide table of contents
%%%%%%%%%%%%%%%%%%%%%%%%%%%%%%%%%%%%%%%%%%%%%%%%%%%%%%%%%%%%%%%%%%%%%%%%%%%%%%%%
%					 	                        BODY							                         %
%%%%%%%%%%%%%%%%%%%%%%%%%%%%%%%%%%%%%%%%%%%%%%%%%%%%%%%%%%%%%%%%%%%%%%%%%%%%%%%%
% Add include statements below. All the text should be in the folders.
\input{rattle/main}
\input{ke_polymer/main}
%-------------------------------------------------------------------------------
%%%%%%%%%%%%%%%%%%%%%%%%%%%%%%%%%%%%%%%%%%%%%%%%%%%%%%%%%%%%%%%%%%%%%%%%%%%%%%%%
%					                        BIBLIOGRAPHY							                   %
%%%%%%%%%%%%%%%%%%%%%%%%%%%%%%%%%%%%%%%%%%%%%%%%%%%%%%%%%%%%%%%%%%%%%%%%%%%%%%%%
% The bibliography file is reference,bib. If you need to add a
% reference, make sure all the entries look good (e.g., ensure
% proper capitalization by putting {} around the letters to be
% capitalized. Reference the books by chapters!
\nocite{*}
\bibliographystyle{apalike}
\bibliography{reference}
%-----------------------------
%%%%%%%%%%%%%%%%%%%%%%%%%%%%%%%%%%%%%%%%%%%%%%%%%%%%%%%%%%%%%%%%%%%%%%%%%%%%%%%%
%					 	                      APPENDICES							                     %
%%%%%%%%%%%%%%%%%%%%%%%%%%%%%%%%%%%%%%%%%%%%%%%%%%%%%%%%%%%%%%%%%%%%%%%%%%%%%%%%
% Add all appendices in appendices/main.tex with include statements.
% see appendices/sub-example-app.tex for an example.
\onecolumngrid
\input{appendices/main}
%-------------------------------------------------------------------------------
\end{document}

%-------------------------------------------------------------------------------
%%%%%%%%%%%%%%%%%%%%%%%%%%%%%%%%%%%%%%%%%%%%%%%%%%%%%%%%%%%%%%%%%%%%%%%%%%%%%%%%
%					                        BIBLIOGRAPHY							                   %
%%%%%%%%%%%%%%%%%%%%%%%%%%%%%%%%%%%%%%%%%%%%%%%%%%%%%%%%%%%%%%%%%%%%%%%%%%%%%%%%
% The bibliography file is reference,bib. If you need to add a
% reference, make sure all the entries look good (e.g., ensure
% proper capitalization by putting {} around the letters to be
% capitalized. Reference the books by chapters!
\nocite{*}
\bibliographystyle{apalike}
\bibliography{reference}
%-----------------------------
%%%%%%%%%%%%%%%%%%%%%%%%%%%%%%%%%%%%%%%%%%%%%%%%%%%%%%%%%%%%%%%%%%%%%%%%%%%%%%%%
%					 	                      APPENDICES							                     %
%%%%%%%%%%%%%%%%%%%%%%%%%%%%%%%%%%%%%%%%%%%%%%%%%%%%%%%%%%%%%%%%%%%%%%%%%%%%%%%%
% Add all appendices in appendices/main.tex with include statements.
% see appendices/sub-example-app.tex for an example.
\onecolumngrid
\documentclass[%
    reprint,
    superscriptaddress,
    %groupedaddress,
    %unsortedaddress,
    %runinaddress,
    %frontmatterverbose,
    %preprint,
    longbibliography,
    bibnotes,
    %showpacs,preprintnumbers,
    %nofootinbib,
    %nobibnotes,
    %bibnotes,
     amsmath,amssymb,
     %aps,
     aip,
     jcp,                                       % J. Chem. Phys
    %pra,
    %prb,
    %rmp,
    %prstab,
    %prstper,
    %floatfix,
    ]{revtex4-1}
%%%%%%%%%%%%%%%%%%%%%%%%%%%%%%%%%%%%%%%%%%%%%%%%%%%%%%%%%%%%%%%%%%%%%%%%%%%%%%%%
%					 	                       LIBRARIES							                     %
%%%%%%%%%%%%%%%%%%%%%%%%%%%%%%%%%%%%%%%%%%%%%%%%%%%%%%%%%%%%%%%%%%%%%%%%%%%%%%%%
% Add whatever libraries you need here. Add a description if you want
\usepackage{graphicx}			 % Include figure files
\usepackage{dcolumn}			 % Align table columns on decimal point
\usepackage{tcolorbox}		 % use tcolorbox environment to box equations
\usepackage[USenglish]{babel}
\usepackage[useregional]{datetime2}
\usepackage[utf8]{inputenc}
\usepackage[T1]{fontenc}
\usepackage{enumerate}
\usepackage{caption}
\usepackage{subcaption}
\usepackage{physics}
\usepackage{hyperref}
\usepackage{chemformula}
\usepackage{booktabs}
\usepackage{makecell}
\usepackage{fullpage}
\usepackage{natbib}
\usepackage{setspace}
\usepackage{amsmath}
\usepackage{bm}              % enchanced bold math symbols
\usepackage{amssymb}
\usepackage{listings} 			 % For code citations
\usepackage{amsfonts}
\usepackage{mathtools}
\usepackage{commath}
\usepackage{fancyhdr}
\usepackage{lastpage}
\usepackage{tikz}
\usepackage{graphicx}
	\graphicspath{ {images/} }
\usepackage{feynmp-auto}
\usepackage[toc,page]{appendix}
%%%%%%%%%%%%%%%%%%%%%%%%%%%%%%%%%%%%%%%%%%%%%%%%%%%%%%%%%%%%%%%%%%%%%%%%%%%%%%%%
%					 	                      NEW COMMANDS							                   %
%%%%%%%%%%%%%%%%%%%%%%%%%%%%%%%%%%%%%%%%%%%%%%%%%%%%%%%%%%%%%%%%%%%%%%%%%%%%%%%%
% Redefine commands if necessary
\renewcommand\vec{\mathbf}                      % use boldface for vectors
% New commands
%------------------------------------Date---------------------------------------
\newcommand{\seminardate}{\DTMdisplaydate{2017}{7}{25}{-1}}
\newcommand{\revisiondate}{\DTMdisplaydate{2017}{7}{26}{-1}}
%-------------------------------------------------------------------------------
\newcommand*\Del{\vec{\nabla}}                  % del operator
\newcommand*\diff{\mathop{}\!\mathrm{d}}		    % differential d
\newcommand*\Diff[1]{\mathop{}\!\mathrm{d^#1}}	% d^(...)
\newcommand*{\scalarnorm}[1]                    % scalar norm |*|
  {\left\lvert#1\right\rvert}
\newcommand*{\vectornorm}[1]                    % vector norm ||*||
  {\left\lVert#1\right\rVert}
\newcommand*\onehalf{\frac{1}{2}}               % fraction 1/2
%------------------------------------Time---------------------------------------
\newcommand*\dlt[1]{\delta #1}
\newcommand*\timestep{\dlt{t}}              % timestep dt
\newcommand*\halfstep                       % half timestep 1/2 dt
  {\onehalf \timestep}
\newcommand*\timeInHalfStep{\left(t + \halfstep\right)}
\newcommand*\timeInFullStep{\left(t + \timestep\right)}
%-------------------------------------------------------------------------------
\input{rattle/definitions}
\input{ke_polymer/definitions}
%-------------------------------------------------------------------------------
%\setlength{\parindent}{0pt}						          % do not indent
\numberwithin{equation}{section}                % number eqs within sections
\begin{document}
\lstset{language=Python}
%%%%%%%%%%%%%%%%%%%%%%%%%%%%%%%%%%%%%%%%%%%%%%%%%%%%%%%%%%%%%%%%%%%%%%%%%%%%%%%%
%					 	                        TITLE							                         %
%%%%%%%%%%%%%%%%%%%%%%%%%%%%%%%%%%%%%%%%%%%%%%%%%%%%%%%%%%%%%%%%%%%%%%%%%%%%%%%%
\title{Group Seminar (revised: \revisiondate)}
\thanks{Professor Stratt, Vale Cofer-Shabica, Yan Zhao, Andrew Ton, Mansheej Paul, Evan Coleman}
\author{Artur Avkhadiev}
\email{artur\_avkhadiev@brown.edu}
\affiliation{Department of Physics, Brown University, Providence RI 02912, USA}
\date{\seminardate}
%%%%%%%%%%%%%%%%%%%%%%%%%%%%%%%%%%%%%%%%%%%%%%%%%%%%%%%%%%%%%%%%%%%%%%%%%%%%%%%%
%					 	                       ABSTRACT							                       %
%%%%%%%%%%%%%%%%%%%%%%%%%%%%%%%%%%%%%%%%%%%%%%%%%%%%%%%%%%%%%%%%%%%%%%%%%%%%%%%%
\begin{abstract}
  This report consist of two parts:
  \begin{enumerate}
      \item A discussion of the constraint dynamics algorithm based on velocity Verlet numerical integration scheme, \rattle.
      \item A calculation of the kinetic energy of a freely-jointed polymer chain using the bond-vector representation in the CM frame of the molecule.
    \end{enumerate}
\end{abstract}
\maketitle
\tableofcontents						 % hide table of contents
%%%%%%%%%%%%%%%%%%%%%%%%%%%%%%%%%%%%%%%%%%%%%%%%%%%%%%%%%%%%%%%%%%%%%%%%%%%%%%%%
%					 	                        BODY							                         %
%%%%%%%%%%%%%%%%%%%%%%%%%%%%%%%%%%%%%%%%%%%%%%%%%%%%%%%%%%%%%%%%%%%%%%%%%%%%%%%%
% Add include statements below. All the text should be in the folders.
\input{rattle/main}
\input{ke_polymer/main}
%-------------------------------------------------------------------------------
%%%%%%%%%%%%%%%%%%%%%%%%%%%%%%%%%%%%%%%%%%%%%%%%%%%%%%%%%%%%%%%%%%%%%%%%%%%%%%%%
%					                        BIBLIOGRAPHY							                   %
%%%%%%%%%%%%%%%%%%%%%%%%%%%%%%%%%%%%%%%%%%%%%%%%%%%%%%%%%%%%%%%%%%%%%%%%%%%%%%%%
% The bibliography file is reference,bib. If you need to add a
% reference, make sure all the entries look good (e.g., ensure
% proper capitalization by putting {} around the letters to be
% capitalized. Reference the books by chapters!
\nocite{*}
\bibliographystyle{apalike}
\bibliography{reference}
%-----------------------------
%%%%%%%%%%%%%%%%%%%%%%%%%%%%%%%%%%%%%%%%%%%%%%%%%%%%%%%%%%%%%%%%%%%%%%%%%%%%%%%%
%					 	                      APPENDICES							                     %
%%%%%%%%%%%%%%%%%%%%%%%%%%%%%%%%%%%%%%%%%%%%%%%%%%%%%%%%%%%%%%%%%%%%%%%%%%%%%%%%
% Add all appendices in appendices/main.tex with include statements.
% see appendices/sub-example-app.tex for an example.
\onecolumngrid
\input{appendices/main}
%-------------------------------------------------------------------------------
\end{document}

%-------------------------------------------------------------------------------
\end{document}

%-------------------------------------------------------------------------------
%%%%%%%%%%%%%%%%%%%%%%%%%%%%%%%%%%%%%%%%%%%%%%%%%%%%%%%%%%%%%%%%%%%%%%%%%%%%%%%%
%					                        BIBLIOGRAPHY							                   %
%%%%%%%%%%%%%%%%%%%%%%%%%%%%%%%%%%%%%%%%%%%%%%%%%%%%%%%%%%%%%%%%%%%%%%%%%%%%%%%%
% The bibliography file is reference,bib. If you need to add a
% reference, make sure all the entries look good (e.g., ensure
% proper capitalization by putting {} around the letters to be
% capitalized. Reference the books by chapters!
\nocite{*}
\bibliographystyle{apalike}
\bibliography{reference}
%-----------------------------
%%%%%%%%%%%%%%%%%%%%%%%%%%%%%%%%%%%%%%%%%%%%%%%%%%%%%%%%%%%%%%%%%%%%%%%%%%%%%%%%
%					 	                      APPENDICES							                     %
%%%%%%%%%%%%%%%%%%%%%%%%%%%%%%%%%%%%%%%%%%%%%%%%%%%%%%%%%%%%%%%%%%%%%%%%%%%%%%%%
% Add all appendices in appendices/main.tex with include statements.
% see appendices/sub-example-app.tex for an example.
\onecolumngrid
\documentclass[%
    reprint,
    superscriptaddress,
    %groupedaddress,
    %unsortedaddress,
    %runinaddress,
    %frontmatterverbose,
    %preprint,
    longbibliography,
    bibnotes,
    %showpacs,preprintnumbers,
    %nofootinbib,
    %nobibnotes,
    %bibnotes,
     amsmath,amssymb,
     %aps,
     aip,
     jcp,                                       % J. Chem. Phys
    %pra,
    %prb,
    %rmp,
    %prstab,
    %prstper,
    %floatfix,
    ]{revtex4-1}
%%%%%%%%%%%%%%%%%%%%%%%%%%%%%%%%%%%%%%%%%%%%%%%%%%%%%%%%%%%%%%%%%%%%%%%%%%%%%%%%
%					 	                       LIBRARIES							                     %
%%%%%%%%%%%%%%%%%%%%%%%%%%%%%%%%%%%%%%%%%%%%%%%%%%%%%%%%%%%%%%%%%%%%%%%%%%%%%%%%
% Add whatever libraries you need here. Add a description if you want
\usepackage{graphicx}			 % Include figure files
\usepackage{dcolumn}			 % Align table columns on decimal point
\usepackage{tcolorbox}		 % use tcolorbox environment to box equations
\usepackage[USenglish]{babel}
\usepackage[useregional]{datetime2}
\usepackage[utf8]{inputenc}
\usepackage[T1]{fontenc}
\usepackage{enumerate}
\usepackage{caption}
\usepackage{subcaption}
\usepackage{physics}
\usepackage{hyperref}
\usepackage{chemformula}
\usepackage{booktabs}
\usepackage{makecell}
\usepackage{fullpage}
\usepackage{natbib}
\usepackage{setspace}
\usepackage{amsmath}
\usepackage{bm}              % enchanced bold math symbols
\usepackage{amssymb}
\usepackage{listings} 			 % For code citations
\usepackage{amsfonts}
\usepackage{mathtools}
\usepackage{commath}
\usepackage{fancyhdr}
\usepackage{lastpage}
\usepackage{tikz}
\usepackage{graphicx}
	\graphicspath{ {images/} }
\usepackage{feynmp-auto}
\usepackage[toc,page]{appendix}
%%%%%%%%%%%%%%%%%%%%%%%%%%%%%%%%%%%%%%%%%%%%%%%%%%%%%%%%%%%%%%%%%%%%%%%%%%%%%%%%
%					 	                      NEW COMMANDS							                   %
%%%%%%%%%%%%%%%%%%%%%%%%%%%%%%%%%%%%%%%%%%%%%%%%%%%%%%%%%%%%%%%%%%%%%%%%%%%%%%%%
% Redefine commands if necessary
\renewcommand\vec{\mathbf}                      % use boldface for vectors
% New commands
%------------------------------------Date---------------------------------------
\newcommand{\seminardate}{\DTMdisplaydate{2017}{7}{25}{-1}}
\newcommand{\revisiondate}{\DTMdisplaydate{2017}{7}{26}{-1}}
%-------------------------------------------------------------------------------
\newcommand*\Del{\vec{\nabla}}                  % del operator
\newcommand*\diff{\mathop{}\!\mathrm{d}}		    % differential d
\newcommand*\Diff[1]{\mathop{}\!\mathrm{d^#1}}	% d^(...)
\newcommand*{\scalarnorm}[1]                    % scalar norm |*|
  {\left\lvert#1\right\rvert}
\newcommand*{\vectornorm}[1]                    % vector norm ||*||
  {\left\lVert#1\right\rVert}
\newcommand*\onehalf{\frac{1}{2}}               % fraction 1/2
%------------------------------------Time---------------------------------------
\newcommand*\dlt[1]{\delta #1}
\newcommand*\timestep{\dlt{t}}              % timestep dt
\newcommand*\halfstep                       % half timestep 1/2 dt
  {\onehalf \timestep}
\newcommand*\timeInHalfStep{\left(t + \halfstep\right)}
\newcommand*\timeInFullStep{\left(t + \timestep\right)}
%-------------------------------------------------------------------------------
%%%%%%%%%%%%%%%%%%%%%%%%%%%%%%%%%%%%%%%%%%%%%%%%%%%%%%%%%%%%%%%%%%%%%%%%%%%%%%%%
%					 	                    DEFINITIONS								                     %
%%%%%%%%%%%%%%%%%%%%%%%%%%%%%%%%%%%%%%%%%%%%%%%%%%%%%%%%%%%%%%%%%%%%%%%%%%%%%%%%
%-------------------------------------------------------------------------------
\newcommand*\rattle{\textsf{RATTLE} }       % display name of algorithm
%-------------------------------------------------------------------------------
\newcommand*\rattlePos{\vec{r}}             % position vector
\newcommand*\rattleVel{\vec{v}}             % velocity vector
\newcommand*\rattleAcc{\vec{a}}             % acceleration vector
%-------------------------------------------------------------------------------
\newcommand*\rattleMassNoIndex{m}
\newcommand*\rattleForceNoIndex{\vec{F}}
\newcommand*\rattleConstraintForceNoIndex{\vec{G}}
\newcommand*\rattleConstraintNoIndex{\sigma}
\newcommand*\rattleLagrangeNoIndex{\lambda}
\newcommand*\rattleLagrangeApproxNoIndex{\gamma}
\newcommand*\rattleCorrectiveValueNoIndex{g}
%---------------------Definitions for General Formulation ----------------------
\input{rattle/general/definitions}
%-----------------------------------Indexing------------------------------------
\newcommand*\rattleAtomIndex{a}               % solo atom index
\newcommand*\rattleAtomIndexFirst{a}          % first index in atomic pair
\newcommand*\rattleAtomIndexSecond{b}         % second index in atomic pair
\newcommand*\rattleNAtoms{N}                  % number of atoms
\newcommand*\rattleConstraintSet{K}           % set of constraints
\newcommand*\rattleConstraintIndex            % index for constraints
  {\left(
    \rattleAtomIndexFirst,\,\rattleAtomIndexSecond
  \right) \in \rattleConstraintSet}
\newcommand*\rattleConstraintIndexSimple      % simplified index for constraints
  {\rattleAtomIndexFirst\rattleAtomIndexSecond}
\newcommand*\rattleCorrectionIndex{i}         % index for iterative correction
\newcommand*\rattleCorrectionIndexThis        % current for iterative correction
  {\rattleCorrectionIndex}
\newcommand*\rattleCorrectionIndexNext        % next for iterative correction
    {\rattleCorrectionIndex + 1}
\newcommand*\rattleCorrectionIndexLast{m}     % last step of correction
\newcommand*\rattleNCorrections{M}            % number of iterative corrections
%-----------------------------------Dynamics------------------------------------
\newcommand*\rattleMass                       % solo atomic mass
  {\rattleMassNoIndex_{\rattleAtomIndex}}
\newcommand*\rattleMassFirst                  % first atomic mass in pair
    {\rattleMassNoIndex_{\rattleAtomIndexFirst}}
\newcommand*\rattleMassSecond                 % second atomic mass in pair
    {\rattleMassNoIndex_{\rattleAtomIndexSecond}}
\newcommand*\rattleForce
  {\rattleForceNoIndex_{\rattleAtomIndex}}
\newcommand*\rattleConstraintForce            % solo constraint force
  {\rattleConstraintForceNoIndex_{\rattleAtomIndex}}
\newcommand*\rattleConstraintForceFirst       % first constraint force in pair
    {\rattleConstraintForceNoIndex_{\rattleAtomIndexFirst}}
\newcommand*\rattleConstraintForceSecond      % second constraint force in pair
    {\rattleConstraintForceNoIndex_{\rattleAtomIndexSecond}}
\newcommand*\rattleConstraintForceBond        % constraint force along bond
        {\rattleConstraintForceNoIndex_{\rattleConstraintIndexSimple}}
\newcommand*\rattleConstraint
  {\rattleConstraintNoIndex_{\rattleConstraintIndexSimple}}
  \newcommand*\rattleConstraintDot
    {\dot{\rattleConstraintNoIndex}_{\rattleConstraintIndexSimple}}
\newcommand*\rattleLagrange
    {\rattleLagrangeNoIndex_{\rattleConstraintIndexSimple}}
\newcommand*\rattleLagrangeApprox
        {\rattleLagrangeApproxNoIndex_{\rattleConstraintIndexSimple}}
\newcommand*\rattleCorrectiveValue
  {\rattleCorrectiveValueNoIndex_{\rattleConstraintIndexSimple}}
%-----------------------------IterativeCorrection-------------------------------
\newcommand*\iteration[2]{#1^{(#2)}}            % iterative correction steps
\newcommand*\iterationThis[1]                   % arg^{i}
  {\iteration{#1}{\rattleCorrectionIndexThis}}
\newcommand*\iterationNext[1]                   % arg^{i+1}
  {\iteration{#1}{\rattleCorrectionIndexNext}}
%----------------------------------Vectors--------------------------------------
\newcommand*\rattleAtomPos{\rattlePos_{\rattleAtomIndex}}
\newcommand*\rattleAtomPosFirst{\rattlePos_{\rattleAtomIndexFirst}}
\newcommand*\rattleAtomPosSecond{\rattlePos_{\rattleAtomIndexSecond}}
\newcommand*\rattleAtomVel{\rattleVel_{\rattleAtomIndex}}
\newcommand*\rattleAtomVelFirst{\rattleVel_{\rattleAtomIndexFirst}}
\newcommand*\rattleAtomVelSecond{\rattleVel_{\rattleAtomIndexSecond}}
\newcommand*\rattlePosUnconstrained             % r^{(0)}
  {\iteration{\rattlePos}{0}}
\newcommand*\rattleVelUnconstrained             % v^{(0)}
  {\iteration{\rattleVel}{0}}
\newcommand*\rattleAtomPosUnconstrained         % r^{(0)}
  {\iteration{\rattleAtomPos}{0}}
\newcommand*\rattleAtomVelUnconstrained         % v^{(0)}
  {\iteration{\rattleAtomVel}{0}}
\newcommand*\rattleBondPos{\rattlePos_{\rattleConstraintIndexSimple}}
\newcommand*\rattleBondPosUnit{\hat{\rattlePos}_{\rattleConstraintIndexSimple}}
\newcommand*\rattleBondVel{\rattleVel_{\rattleConstraintIndexSimple}}
\newcommand*\rattleBondVelUnit{\hat{\rattleVel}_{\rattleConstraintIndexSimple}}
\newcommand*\rattleBondPosUnconstrained
  {\iteration{\rattleBondPos}{0}}
\newcommand*\rattleBondVelUnconstrained
  {\iteration{\rattleBondVel}{0}}
\newcommand*\rattleAtomPosUnconstrainedFirst
    {\iteration{\rattleAtomPosFirst}{0}}
\newcommand*\rattleAtomPosUnconstrainedSecond
    {\iteration{\rattleAtomPosSecond}{0}}
\newcommand*\rattleAtomVelUnconstrainedFirst
    {\iteration{\rattleAtomVelFirst}{0}}
\newcommand*\rattleAtomVelUnconstrainedSecond
    {\iteration{\rattleAtomVelSecond}{0}}
%---------------------------------Constants-------------------------------------
\newcommand*\rattleBondlength                   % fixed bond length
  {d_{\rattleConstraintIndexSimple}}
\newcommand*\rattleTol{\xi}                     % d-r/d tolerance
\newcommand*\rattleRRTol{\varepsilon}           % r(t) * r(t+dt) tolerance
\newcommand*\rattleRVTol{\rattleTol}            % v(t)/d tolerance
%-------------------------------------------------------------------------------

%%%%%%%%%%%%%%%%%%%%%%%%%%%%%%%%%%%%%%%%%%%%%%%%%%%%%%%%%%%%%%%%%%%%%%%%%%%%%%%%
%					 	                    DEFINITIONS								                     %
%%%%%%%%%%%%%%%%%%%%%%%%%%%%%%%%%%%%%%%%%%%%%%%%%%%%%%%%%%%%%%%%%%%%%%%%%%%%%%%%
%-------------------------------------------------------------------------------
\newcommand*\rattle{\textsf{RATTLE} }       % display name of algorithm
%-------------------------------------------------------------------------------
\newcommand*\rattlePos{\vec{r}}             % position vector
\newcommand*\rattleVel{\vec{v}}             % velocity vector
\newcommand*\rattleAcc{\vec{a}}             % acceleration vector
%-------------------------------------------------------------------------------
\newcommand*\rattleMassNoIndex{m}
\newcommand*\rattleForceNoIndex{\vec{F}}
\newcommand*\rattleConstraintForceNoIndex{\vec{G}}
\newcommand*\rattleConstraintNoIndex{\sigma}
\newcommand*\rattleLagrangeNoIndex{\lambda}
\newcommand*\rattleLagrangeApproxNoIndex{\gamma}
\newcommand*\rattleCorrectiveValueNoIndex{g}
%---------------------Definitions for General Formulation ----------------------
\input{rattle/general/definitions}
%-----------------------------------Indexing------------------------------------
\newcommand*\rattleAtomIndex{a}               % solo atom index
\newcommand*\rattleAtomIndexFirst{a}          % first index in atomic pair
\newcommand*\rattleAtomIndexSecond{b}         % second index in atomic pair
\newcommand*\rattleNAtoms{N}                  % number of atoms
\newcommand*\rattleConstraintSet{K}           % set of constraints
\newcommand*\rattleConstraintIndex            % index for constraints
  {\left(
    \rattleAtomIndexFirst,\,\rattleAtomIndexSecond
  \right) \in \rattleConstraintSet}
\newcommand*\rattleConstraintIndexSimple      % simplified index for constraints
  {\rattleAtomIndexFirst\rattleAtomIndexSecond}
\newcommand*\rattleCorrectionIndex{i}         % index for iterative correction
\newcommand*\rattleCorrectionIndexThis        % current for iterative correction
  {\rattleCorrectionIndex}
\newcommand*\rattleCorrectionIndexNext        % next for iterative correction
    {\rattleCorrectionIndex + 1}
\newcommand*\rattleCorrectionIndexLast{m}     % last step of correction
\newcommand*\rattleNCorrections{M}            % number of iterative corrections
%-----------------------------------Dynamics------------------------------------
\newcommand*\rattleMass                       % solo atomic mass
  {\rattleMassNoIndex_{\rattleAtomIndex}}
\newcommand*\rattleMassFirst                  % first atomic mass in pair
    {\rattleMassNoIndex_{\rattleAtomIndexFirst}}
\newcommand*\rattleMassSecond                 % second atomic mass in pair
    {\rattleMassNoIndex_{\rattleAtomIndexSecond}}
\newcommand*\rattleForce
  {\rattleForceNoIndex_{\rattleAtomIndex}}
\newcommand*\rattleConstraintForce            % solo constraint force
  {\rattleConstraintForceNoIndex_{\rattleAtomIndex}}
\newcommand*\rattleConstraintForceFirst       % first constraint force in pair
    {\rattleConstraintForceNoIndex_{\rattleAtomIndexFirst}}
\newcommand*\rattleConstraintForceSecond      % second constraint force in pair
    {\rattleConstraintForceNoIndex_{\rattleAtomIndexSecond}}
\newcommand*\rattleConstraintForceBond        % constraint force along bond
        {\rattleConstraintForceNoIndex_{\rattleConstraintIndexSimple}}
\newcommand*\rattleConstraint
  {\rattleConstraintNoIndex_{\rattleConstraintIndexSimple}}
  \newcommand*\rattleConstraintDot
    {\dot{\rattleConstraintNoIndex}_{\rattleConstraintIndexSimple}}
\newcommand*\rattleLagrange
    {\rattleLagrangeNoIndex_{\rattleConstraintIndexSimple}}
\newcommand*\rattleLagrangeApprox
        {\rattleLagrangeApproxNoIndex_{\rattleConstraintIndexSimple}}
\newcommand*\rattleCorrectiveValue
  {\rattleCorrectiveValueNoIndex_{\rattleConstraintIndexSimple}}
%-----------------------------IterativeCorrection-------------------------------
\newcommand*\iteration[2]{#1^{(#2)}}            % iterative correction steps
\newcommand*\iterationThis[1]                   % arg^{i}
  {\iteration{#1}{\rattleCorrectionIndexThis}}
\newcommand*\iterationNext[1]                   % arg^{i+1}
  {\iteration{#1}{\rattleCorrectionIndexNext}}
%----------------------------------Vectors--------------------------------------
\newcommand*\rattleAtomPos{\rattlePos_{\rattleAtomIndex}}
\newcommand*\rattleAtomPosFirst{\rattlePos_{\rattleAtomIndexFirst}}
\newcommand*\rattleAtomPosSecond{\rattlePos_{\rattleAtomIndexSecond}}
\newcommand*\rattleAtomVel{\rattleVel_{\rattleAtomIndex}}
\newcommand*\rattleAtomVelFirst{\rattleVel_{\rattleAtomIndexFirst}}
\newcommand*\rattleAtomVelSecond{\rattleVel_{\rattleAtomIndexSecond}}
\newcommand*\rattlePosUnconstrained             % r^{(0)}
  {\iteration{\rattlePos}{0}}
\newcommand*\rattleVelUnconstrained             % v^{(0)}
  {\iteration{\rattleVel}{0}}
\newcommand*\rattleAtomPosUnconstrained         % r^{(0)}
  {\iteration{\rattleAtomPos}{0}}
\newcommand*\rattleAtomVelUnconstrained         % v^{(0)}
  {\iteration{\rattleAtomVel}{0}}
\newcommand*\rattleBondPos{\rattlePos_{\rattleConstraintIndexSimple}}
\newcommand*\rattleBondPosUnit{\hat{\rattlePos}_{\rattleConstraintIndexSimple}}
\newcommand*\rattleBondVel{\rattleVel_{\rattleConstraintIndexSimple}}
\newcommand*\rattleBondVelUnit{\hat{\rattleVel}_{\rattleConstraintIndexSimple}}
\newcommand*\rattleBondPosUnconstrained
  {\iteration{\rattleBondPos}{0}}
\newcommand*\rattleBondVelUnconstrained
  {\iteration{\rattleBondVel}{0}}
\newcommand*\rattleAtomPosUnconstrainedFirst
    {\iteration{\rattleAtomPosFirst}{0}}
\newcommand*\rattleAtomPosUnconstrainedSecond
    {\iteration{\rattleAtomPosSecond}{0}}
\newcommand*\rattleAtomVelUnconstrainedFirst
    {\iteration{\rattleAtomVelFirst}{0}}
\newcommand*\rattleAtomVelUnconstrainedSecond
    {\iteration{\rattleAtomVelSecond}{0}}
%---------------------------------Constants-------------------------------------
\newcommand*\rattleBondlength                   % fixed bond length
  {d_{\rattleConstraintIndexSimple}}
\newcommand*\rattleTol{\xi}                     % d-r/d tolerance
\newcommand*\rattleRRTol{\varepsilon}           % r(t) * r(t+dt) tolerance
\newcommand*\rattleRVTol{\rattleTol}            % v(t)/d tolerance
%-------------------------------------------------------------------------------

%-------------------------------------------------------------------------------
%\setlength{\parindent}{0pt}						          % do not indent
\numberwithin{equation}{section}                % number eqs within sections
\begin{document}
\lstset{language=Python}
%%%%%%%%%%%%%%%%%%%%%%%%%%%%%%%%%%%%%%%%%%%%%%%%%%%%%%%%%%%%%%%%%%%%%%%%%%%%%%%%
%					 	                        TITLE							                         %
%%%%%%%%%%%%%%%%%%%%%%%%%%%%%%%%%%%%%%%%%%%%%%%%%%%%%%%%%%%%%%%%%%%%%%%%%%%%%%%%
\title{Group Seminar (revised: \revisiondate)}
\thanks{Professor Stratt, Vale Cofer-Shabica, Yan Zhao, Andrew Ton, Mansheej Paul, Evan Coleman}
\author{Artur Avkhadiev}
\email{artur\_avkhadiev@brown.edu}
\affiliation{Department of Physics, Brown University, Providence RI 02912, USA}
\date{\seminardate}
%%%%%%%%%%%%%%%%%%%%%%%%%%%%%%%%%%%%%%%%%%%%%%%%%%%%%%%%%%%%%%%%%%%%%%%%%%%%%%%%
%					 	                       ABSTRACT							                       %
%%%%%%%%%%%%%%%%%%%%%%%%%%%%%%%%%%%%%%%%%%%%%%%%%%%%%%%%%%%%%%%%%%%%%%%%%%%%%%%%
\begin{abstract}
  This report consist of two parts:
  \begin{enumerate}
      \item A discussion of the constraint dynamics algorithm based on velocity Verlet numerical integration scheme, \rattle.
      \item A calculation of the kinetic energy of a freely-jointed polymer chain using the bond-vector representation in the CM frame of the molecule.
    \end{enumerate}
\end{abstract}
\maketitle
\tableofcontents						 % hide table of contents
%%%%%%%%%%%%%%%%%%%%%%%%%%%%%%%%%%%%%%%%%%%%%%%%%%%%%%%%%%%%%%%%%%%%%%%%%%%%%%%%
%					 	                        BODY							                         %
%%%%%%%%%%%%%%%%%%%%%%%%%%%%%%%%%%%%%%%%%%%%%%%%%%%%%%%%%%%%%%%%%%%%%%%%%%%%%%%%
% Add include statements below. All the text should be in the folders.
\documentclass[%
    reprint,
    superscriptaddress,
    %groupedaddress,
    %unsortedaddress,
    %runinaddress,
    %frontmatterverbose,
    %preprint,
    longbibliography,
    bibnotes,
    %showpacs,preprintnumbers,
    %nofootinbib,
    %nobibnotes,
    %bibnotes,
     amsmath,amssymb,
     %aps,
     aip,
     jcp,                                       % J. Chem. Phys
    %pra,
    %prb,
    %rmp,
    %prstab,
    %prstper,
    %floatfix,
    ]{revtex4-1}
%%%%%%%%%%%%%%%%%%%%%%%%%%%%%%%%%%%%%%%%%%%%%%%%%%%%%%%%%%%%%%%%%%%%%%%%%%%%%%%%
%					 	                       LIBRARIES							                     %
%%%%%%%%%%%%%%%%%%%%%%%%%%%%%%%%%%%%%%%%%%%%%%%%%%%%%%%%%%%%%%%%%%%%%%%%%%%%%%%%
% Add whatever libraries you need here. Add a description if you want
\usepackage{graphicx}			 % Include figure files
\usepackage{dcolumn}			 % Align table columns on decimal point
\usepackage{tcolorbox}		 % use tcolorbox environment to box equations
\usepackage[USenglish]{babel}
\usepackage[useregional]{datetime2}
\usepackage[utf8]{inputenc}
\usepackage[T1]{fontenc}
\usepackage{enumerate}
\usepackage{caption}
\usepackage{subcaption}
\usepackage{physics}
\usepackage{hyperref}
\usepackage{chemformula}
\usepackage{booktabs}
\usepackage{makecell}
\usepackage{fullpage}
\usepackage{natbib}
\usepackage{setspace}
\usepackage{amsmath}
\usepackage{bm}              % enchanced bold math symbols
\usepackage{amssymb}
\usepackage{listings} 			 % For code citations
\usepackage{amsfonts}
\usepackage{mathtools}
\usepackage{commath}
\usepackage{fancyhdr}
\usepackage{lastpage}
\usepackage{tikz}
\usepackage{graphicx}
	\graphicspath{ {images/} }
\usepackage{feynmp-auto}
\usepackage[toc,page]{appendix}
%%%%%%%%%%%%%%%%%%%%%%%%%%%%%%%%%%%%%%%%%%%%%%%%%%%%%%%%%%%%%%%%%%%%%%%%%%%%%%%%
%					 	                      NEW COMMANDS							                   %
%%%%%%%%%%%%%%%%%%%%%%%%%%%%%%%%%%%%%%%%%%%%%%%%%%%%%%%%%%%%%%%%%%%%%%%%%%%%%%%%
% Redefine commands if necessary
\renewcommand\vec{\mathbf}                      % use boldface for vectors
% New commands
%------------------------------------Date---------------------------------------
\newcommand{\seminardate}{\DTMdisplaydate{2017}{7}{25}{-1}}
\newcommand{\revisiondate}{\DTMdisplaydate{2017}{7}{26}{-1}}
%-------------------------------------------------------------------------------
\newcommand*\Del{\vec{\nabla}}                  % del operator
\newcommand*\diff{\mathop{}\!\mathrm{d}}		    % differential d
\newcommand*\Diff[1]{\mathop{}\!\mathrm{d^#1}}	% d^(...)
\newcommand*{\scalarnorm}[1]                    % scalar norm |*|
  {\left\lvert#1\right\rvert}
\newcommand*{\vectornorm}[1]                    % vector norm ||*||
  {\left\lVert#1\right\rVert}
\newcommand*\onehalf{\frac{1}{2}}               % fraction 1/2
%------------------------------------Time---------------------------------------
\newcommand*\dlt[1]{\delta #1}
\newcommand*\timestep{\dlt{t}}              % timestep dt
\newcommand*\halfstep                       % half timestep 1/2 dt
  {\onehalf \timestep}
\newcommand*\timeInHalfStep{\left(t + \halfstep\right)}
\newcommand*\timeInFullStep{\left(t + \timestep\right)}
%-------------------------------------------------------------------------------
\input{rattle/definitions}
\input{ke_polymer/definitions}
%-------------------------------------------------------------------------------
%\setlength{\parindent}{0pt}						          % do not indent
\numberwithin{equation}{section}                % number eqs within sections
\begin{document}
\lstset{language=Python}
%%%%%%%%%%%%%%%%%%%%%%%%%%%%%%%%%%%%%%%%%%%%%%%%%%%%%%%%%%%%%%%%%%%%%%%%%%%%%%%%
%					 	                        TITLE							                         %
%%%%%%%%%%%%%%%%%%%%%%%%%%%%%%%%%%%%%%%%%%%%%%%%%%%%%%%%%%%%%%%%%%%%%%%%%%%%%%%%
\title{Group Seminar (revised: \revisiondate)}
\thanks{Professor Stratt, Vale Cofer-Shabica, Yan Zhao, Andrew Ton, Mansheej Paul, Evan Coleman}
\author{Artur Avkhadiev}
\email{artur\_avkhadiev@brown.edu}
\affiliation{Department of Physics, Brown University, Providence RI 02912, USA}
\date{\seminardate}
%%%%%%%%%%%%%%%%%%%%%%%%%%%%%%%%%%%%%%%%%%%%%%%%%%%%%%%%%%%%%%%%%%%%%%%%%%%%%%%%
%					 	                       ABSTRACT							                       %
%%%%%%%%%%%%%%%%%%%%%%%%%%%%%%%%%%%%%%%%%%%%%%%%%%%%%%%%%%%%%%%%%%%%%%%%%%%%%%%%
\begin{abstract}
  This report consist of two parts:
  \begin{enumerate}
      \item A discussion of the constraint dynamics algorithm based on velocity Verlet numerical integration scheme, \rattle.
      \item A calculation of the kinetic energy of a freely-jointed polymer chain using the bond-vector representation in the CM frame of the molecule.
    \end{enumerate}
\end{abstract}
\maketitle
\tableofcontents						 % hide table of contents
%%%%%%%%%%%%%%%%%%%%%%%%%%%%%%%%%%%%%%%%%%%%%%%%%%%%%%%%%%%%%%%%%%%%%%%%%%%%%%%%
%					 	                        BODY							                         %
%%%%%%%%%%%%%%%%%%%%%%%%%%%%%%%%%%%%%%%%%%%%%%%%%%%%%%%%%%%%%%%%%%%%%%%%%%%%%%%%
% Add include statements below. All the text should be in the folders.
\input{rattle/main}
\input{ke_polymer/main}
%-------------------------------------------------------------------------------
%%%%%%%%%%%%%%%%%%%%%%%%%%%%%%%%%%%%%%%%%%%%%%%%%%%%%%%%%%%%%%%%%%%%%%%%%%%%%%%%
%					                        BIBLIOGRAPHY							                   %
%%%%%%%%%%%%%%%%%%%%%%%%%%%%%%%%%%%%%%%%%%%%%%%%%%%%%%%%%%%%%%%%%%%%%%%%%%%%%%%%
% The bibliography file is reference,bib. If you need to add a
% reference, make sure all the entries look good (e.g., ensure
% proper capitalization by putting {} around the letters to be
% capitalized. Reference the books by chapters!
\nocite{*}
\bibliographystyle{apalike}
\bibliography{reference}
%-----------------------------
%%%%%%%%%%%%%%%%%%%%%%%%%%%%%%%%%%%%%%%%%%%%%%%%%%%%%%%%%%%%%%%%%%%%%%%%%%%%%%%%
%					 	                      APPENDICES							                     %
%%%%%%%%%%%%%%%%%%%%%%%%%%%%%%%%%%%%%%%%%%%%%%%%%%%%%%%%%%%%%%%%%%%%%%%%%%%%%%%%
% Add all appendices in appendices/main.tex with include statements.
% see appendices/sub-example-app.tex for an example.
\onecolumngrid
\input{appendices/main}
%-------------------------------------------------------------------------------
\end{document}

\documentclass[%
    reprint,
    superscriptaddress,
    %groupedaddress,
    %unsortedaddress,
    %runinaddress,
    %frontmatterverbose,
    %preprint,
    longbibliography,
    bibnotes,
    %showpacs,preprintnumbers,
    %nofootinbib,
    %nobibnotes,
    %bibnotes,
     amsmath,amssymb,
     %aps,
     aip,
     jcp,                                       % J. Chem. Phys
    %pra,
    %prb,
    %rmp,
    %prstab,
    %prstper,
    %floatfix,
    ]{revtex4-1}
%%%%%%%%%%%%%%%%%%%%%%%%%%%%%%%%%%%%%%%%%%%%%%%%%%%%%%%%%%%%%%%%%%%%%%%%%%%%%%%%
%					 	                       LIBRARIES							                     %
%%%%%%%%%%%%%%%%%%%%%%%%%%%%%%%%%%%%%%%%%%%%%%%%%%%%%%%%%%%%%%%%%%%%%%%%%%%%%%%%
% Add whatever libraries you need here. Add a description if you want
\usepackage{graphicx}			 % Include figure files
\usepackage{dcolumn}			 % Align table columns on decimal point
\usepackage{tcolorbox}		 % use tcolorbox environment to box equations
\usepackage[USenglish]{babel}
\usepackage[useregional]{datetime2}
\usepackage[utf8]{inputenc}
\usepackage[T1]{fontenc}
\usepackage{enumerate}
\usepackage{caption}
\usepackage{subcaption}
\usepackage{physics}
\usepackage{hyperref}
\usepackage{chemformula}
\usepackage{booktabs}
\usepackage{makecell}
\usepackage{fullpage}
\usepackage{natbib}
\usepackage{setspace}
\usepackage{amsmath}
\usepackage{bm}              % enchanced bold math symbols
\usepackage{amssymb}
\usepackage{listings} 			 % For code citations
\usepackage{amsfonts}
\usepackage{mathtools}
\usepackage{commath}
\usepackage{fancyhdr}
\usepackage{lastpage}
\usepackage{tikz}
\usepackage{graphicx}
	\graphicspath{ {images/} }
\usepackage{feynmp-auto}
\usepackage[toc,page]{appendix}
%%%%%%%%%%%%%%%%%%%%%%%%%%%%%%%%%%%%%%%%%%%%%%%%%%%%%%%%%%%%%%%%%%%%%%%%%%%%%%%%
%					 	                      NEW COMMANDS							                   %
%%%%%%%%%%%%%%%%%%%%%%%%%%%%%%%%%%%%%%%%%%%%%%%%%%%%%%%%%%%%%%%%%%%%%%%%%%%%%%%%
% Redefine commands if necessary
\renewcommand\vec{\mathbf}                      % use boldface for vectors
% New commands
%------------------------------------Date---------------------------------------
\newcommand{\seminardate}{\DTMdisplaydate{2017}{7}{25}{-1}}
\newcommand{\revisiondate}{\DTMdisplaydate{2017}{7}{26}{-1}}
%-------------------------------------------------------------------------------
\newcommand*\Del{\vec{\nabla}}                  % del operator
\newcommand*\diff{\mathop{}\!\mathrm{d}}		    % differential d
\newcommand*\Diff[1]{\mathop{}\!\mathrm{d^#1}}	% d^(...)
\newcommand*{\scalarnorm}[1]                    % scalar norm |*|
  {\left\lvert#1\right\rvert}
\newcommand*{\vectornorm}[1]                    % vector norm ||*||
  {\left\lVert#1\right\rVert}
\newcommand*\onehalf{\frac{1}{2}}               % fraction 1/2
%------------------------------------Time---------------------------------------
\newcommand*\dlt[1]{\delta #1}
\newcommand*\timestep{\dlt{t}}              % timestep dt
\newcommand*\halfstep                       % half timestep 1/2 dt
  {\onehalf \timestep}
\newcommand*\timeInHalfStep{\left(t + \halfstep\right)}
\newcommand*\timeInFullStep{\left(t + \timestep\right)}
%-------------------------------------------------------------------------------
\input{rattle/definitions}
\input{ke_polymer/definitions}
%-------------------------------------------------------------------------------
%\setlength{\parindent}{0pt}						          % do not indent
\numberwithin{equation}{section}                % number eqs within sections
\begin{document}
\lstset{language=Python}
%%%%%%%%%%%%%%%%%%%%%%%%%%%%%%%%%%%%%%%%%%%%%%%%%%%%%%%%%%%%%%%%%%%%%%%%%%%%%%%%
%					 	                        TITLE							                         %
%%%%%%%%%%%%%%%%%%%%%%%%%%%%%%%%%%%%%%%%%%%%%%%%%%%%%%%%%%%%%%%%%%%%%%%%%%%%%%%%
\title{Group Seminar (revised: \revisiondate)}
\thanks{Professor Stratt, Vale Cofer-Shabica, Yan Zhao, Andrew Ton, Mansheej Paul, Evan Coleman}
\author{Artur Avkhadiev}
\email{artur\_avkhadiev@brown.edu}
\affiliation{Department of Physics, Brown University, Providence RI 02912, USA}
\date{\seminardate}
%%%%%%%%%%%%%%%%%%%%%%%%%%%%%%%%%%%%%%%%%%%%%%%%%%%%%%%%%%%%%%%%%%%%%%%%%%%%%%%%
%					 	                       ABSTRACT							                       %
%%%%%%%%%%%%%%%%%%%%%%%%%%%%%%%%%%%%%%%%%%%%%%%%%%%%%%%%%%%%%%%%%%%%%%%%%%%%%%%%
\begin{abstract}
  This report consist of two parts:
  \begin{enumerate}
      \item A discussion of the constraint dynamics algorithm based on velocity Verlet numerical integration scheme, \rattle.
      \item A calculation of the kinetic energy of a freely-jointed polymer chain using the bond-vector representation in the CM frame of the molecule.
    \end{enumerate}
\end{abstract}
\maketitle
\tableofcontents						 % hide table of contents
%%%%%%%%%%%%%%%%%%%%%%%%%%%%%%%%%%%%%%%%%%%%%%%%%%%%%%%%%%%%%%%%%%%%%%%%%%%%%%%%
%					 	                        BODY							                         %
%%%%%%%%%%%%%%%%%%%%%%%%%%%%%%%%%%%%%%%%%%%%%%%%%%%%%%%%%%%%%%%%%%%%%%%%%%%%%%%%
% Add include statements below. All the text should be in the folders.
\input{rattle/main}
\input{ke_polymer/main}
%-------------------------------------------------------------------------------
%%%%%%%%%%%%%%%%%%%%%%%%%%%%%%%%%%%%%%%%%%%%%%%%%%%%%%%%%%%%%%%%%%%%%%%%%%%%%%%%
%					                        BIBLIOGRAPHY							                   %
%%%%%%%%%%%%%%%%%%%%%%%%%%%%%%%%%%%%%%%%%%%%%%%%%%%%%%%%%%%%%%%%%%%%%%%%%%%%%%%%
% The bibliography file is reference,bib. If you need to add a
% reference, make sure all the entries look good (e.g., ensure
% proper capitalization by putting {} around the letters to be
% capitalized. Reference the books by chapters!
\nocite{*}
\bibliographystyle{apalike}
\bibliography{reference}
%-----------------------------
%%%%%%%%%%%%%%%%%%%%%%%%%%%%%%%%%%%%%%%%%%%%%%%%%%%%%%%%%%%%%%%%%%%%%%%%%%%%%%%%
%					 	                      APPENDICES							                     %
%%%%%%%%%%%%%%%%%%%%%%%%%%%%%%%%%%%%%%%%%%%%%%%%%%%%%%%%%%%%%%%%%%%%%%%%%%%%%%%%
% Add all appendices in appendices/main.tex with include statements.
% see appendices/sub-example-app.tex for an example.
\onecolumngrid
\input{appendices/main}
%-------------------------------------------------------------------------------
\end{document}

%-------------------------------------------------------------------------------
%%%%%%%%%%%%%%%%%%%%%%%%%%%%%%%%%%%%%%%%%%%%%%%%%%%%%%%%%%%%%%%%%%%%%%%%%%%%%%%%
%					                        BIBLIOGRAPHY							                   %
%%%%%%%%%%%%%%%%%%%%%%%%%%%%%%%%%%%%%%%%%%%%%%%%%%%%%%%%%%%%%%%%%%%%%%%%%%%%%%%%
% The bibliography file is reference,bib. If you need to add a
% reference, make sure all the entries look good (e.g., ensure
% proper capitalization by putting {} around the letters to be
% capitalized. Reference the books by chapters!
\nocite{*}
\bibliographystyle{apalike}
\bibliography{reference}
%-----------------------------
%%%%%%%%%%%%%%%%%%%%%%%%%%%%%%%%%%%%%%%%%%%%%%%%%%%%%%%%%%%%%%%%%%%%%%%%%%%%%%%%
%					 	                      APPENDICES							                     %
%%%%%%%%%%%%%%%%%%%%%%%%%%%%%%%%%%%%%%%%%%%%%%%%%%%%%%%%%%%%%%%%%%%%%%%%%%%%%%%%
% Add all appendices in appendices/main.tex with include statements.
% see appendices/sub-example-app.tex for an example.
\onecolumngrid
\documentclass[%
    reprint,
    superscriptaddress,
    %groupedaddress,
    %unsortedaddress,
    %runinaddress,
    %frontmatterverbose,
    %preprint,
    longbibliography,
    bibnotes,
    %showpacs,preprintnumbers,
    %nofootinbib,
    %nobibnotes,
    %bibnotes,
     amsmath,amssymb,
     %aps,
     aip,
     jcp,                                       % J. Chem. Phys
    %pra,
    %prb,
    %rmp,
    %prstab,
    %prstper,
    %floatfix,
    ]{revtex4-1}
%%%%%%%%%%%%%%%%%%%%%%%%%%%%%%%%%%%%%%%%%%%%%%%%%%%%%%%%%%%%%%%%%%%%%%%%%%%%%%%%
%					 	                       LIBRARIES							                     %
%%%%%%%%%%%%%%%%%%%%%%%%%%%%%%%%%%%%%%%%%%%%%%%%%%%%%%%%%%%%%%%%%%%%%%%%%%%%%%%%
% Add whatever libraries you need here. Add a description if you want
\usepackage{graphicx}			 % Include figure files
\usepackage{dcolumn}			 % Align table columns on decimal point
\usepackage{tcolorbox}		 % use tcolorbox environment to box equations
\usepackage[USenglish]{babel}
\usepackage[useregional]{datetime2}
\usepackage[utf8]{inputenc}
\usepackage[T1]{fontenc}
\usepackage{enumerate}
\usepackage{caption}
\usepackage{subcaption}
\usepackage{physics}
\usepackage{hyperref}
\usepackage{chemformula}
\usepackage{booktabs}
\usepackage{makecell}
\usepackage{fullpage}
\usepackage{natbib}
\usepackage{setspace}
\usepackage{amsmath}
\usepackage{bm}              % enchanced bold math symbols
\usepackage{amssymb}
\usepackage{listings} 			 % For code citations
\usepackage{amsfonts}
\usepackage{mathtools}
\usepackage{commath}
\usepackage{fancyhdr}
\usepackage{lastpage}
\usepackage{tikz}
\usepackage{graphicx}
	\graphicspath{ {images/} }
\usepackage{feynmp-auto}
\usepackage[toc,page]{appendix}
%%%%%%%%%%%%%%%%%%%%%%%%%%%%%%%%%%%%%%%%%%%%%%%%%%%%%%%%%%%%%%%%%%%%%%%%%%%%%%%%
%					 	                      NEW COMMANDS							                   %
%%%%%%%%%%%%%%%%%%%%%%%%%%%%%%%%%%%%%%%%%%%%%%%%%%%%%%%%%%%%%%%%%%%%%%%%%%%%%%%%
% Redefine commands if necessary
\renewcommand\vec{\mathbf}                      % use boldface for vectors
% New commands
%------------------------------------Date---------------------------------------
\newcommand{\seminardate}{\DTMdisplaydate{2017}{7}{25}{-1}}
\newcommand{\revisiondate}{\DTMdisplaydate{2017}{7}{26}{-1}}
%-------------------------------------------------------------------------------
\newcommand*\Del{\vec{\nabla}}                  % del operator
\newcommand*\diff{\mathop{}\!\mathrm{d}}		    % differential d
\newcommand*\Diff[1]{\mathop{}\!\mathrm{d^#1}}	% d^(...)
\newcommand*{\scalarnorm}[1]                    % scalar norm |*|
  {\left\lvert#1\right\rvert}
\newcommand*{\vectornorm}[1]                    % vector norm ||*||
  {\left\lVert#1\right\rVert}
\newcommand*\onehalf{\frac{1}{2}}               % fraction 1/2
%------------------------------------Time---------------------------------------
\newcommand*\dlt[1]{\delta #1}
\newcommand*\timestep{\dlt{t}}              % timestep dt
\newcommand*\halfstep                       % half timestep 1/2 dt
  {\onehalf \timestep}
\newcommand*\timeInHalfStep{\left(t + \halfstep\right)}
\newcommand*\timeInFullStep{\left(t + \timestep\right)}
%-------------------------------------------------------------------------------
\input{rattle/definitions}
\input{ke_polymer/definitions}
%-------------------------------------------------------------------------------
%\setlength{\parindent}{0pt}						          % do not indent
\numberwithin{equation}{section}                % number eqs within sections
\begin{document}
\lstset{language=Python}
%%%%%%%%%%%%%%%%%%%%%%%%%%%%%%%%%%%%%%%%%%%%%%%%%%%%%%%%%%%%%%%%%%%%%%%%%%%%%%%%
%					 	                        TITLE							                         %
%%%%%%%%%%%%%%%%%%%%%%%%%%%%%%%%%%%%%%%%%%%%%%%%%%%%%%%%%%%%%%%%%%%%%%%%%%%%%%%%
\title{Group Seminar (revised: \revisiondate)}
\thanks{Professor Stratt, Vale Cofer-Shabica, Yan Zhao, Andrew Ton, Mansheej Paul, Evan Coleman}
\author{Artur Avkhadiev}
\email{artur\_avkhadiev@brown.edu}
\affiliation{Department of Physics, Brown University, Providence RI 02912, USA}
\date{\seminardate}
%%%%%%%%%%%%%%%%%%%%%%%%%%%%%%%%%%%%%%%%%%%%%%%%%%%%%%%%%%%%%%%%%%%%%%%%%%%%%%%%
%					 	                       ABSTRACT							                       %
%%%%%%%%%%%%%%%%%%%%%%%%%%%%%%%%%%%%%%%%%%%%%%%%%%%%%%%%%%%%%%%%%%%%%%%%%%%%%%%%
\begin{abstract}
  This report consist of two parts:
  \begin{enumerate}
      \item A discussion of the constraint dynamics algorithm based on velocity Verlet numerical integration scheme, \rattle.
      \item A calculation of the kinetic energy of a freely-jointed polymer chain using the bond-vector representation in the CM frame of the molecule.
    \end{enumerate}
\end{abstract}
\maketitle
\tableofcontents						 % hide table of contents
%%%%%%%%%%%%%%%%%%%%%%%%%%%%%%%%%%%%%%%%%%%%%%%%%%%%%%%%%%%%%%%%%%%%%%%%%%%%%%%%
%					 	                        BODY							                         %
%%%%%%%%%%%%%%%%%%%%%%%%%%%%%%%%%%%%%%%%%%%%%%%%%%%%%%%%%%%%%%%%%%%%%%%%%%%%%%%%
% Add include statements below. All the text should be in the folders.
\input{rattle/main}
\input{ke_polymer/main}
%-------------------------------------------------------------------------------
%%%%%%%%%%%%%%%%%%%%%%%%%%%%%%%%%%%%%%%%%%%%%%%%%%%%%%%%%%%%%%%%%%%%%%%%%%%%%%%%
%					                        BIBLIOGRAPHY							                   %
%%%%%%%%%%%%%%%%%%%%%%%%%%%%%%%%%%%%%%%%%%%%%%%%%%%%%%%%%%%%%%%%%%%%%%%%%%%%%%%%
% The bibliography file is reference,bib. If you need to add a
% reference, make sure all the entries look good (e.g., ensure
% proper capitalization by putting {} around the letters to be
% capitalized. Reference the books by chapters!
\nocite{*}
\bibliographystyle{apalike}
\bibliography{reference}
%-----------------------------
%%%%%%%%%%%%%%%%%%%%%%%%%%%%%%%%%%%%%%%%%%%%%%%%%%%%%%%%%%%%%%%%%%%%%%%%%%%%%%%%
%					 	                      APPENDICES							                     %
%%%%%%%%%%%%%%%%%%%%%%%%%%%%%%%%%%%%%%%%%%%%%%%%%%%%%%%%%%%%%%%%%%%%%%%%%%%%%%%%
% Add all appendices in appendices/main.tex with include statements.
% see appendices/sub-example-app.tex for an example.
\onecolumngrid
\input{appendices/main}
%-------------------------------------------------------------------------------
\end{document}

%-------------------------------------------------------------------------------
\end{document}

%-------------------------------------------------------------------------------
\end{document}

  \documentclass[%
    reprint,
    superscriptaddress,
    %groupedaddress,
    %unsortedaddress,
    %runinaddress,
    %frontmatterverbose,
    %preprint,
    longbibliography,
    bibnotes,
    %showpacs,preprintnumbers,
    %nofootinbib,
    %nobibnotes,
    %bibnotes,
     amsmath,amssymb,
     %aps,
     aip,
     jcp,                                       % J. Chem. Phys
    %pra,
    %prb,
    %rmp,
    %prstab,
    %prstper,
    %floatfix,
    ]{revtex4-1}
%%%%%%%%%%%%%%%%%%%%%%%%%%%%%%%%%%%%%%%%%%%%%%%%%%%%%%%%%%%%%%%%%%%%%%%%%%%%%%%%
%					 	                       LIBRARIES							                     %
%%%%%%%%%%%%%%%%%%%%%%%%%%%%%%%%%%%%%%%%%%%%%%%%%%%%%%%%%%%%%%%%%%%%%%%%%%%%%%%%
% Add whatever libraries you need here. Add a description if you want
\usepackage{graphicx}			 % Include figure files
\usepackage{dcolumn}			 % Align table columns on decimal point
\usepackage{tcolorbox}		 % use tcolorbox environment to box equations
\usepackage[USenglish]{babel}
\usepackage[useregional]{datetime2}
\usepackage[utf8]{inputenc}
\usepackage[T1]{fontenc}
\usepackage{enumerate}
\usepackage{caption}
\usepackage{subcaption}
\usepackage{physics}
\usepackage{hyperref}
\usepackage{chemformula}
\usepackage{booktabs}
\usepackage{makecell}
\usepackage{fullpage}
\usepackage{natbib}
\usepackage{setspace}
\usepackage{amsmath}
\usepackage{bm}              % enchanced bold math symbols
\usepackage{amssymb}
\usepackage{listings} 			 % For code citations
\usepackage{amsfonts}
\usepackage{mathtools}
\usepackage{commath}
\usepackage{fancyhdr}
\usepackage{lastpage}
\usepackage{tikz}
\usepackage{graphicx}
	\graphicspath{ {images/} }
\usepackage{feynmp-auto}
\usepackage[toc,page]{appendix}
%%%%%%%%%%%%%%%%%%%%%%%%%%%%%%%%%%%%%%%%%%%%%%%%%%%%%%%%%%%%%%%%%%%%%%%%%%%%%%%%
%					 	                      NEW COMMANDS							                   %
%%%%%%%%%%%%%%%%%%%%%%%%%%%%%%%%%%%%%%%%%%%%%%%%%%%%%%%%%%%%%%%%%%%%%%%%%%%%%%%%
% Redefine commands if necessary
\renewcommand\vec{\mathbf}                      % use boldface for vectors
% New commands
%------------------------------------Date---------------------------------------
\newcommand{\seminardate}{\DTMdisplaydate{2017}{7}{25}{-1}}
\newcommand{\revisiondate}{\DTMdisplaydate{2017}{7}{26}{-1}}
%-------------------------------------------------------------------------------
\newcommand*\Del{\vec{\nabla}}                  % del operator
\newcommand*\diff{\mathop{}\!\mathrm{d}}		    % differential d
\newcommand*\Diff[1]{\mathop{}\!\mathrm{d^#1}}	% d^(...)
\newcommand*{\scalarnorm}[1]                    % scalar norm |*|
  {\left\lvert#1\right\rvert}
\newcommand*{\vectornorm}[1]                    % vector norm ||*||
  {\left\lVert#1\right\rVert}
\newcommand*\onehalf{\frac{1}{2}}               % fraction 1/2
%------------------------------------Time---------------------------------------
\newcommand*\dlt[1]{\delta #1}
\newcommand*\timestep{\dlt{t}}              % timestep dt
\newcommand*\halfstep                       % half timestep 1/2 dt
  {\onehalf \timestep}
\newcommand*\timeInHalfStep{\left(t + \halfstep\right)}
\newcommand*\timeInFullStep{\left(t + \timestep\right)}
%-------------------------------------------------------------------------------
%%%%%%%%%%%%%%%%%%%%%%%%%%%%%%%%%%%%%%%%%%%%%%%%%%%%%%%%%%%%%%%%%%%%%%%%%%%%%%%%
%					 	                    DEFINITIONS								                     %
%%%%%%%%%%%%%%%%%%%%%%%%%%%%%%%%%%%%%%%%%%%%%%%%%%%%%%%%%%%%%%%%%%%%%%%%%%%%%%%%
%-------------------------------------------------------------------------------
\newcommand*\rattle{\textsf{RATTLE} }       % display name of algorithm
%-------------------------------------------------------------------------------
\newcommand*\rattlePos{\vec{r}}             % position vector
\newcommand*\rattleVel{\vec{v}}             % velocity vector
\newcommand*\rattleAcc{\vec{a}}             % acceleration vector
%-------------------------------------------------------------------------------
\newcommand*\rattleMassNoIndex{m}
\newcommand*\rattleForceNoIndex{\vec{F}}
\newcommand*\rattleConstraintForceNoIndex{\vec{G}}
\newcommand*\rattleConstraintNoIndex{\sigma}
\newcommand*\rattleLagrangeNoIndex{\lambda}
\newcommand*\rattleLagrangeApproxNoIndex{\gamma}
\newcommand*\rattleCorrectiveValueNoIndex{g}
%---------------------Definitions for General Formulation ----------------------
%%%%%%%%%%%%%%%%%%%%%%%%%%%%%%%%%%%%%%%%%%%%%%%%%%%%%%%%%%%%%%%%%%%%%%%%%%%%%%%%
%					 	                    DEFINITIONS								                     %
%%%%%%%%%%%%%%%%%%%%%%%%%%%%%%%%%%%%%%%%%%%%%%%%%%%%%%%%%%%%%%%%%%%%%%%%%%%%%%%%
%-------------------------------------------------------------------------------
\newcommand*\rattle{\textsf{RATTLE} }       % display name of algorithm
%-------------------------------------------------------------------------------
\newcommand*\rattlePos{\vec{r}}             % position vector
\newcommand*\rattleVel{\vec{v}}             % velocity vector
\newcommand*\rattleAcc{\vec{a}}             % acceleration vector
%-------------------------------------------------------------------------------
\newcommand*\rattleMassNoIndex{m}
\newcommand*\rattleForceNoIndex{\vec{F}}
\newcommand*\rattleConstraintForceNoIndex{\vec{G}}
\newcommand*\rattleConstraintNoIndex{\sigma}
\newcommand*\rattleLagrangeNoIndex{\lambda}
\newcommand*\rattleLagrangeApproxNoIndex{\gamma}
\newcommand*\rattleCorrectiveValueNoIndex{g}
%---------------------Definitions for General Formulation ----------------------
\input{rattle/general/definitions}
%-----------------------------------Indexing------------------------------------
\newcommand*\rattleAtomIndex{a}               % solo atom index
\newcommand*\rattleAtomIndexFirst{a}          % first index in atomic pair
\newcommand*\rattleAtomIndexSecond{b}         % second index in atomic pair
\newcommand*\rattleNAtoms{N}                  % number of atoms
\newcommand*\rattleConstraintSet{K}           % set of constraints
\newcommand*\rattleConstraintIndex            % index for constraints
  {\left(
    \rattleAtomIndexFirst,\,\rattleAtomIndexSecond
  \right) \in \rattleConstraintSet}
\newcommand*\rattleConstraintIndexSimple      % simplified index for constraints
  {\rattleAtomIndexFirst\rattleAtomIndexSecond}
\newcommand*\rattleCorrectionIndex{i}         % index for iterative correction
\newcommand*\rattleCorrectionIndexThis        % current for iterative correction
  {\rattleCorrectionIndex}
\newcommand*\rattleCorrectionIndexNext        % next for iterative correction
    {\rattleCorrectionIndex + 1}
\newcommand*\rattleCorrectionIndexLast{m}     % last step of correction
\newcommand*\rattleNCorrections{M}            % number of iterative corrections
%-----------------------------------Dynamics------------------------------------
\newcommand*\rattleMass                       % solo atomic mass
  {\rattleMassNoIndex_{\rattleAtomIndex}}
\newcommand*\rattleMassFirst                  % first atomic mass in pair
    {\rattleMassNoIndex_{\rattleAtomIndexFirst}}
\newcommand*\rattleMassSecond                 % second atomic mass in pair
    {\rattleMassNoIndex_{\rattleAtomIndexSecond}}
\newcommand*\rattleForce
  {\rattleForceNoIndex_{\rattleAtomIndex}}
\newcommand*\rattleConstraintForce            % solo constraint force
  {\rattleConstraintForceNoIndex_{\rattleAtomIndex}}
\newcommand*\rattleConstraintForceFirst       % first constraint force in pair
    {\rattleConstraintForceNoIndex_{\rattleAtomIndexFirst}}
\newcommand*\rattleConstraintForceSecond      % second constraint force in pair
    {\rattleConstraintForceNoIndex_{\rattleAtomIndexSecond}}
\newcommand*\rattleConstraintForceBond        % constraint force along bond
        {\rattleConstraintForceNoIndex_{\rattleConstraintIndexSimple}}
\newcommand*\rattleConstraint
  {\rattleConstraintNoIndex_{\rattleConstraintIndexSimple}}
  \newcommand*\rattleConstraintDot
    {\dot{\rattleConstraintNoIndex}_{\rattleConstraintIndexSimple}}
\newcommand*\rattleLagrange
    {\rattleLagrangeNoIndex_{\rattleConstraintIndexSimple}}
\newcommand*\rattleLagrangeApprox
        {\rattleLagrangeApproxNoIndex_{\rattleConstraintIndexSimple}}
\newcommand*\rattleCorrectiveValue
  {\rattleCorrectiveValueNoIndex_{\rattleConstraintIndexSimple}}
%-----------------------------IterativeCorrection-------------------------------
\newcommand*\iteration[2]{#1^{(#2)}}            % iterative correction steps
\newcommand*\iterationThis[1]                   % arg^{i}
  {\iteration{#1}{\rattleCorrectionIndexThis}}
\newcommand*\iterationNext[1]                   % arg^{i+1}
  {\iteration{#1}{\rattleCorrectionIndexNext}}
%----------------------------------Vectors--------------------------------------
\newcommand*\rattleAtomPos{\rattlePos_{\rattleAtomIndex}}
\newcommand*\rattleAtomPosFirst{\rattlePos_{\rattleAtomIndexFirst}}
\newcommand*\rattleAtomPosSecond{\rattlePos_{\rattleAtomIndexSecond}}
\newcommand*\rattleAtomVel{\rattleVel_{\rattleAtomIndex}}
\newcommand*\rattleAtomVelFirst{\rattleVel_{\rattleAtomIndexFirst}}
\newcommand*\rattleAtomVelSecond{\rattleVel_{\rattleAtomIndexSecond}}
\newcommand*\rattlePosUnconstrained             % r^{(0)}
  {\iteration{\rattlePos}{0}}
\newcommand*\rattleVelUnconstrained             % v^{(0)}
  {\iteration{\rattleVel}{0}}
\newcommand*\rattleAtomPosUnconstrained         % r^{(0)}
  {\iteration{\rattleAtomPos}{0}}
\newcommand*\rattleAtomVelUnconstrained         % v^{(0)}
  {\iteration{\rattleAtomVel}{0}}
\newcommand*\rattleBondPos{\rattlePos_{\rattleConstraintIndexSimple}}
\newcommand*\rattleBondPosUnit{\hat{\rattlePos}_{\rattleConstraintIndexSimple}}
\newcommand*\rattleBondVel{\rattleVel_{\rattleConstraintIndexSimple}}
\newcommand*\rattleBondVelUnit{\hat{\rattleVel}_{\rattleConstraintIndexSimple}}
\newcommand*\rattleBondPosUnconstrained
  {\iteration{\rattleBondPos}{0}}
\newcommand*\rattleBondVelUnconstrained
  {\iteration{\rattleBondVel}{0}}
\newcommand*\rattleAtomPosUnconstrainedFirst
    {\iteration{\rattleAtomPosFirst}{0}}
\newcommand*\rattleAtomPosUnconstrainedSecond
    {\iteration{\rattleAtomPosSecond}{0}}
\newcommand*\rattleAtomVelUnconstrainedFirst
    {\iteration{\rattleAtomVelFirst}{0}}
\newcommand*\rattleAtomVelUnconstrainedSecond
    {\iteration{\rattleAtomVelSecond}{0}}
%---------------------------------Constants-------------------------------------
\newcommand*\rattleBondlength                   % fixed bond length
  {d_{\rattleConstraintIndexSimple}}
\newcommand*\rattleTol{\xi}                     % d-r/d tolerance
\newcommand*\rattleRRTol{\varepsilon}           % r(t) * r(t+dt) tolerance
\newcommand*\rattleRVTol{\rattleTol}            % v(t)/d tolerance
%-------------------------------------------------------------------------------

%-----------------------------------Indexing------------------------------------
\newcommand*\rattleAtomIndex{a}               % solo atom index
\newcommand*\rattleAtomIndexFirst{a}          % first index in atomic pair
\newcommand*\rattleAtomIndexSecond{b}         % second index in atomic pair
\newcommand*\rattleNAtoms{N}                  % number of atoms
\newcommand*\rattleConstraintSet{K}           % set of constraints
\newcommand*\rattleConstraintIndex            % index for constraints
  {\left(
    \rattleAtomIndexFirst,\,\rattleAtomIndexSecond
  \right) \in \rattleConstraintSet}
\newcommand*\rattleConstraintIndexSimple      % simplified index for constraints
  {\rattleAtomIndexFirst\rattleAtomIndexSecond}
\newcommand*\rattleCorrectionIndex{i}         % index for iterative correction
\newcommand*\rattleCorrectionIndexThis        % current for iterative correction
  {\rattleCorrectionIndex}
\newcommand*\rattleCorrectionIndexNext        % next for iterative correction
    {\rattleCorrectionIndex + 1}
\newcommand*\rattleCorrectionIndexLast{m}     % last step of correction
\newcommand*\rattleNCorrections{M}            % number of iterative corrections
%-----------------------------------Dynamics------------------------------------
\newcommand*\rattleMass                       % solo atomic mass
  {\rattleMassNoIndex_{\rattleAtomIndex}}
\newcommand*\rattleMassFirst                  % first atomic mass in pair
    {\rattleMassNoIndex_{\rattleAtomIndexFirst}}
\newcommand*\rattleMassSecond                 % second atomic mass in pair
    {\rattleMassNoIndex_{\rattleAtomIndexSecond}}
\newcommand*\rattleForce
  {\rattleForceNoIndex_{\rattleAtomIndex}}
\newcommand*\rattleConstraintForce            % solo constraint force
  {\rattleConstraintForceNoIndex_{\rattleAtomIndex}}
\newcommand*\rattleConstraintForceFirst       % first constraint force in pair
    {\rattleConstraintForceNoIndex_{\rattleAtomIndexFirst}}
\newcommand*\rattleConstraintForceSecond      % second constraint force in pair
    {\rattleConstraintForceNoIndex_{\rattleAtomIndexSecond}}
\newcommand*\rattleConstraintForceBond        % constraint force along bond
        {\rattleConstraintForceNoIndex_{\rattleConstraintIndexSimple}}
\newcommand*\rattleConstraint
  {\rattleConstraintNoIndex_{\rattleConstraintIndexSimple}}
  \newcommand*\rattleConstraintDot
    {\dot{\rattleConstraintNoIndex}_{\rattleConstraintIndexSimple}}
\newcommand*\rattleLagrange
    {\rattleLagrangeNoIndex_{\rattleConstraintIndexSimple}}
\newcommand*\rattleLagrangeApprox
        {\rattleLagrangeApproxNoIndex_{\rattleConstraintIndexSimple}}
\newcommand*\rattleCorrectiveValue
  {\rattleCorrectiveValueNoIndex_{\rattleConstraintIndexSimple}}
%-----------------------------IterativeCorrection-------------------------------
\newcommand*\iteration[2]{#1^{(#2)}}            % iterative correction steps
\newcommand*\iterationThis[1]                   % arg^{i}
  {\iteration{#1}{\rattleCorrectionIndexThis}}
\newcommand*\iterationNext[1]                   % arg^{i+1}
  {\iteration{#1}{\rattleCorrectionIndexNext}}
%----------------------------------Vectors--------------------------------------
\newcommand*\rattleAtomPos{\rattlePos_{\rattleAtomIndex}}
\newcommand*\rattleAtomPosFirst{\rattlePos_{\rattleAtomIndexFirst}}
\newcommand*\rattleAtomPosSecond{\rattlePos_{\rattleAtomIndexSecond}}
\newcommand*\rattleAtomVel{\rattleVel_{\rattleAtomIndex}}
\newcommand*\rattleAtomVelFirst{\rattleVel_{\rattleAtomIndexFirst}}
\newcommand*\rattleAtomVelSecond{\rattleVel_{\rattleAtomIndexSecond}}
\newcommand*\rattlePosUnconstrained             % r^{(0)}
  {\iteration{\rattlePos}{0}}
\newcommand*\rattleVelUnconstrained             % v^{(0)}
  {\iteration{\rattleVel}{0}}
\newcommand*\rattleAtomPosUnconstrained         % r^{(0)}
  {\iteration{\rattleAtomPos}{0}}
\newcommand*\rattleAtomVelUnconstrained         % v^{(0)}
  {\iteration{\rattleAtomVel}{0}}
\newcommand*\rattleBondPos{\rattlePos_{\rattleConstraintIndexSimple}}
\newcommand*\rattleBondPosUnit{\hat{\rattlePos}_{\rattleConstraintIndexSimple}}
\newcommand*\rattleBondVel{\rattleVel_{\rattleConstraintIndexSimple}}
\newcommand*\rattleBondVelUnit{\hat{\rattleVel}_{\rattleConstraintIndexSimple}}
\newcommand*\rattleBondPosUnconstrained
  {\iteration{\rattleBondPos}{0}}
\newcommand*\rattleBondVelUnconstrained
  {\iteration{\rattleBondVel}{0}}
\newcommand*\rattleAtomPosUnconstrainedFirst
    {\iteration{\rattleAtomPosFirst}{0}}
\newcommand*\rattleAtomPosUnconstrainedSecond
    {\iteration{\rattleAtomPosSecond}{0}}
\newcommand*\rattleAtomVelUnconstrainedFirst
    {\iteration{\rattleAtomVelFirst}{0}}
\newcommand*\rattleAtomVelUnconstrainedSecond
    {\iteration{\rattleAtomVelSecond}{0}}
%---------------------------------Constants-------------------------------------
\newcommand*\rattleBondlength                   % fixed bond length
  {d_{\rattleConstraintIndexSimple}}
\newcommand*\rattleTol{\xi}                     % d-r/d tolerance
\newcommand*\rattleRRTol{\varepsilon}           % r(t) * r(t+dt) tolerance
\newcommand*\rattleRVTol{\rattleTol}            % v(t)/d tolerance
%-------------------------------------------------------------------------------

%%%%%%%%%%%%%%%%%%%%%%%%%%%%%%%%%%%%%%%%%%%%%%%%%%%%%%%%%%%%%%%%%%%%%%%%%%%%%%%%
%					 	                    DEFINITIONS								                     %
%%%%%%%%%%%%%%%%%%%%%%%%%%%%%%%%%%%%%%%%%%%%%%%%%%%%%%%%%%%%%%%%%%%%%%%%%%%%%%%%
%-------------------------------------------------------------------------------
\newcommand*\rattle{\textsf{RATTLE} }       % display name of algorithm
%-------------------------------------------------------------------------------
\newcommand*\rattlePos{\vec{r}}             % position vector
\newcommand*\rattleVel{\vec{v}}             % velocity vector
\newcommand*\rattleAcc{\vec{a}}             % acceleration vector
%-------------------------------------------------------------------------------
\newcommand*\rattleMassNoIndex{m}
\newcommand*\rattleForceNoIndex{\vec{F}}
\newcommand*\rattleConstraintForceNoIndex{\vec{G}}
\newcommand*\rattleConstraintNoIndex{\sigma}
\newcommand*\rattleLagrangeNoIndex{\lambda}
\newcommand*\rattleLagrangeApproxNoIndex{\gamma}
\newcommand*\rattleCorrectiveValueNoIndex{g}
%---------------------Definitions for General Formulation ----------------------
%%%%%%%%%%%%%%%%%%%%%%%%%%%%%%%%%%%%%%%%%%%%%%%%%%%%%%%%%%%%%%%%%%%%%%%%%%%%%%%%
%					 	                    DEFINITIONS								                     %
%%%%%%%%%%%%%%%%%%%%%%%%%%%%%%%%%%%%%%%%%%%%%%%%%%%%%%%%%%%%%%%%%%%%%%%%%%%%%%%%
%-------------------------------------------------------------------------------
\newcommand*\rattle{\textsf{RATTLE} }       % display name of algorithm
%-------------------------------------------------------------------------------
\newcommand*\rattlePos{\vec{r}}             % position vector
\newcommand*\rattleVel{\vec{v}}             % velocity vector
\newcommand*\rattleAcc{\vec{a}}             % acceleration vector
%-------------------------------------------------------------------------------
\newcommand*\rattleMassNoIndex{m}
\newcommand*\rattleForceNoIndex{\vec{F}}
\newcommand*\rattleConstraintForceNoIndex{\vec{G}}
\newcommand*\rattleConstraintNoIndex{\sigma}
\newcommand*\rattleLagrangeNoIndex{\lambda}
\newcommand*\rattleLagrangeApproxNoIndex{\gamma}
\newcommand*\rattleCorrectiveValueNoIndex{g}
%---------------------Definitions for General Formulation ----------------------
\input{rattle/general/definitions}
%-----------------------------------Indexing------------------------------------
\newcommand*\rattleAtomIndex{a}               % solo atom index
\newcommand*\rattleAtomIndexFirst{a}          % first index in atomic pair
\newcommand*\rattleAtomIndexSecond{b}         % second index in atomic pair
\newcommand*\rattleNAtoms{N}                  % number of atoms
\newcommand*\rattleConstraintSet{K}           % set of constraints
\newcommand*\rattleConstraintIndex            % index for constraints
  {\left(
    \rattleAtomIndexFirst,\,\rattleAtomIndexSecond
  \right) \in \rattleConstraintSet}
\newcommand*\rattleConstraintIndexSimple      % simplified index for constraints
  {\rattleAtomIndexFirst\rattleAtomIndexSecond}
\newcommand*\rattleCorrectionIndex{i}         % index for iterative correction
\newcommand*\rattleCorrectionIndexThis        % current for iterative correction
  {\rattleCorrectionIndex}
\newcommand*\rattleCorrectionIndexNext        % next for iterative correction
    {\rattleCorrectionIndex + 1}
\newcommand*\rattleCorrectionIndexLast{m}     % last step of correction
\newcommand*\rattleNCorrections{M}            % number of iterative corrections
%-----------------------------------Dynamics------------------------------------
\newcommand*\rattleMass                       % solo atomic mass
  {\rattleMassNoIndex_{\rattleAtomIndex}}
\newcommand*\rattleMassFirst                  % first atomic mass in pair
    {\rattleMassNoIndex_{\rattleAtomIndexFirst}}
\newcommand*\rattleMassSecond                 % second atomic mass in pair
    {\rattleMassNoIndex_{\rattleAtomIndexSecond}}
\newcommand*\rattleForce
  {\rattleForceNoIndex_{\rattleAtomIndex}}
\newcommand*\rattleConstraintForce            % solo constraint force
  {\rattleConstraintForceNoIndex_{\rattleAtomIndex}}
\newcommand*\rattleConstraintForceFirst       % first constraint force in pair
    {\rattleConstraintForceNoIndex_{\rattleAtomIndexFirst}}
\newcommand*\rattleConstraintForceSecond      % second constraint force in pair
    {\rattleConstraintForceNoIndex_{\rattleAtomIndexSecond}}
\newcommand*\rattleConstraintForceBond        % constraint force along bond
        {\rattleConstraintForceNoIndex_{\rattleConstraintIndexSimple}}
\newcommand*\rattleConstraint
  {\rattleConstraintNoIndex_{\rattleConstraintIndexSimple}}
  \newcommand*\rattleConstraintDot
    {\dot{\rattleConstraintNoIndex}_{\rattleConstraintIndexSimple}}
\newcommand*\rattleLagrange
    {\rattleLagrangeNoIndex_{\rattleConstraintIndexSimple}}
\newcommand*\rattleLagrangeApprox
        {\rattleLagrangeApproxNoIndex_{\rattleConstraintIndexSimple}}
\newcommand*\rattleCorrectiveValue
  {\rattleCorrectiveValueNoIndex_{\rattleConstraintIndexSimple}}
%-----------------------------IterativeCorrection-------------------------------
\newcommand*\iteration[2]{#1^{(#2)}}            % iterative correction steps
\newcommand*\iterationThis[1]                   % arg^{i}
  {\iteration{#1}{\rattleCorrectionIndexThis}}
\newcommand*\iterationNext[1]                   % arg^{i+1}
  {\iteration{#1}{\rattleCorrectionIndexNext}}
%----------------------------------Vectors--------------------------------------
\newcommand*\rattleAtomPos{\rattlePos_{\rattleAtomIndex}}
\newcommand*\rattleAtomPosFirst{\rattlePos_{\rattleAtomIndexFirst}}
\newcommand*\rattleAtomPosSecond{\rattlePos_{\rattleAtomIndexSecond}}
\newcommand*\rattleAtomVel{\rattleVel_{\rattleAtomIndex}}
\newcommand*\rattleAtomVelFirst{\rattleVel_{\rattleAtomIndexFirst}}
\newcommand*\rattleAtomVelSecond{\rattleVel_{\rattleAtomIndexSecond}}
\newcommand*\rattlePosUnconstrained             % r^{(0)}
  {\iteration{\rattlePos}{0}}
\newcommand*\rattleVelUnconstrained             % v^{(0)}
  {\iteration{\rattleVel}{0}}
\newcommand*\rattleAtomPosUnconstrained         % r^{(0)}
  {\iteration{\rattleAtomPos}{0}}
\newcommand*\rattleAtomVelUnconstrained         % v^{(0)}
  {\iteration{\rattleAtomVel}{0}}
\newcommand*\rattleBondPos{\rattlePos_{\rattleConstraintIndexSimple}}
\newcommand*\rattleBondPosUnit{\hat{\rattlePos}_{\rattleConstraintIndexSimple}}
\newcommand*\rattleBondVel{\rattleVel_{\rattleConstraintIndexSimple}}
\newcommand*\rattleBondVelUnit{\hat{\rattleVel}_{\rattleConstraintIndexSimple}}
\newcommand*\rattleBondPosUnconstrained
  {\iteration{\rattleBondPos}{0}}
\newcommand*\rattleBondVelUnconstrained
  {\iteration{\rattleBondVel}{0}}
\newcommand*\rattleAtomPosUnconstrainedFirst
    {\iteration{\rattleAtomPosFirst}{0}}
\newcommand*\rattleAtomPosUnconstrainedSecond
    {\iteration{\rattleAtomPosSecond}{0}}
\newcommand*\rattleAtomVelUnconstrainedFirst
    {\iteration{\rattleAtomVelFirst}{0}}
\newcommand*\rattleAtomVelUnconstrainedSecond
    {\iteration{\rattleAtomVelSecond}{0}}
%---------------------------------Constants-------------------------------------
\newcommand*\rattleBondlength                   % fixed bond length
  {d_{\rattleConstraintIndexSimple}}
\newcommand*\rattleTol{\xi}                     % d-r/d tolerance
\newcommand*\rattleRRTol{\varepsilon}           % r(t) * r(t+dt) tolerance
\newcommand*\rattleRVTol{\rattleTol}            % v(t)/d tolerance
%-------------------------------------------------------------------------------

%-----------------------------------Indexing------------------------------------
\newcommand*\rattleAtomIndex{a}               % solo atom index
\newcommand*\rattleAtomIndexFirst{a}          % first index in atomic pair
\newcommand*\rattleAtomIndexSecond{b}         % second index in atomic pair
\newcommand*\rattleNAtoms{N}                  % number of atoms
\newcommand*\rattleConstraintSet{K}           % set of constraints
\newcommand*\rattleConstraintIndex            % index for constraints
  {\left(
    \rattleAtomIndexFirst,\,\rattleAtomIndexSecond
  \right) \in \rattleConstraintSet}
\newcommand*\rattleConstraintIndexSimple      % simplified index for constraints
  {\rattleAtomIndexFirst\rattleAtomIndexSecond}
\newcommand*\rattleCorrectionIndex{i}         % index for iterative correction
\newcommand*\rattleCorrectionIndexThis        % current for iterative correction
  {\rattleCorrectionIndex}
\newcommand*\rattleCorrectionIndexNext        % next for iterative correction
    {\rattleCorrectionIndex + 1}
\newcommand*\rattleCorrectionIndexLast{m}     % last step of correction
\newcommand*\rattleNCorrections{M}            % number of iterative corrections
%-----------------------------------Dynamics------------------------------------
\newcommand*\rattleMass                       % solo atomic mass
  {\rattleMassNoIndex_{\rattleAtomIndex}}
\newcommand*\rattleMassFirst                  % first atomic mass in pair
    {\rattleMassNoIndex_{\rattleAtomIndexFirst}}
\newcommand*\rattleMassSecond                 % second atomic mass in pair
    {\rattleMassNoIndex_{\rattleAtomIndexSecond}}
\newcommand*\rattleForce
  {\rattleForceNoIndex_{\rattleAtomIndex}}
\newcommand*\rattleConstraintForce            % solo constraint force
  {\rattleConstraintForceNoIndex_{\rattleAtomIndex}}
\newcommand*\rattleConstraintForceFirst       % first constraint force in pair
    {\rattleConstraintForceNoIndex_{\rattleAtomIndexFirst}}
\newcommand*\rattleConstraintForceSecond      % second constraint force in pair
    {\rattleConstraintForceNoIndex_{\rattleAtomIndexSecond}}
\newcommand*\rattleConstraintForceBond        % constraint force along bond
        {\rattleConstraintForceNoIndex_{\rattleConstraintIndexSimple}}
\newcommand*\rattleConstraint
  {\rattleConstraintNoIndex_{\rattleConstraintIndexSimple}}
  \newcommand*\rattleConstraintDot
    {\dot{\rattleConstraintNoIndex}_{\rattleConstraintIndexSimple}}
\newcommand*\rattleLagrange
    {\rattleLagrangeNoIndex_{\rattleConstraintIndexSimple}}
\newcommand*\rattleLagrangeApprox
        {\rattleLagrangeApproxNoIndex_{\rattleConstraintIndexSimple}}
\newcommand*\rattleCorrectiveValue
  {\rattleCorrectiveValueNoIndex_{\rattleConstraintIndexSimple}}
%-----------------------------IterativeCorrection-------------------------------
\newcommand*\iteration[2]{#1^{(#2)}}            % iterative correction steps
\newcommand*\iterationThis[1]                   % arg^{i}
  {\iteration{#1}{\rattleCorrectionIndexThis}}
\newcommand*\iterationNext[1]                   % arg^{i+1}
  {\iteration{#1}{\rattleCorrectionIndexNext}}
%----------------------------------Vectors--------------------------------------
\newcommand*\rattleAtomPos{\rattlePos_{\rattleAtomIndex}}
\newcommand*\rattleAtomPosFirst{\rattlePos_{\rattleAtomIndexFirst}}
\newcommand*\rattleAtomPosSecond{\rattlePos_{\rattleAtomIndexSecond}}
\newcommand*\rattleAtomVel{\rattleVel_{\rattleAtomIndex}}
\newcommand*\rattleAtomVelFirst{\rattleVel_{\rattleAtomIndexFirst}}
\newcommand*\rattleAtomVelSecond{\rattleVel_{\rattleAtomIndexSecond}}
\newcommand*\rattlePosUnconstrained             % r^{(0)}
  {\iteration{\rattlePos}{0}}
\newcommand*\rattleVelUnconstrained             % v^{(0)}
  {\iteration{\rattleVel}{0}}
\newcommand*\rattleAtomPosUnconstrained         % r^{(0)}
  {\iteration{\rattleAtomPos}{0}}
\newcommand*\rattleAtomVelUnconstrained         % v^{(0)}
  {\iteration{\rattleAtomVel}{0}}
\newcommand*\rattleBondPos{\rattlePos_{\rattleConstraintIndexSimple}}
\newcommand*\rattleBondPosUnit{\hat{\rattlePos}_{\rattleConstraintIndexSimple}}
\newcommand*\rattleBondVel{\rattleVel_{\rattleConstraintIndexSimple}}
\newcommand*\rattleBondVelUnit{\hat{\rattleVel}_{\rattleConstraintIndexSimple}}
\newcommand*\rattleBondPosUnconstrained
  {\iteration{\rattleBondPos}{0}}
\newcommand*\rattleBondVelUnconstrained
  {\iteration{\rattleBondVel}{0}}
\newcommand*\rattleAtomPosUnconstrainedFirst
    {\iteration{\rattleAtomPosFirst}{0}}
\newcommand*\rattleAtomPosUnconstrainedSecond
    {\iteration{\rattleAtomPosSecond}{0}}
\newcommand*\rattleAtomVelUnconstrainedFirst
    {\iteration{\rattleAtomVelFirst}{0}}
\newcommand*\rattleAtomVelUnconstrainedSecond
    {\iteration{\rattleAtomVelSecond}{0}}
%---------------------------------Constants-------------------------------------
\newcommand*\rattleBondlength                   % fixed bond length
  {d_{\rattleConstraintIndexSimple}}
\newcommand*\rattleTol{\xi}                     % d-r/d tolerance
\newcommand*\rattleRRTol{\varepsilon}           % r(t) * r(t+dt) tolerance
\newcommand*\rattleRVTol{\rattleTol}            % v(t)/d tolerance
%-------------------------------------------------------------------------------

%-------------------------------------------------------------------------------
%\setlength{\parindent}{0pt}						          % do not indent
\numberwithin{equation}{section}                % number eqs within sections
\begin{document}
\lstset{language=Python}
%%%%%%%%%%%%%%%%%%%%%%%%%%%%%%%%%%%%%%%%%%%%%%%%%%%%%%%%%%%%%%%%%%%%%%%%%%%%%%%%
%					 	                        TITLE							                         %
%%%%%%%%%%%%%%%%%%%%%%%%%%%%%%%%%%%%%%%%%%%%%%%%%%%%%%%%%%%%%%%%%%%%%%%%%%%%%%%%
\title{Group Seminar (revised: \revisiondate)}
\thanks{Professor Stratt, Vale Cofer-Shabica, Yan Zhao, Andrew Ton, Mansheej Paul, Evan Coleman}
\author{Artur Avkhadiev}
\email{artur\_avkhadiev@brown.edu}
\affiliation{Department of Physics, Brown University, Providence RI 02912, USA}
\date{\seminardate}
%%%%%%%%%%%%%%%%%%%%%%%%%%%%%%%%%%%%%%%%%%%%%%%%%%%%%%%%%%%%%%%%%%%%%%%%%%%%%%%%
%					 	                       ABSTRACT							                       %
%%%%%%%%%%%%%%%%%%%%%%%%%%%%%%%%%%%%%%%%%%%%%%%%%%%%%%%%%%%%%%%%%%%%%%%%%%%%%%%%
\begin{abstract}
  This report consist of two parts:
  \begin{enumerate}
      \item A discussion of the constraint dynamics algorithm based on velocity Verlet numerical integration scheme, \rattle.
      \item A calculation of the kinetic energy of a freely-jointed polymer chain using the bond-vector representation in the CM frame of the molecule.
    \end{enumerate}
\end{abstract}
\maketitle
\tableofcontents						 % hide table of contents
%%%%%%%%%%%%%%%%%%%%%%%%%%%%%%%%%%%%%%%%%%%%%%%%%%%%%%%%%%%%%%%%%%%%%%%%%%%%%%%%
%					 	                        BODY							                         %
%%%%%%%%%%%%%%%%%%%%%%%%%%%%%%%%%%%%%%%%%%%%%%%%%%%%%%%%%%%%%%%%%%%%%%%%%%%%%%%%
% Add include statements below. All the text should be in the folders.
\documentclass[%
    reprint,
    superscriptaddress,
    %groupedaddress,
    %unsortedaddress,
    %runinaddress,
    %frontmatterverbose,
    %preprint,
    longbibliography,
    bibnotes,
    %showpacs,preprintnumbers,
    %nofootinbib,
    %nobibnotes,
    %bibnotes,
     amsmath,amssymb,
     %aps,
     aip,
     jcp,                                       % J. Chem. Phys
    %pra,
    %prb,
    %rmp,
    %prstab,
    %prstper,
    %floatfix,
    ]{revtex4-1}
%%%%%%%%%%%%%%%%%%%%%%%%%%%%%%%%%%%%%%%%%%%%%%%%%%%%%%%%%%%%%%%%%%%%%%%%%%%%%%%%
%					 	                       LIBRARIES							                     %
%%%%%%%%%%%%%%%%%%%%%%%%%%%%%%%%%%%%%%%%%%%%%%%%%%%%%%%%%%%%%%%%%%%%%%%%%%%%%%%%
% Add whatever libraries you need here. Add a description if you want
\usepackage{graphicx}			 % Include figure files
\usepackage{dcolumn}			 % Align table columns on decimal point
\usepackage{tcolorbox}		 % use tcolorbox environment to box equations
\usepackage[USenglish]{babel}
\usepackage[useregional]{datetime2}
\usepackage[utf8]{inputenc}
\usepackage[T1]{fontenc}
\usepackage{enumerate}
\usepackage{caption}
\usepackage{subcaption}
\usepackage{physics}
\usepackage{hyperref}
\usepackage{chemformula}
\usepackage{booktabs}
\usepackage{makecell}
\usepackage{fullpage}
\usepackage{natbib}
\usepackage{setspace}
\usepackage{amsmath}
\usepackage{bm}              % enchanced bold math symbols
\usepackage{amssymb}
\usepackage{listings} 			 % For code citations
\usepackage{amsfonts}
\usepackage{mathtools}
\usepackage{commath}
\usepackage{fancyhdr}
\usepackage{lastpage}
\usepackage{tikz}
\usepackage{graphicx}
	\graphicspath{ {images/} }
\usepackage{feynmp-auto}
\usepackage[toc,page]{appendix}
%%%%%%%%%%%%%%%%%%%%%%%%%%%%%%%%%%%%%%%%%%%%%%%%%%%%%%%%%%%%%%%%%%%%%%%%%%%%%%%%
%					 	                      NEW COMMANDS							                   %
%%%%%%%%%%%%%%%%%%%%%%%%%%%%%%%%%%%%%%%%%%%%%%%%%%%%%%%%%%%%%%%%%%%%%%%%%%%%%%%%
% Redefine commands if necessary
\renewcommand\vec{\mathbf}                      % use boldface for vectors
% New commands
%------------------------------------Date---------------------------------------
\newcommand{\seminardate}{\DTMdisplaydate{2017}{7}{25}{-1}}
\newcommand{\revisiondate}{\DTMdisplaydate{2017}{7}{26}{-1}}
%-------------------------------------------------------------------------------
\newcommand*\Del{\vec{\nabla}}                  % del operator
\newcommand*\diff{\mathop{}\!\mathrm{d}}		    % differential d
\newcommand*\Diff[1]{\mathop{}\!\mathrm{d^#1}}	% d^(...)
\newcommand*{\scalarnorm}[1]                    % scalar norm |*|
  {\left\lvert#1\right\rvert}
\newcommand*{\vectornorm}[1]                    % vector norm ||*||
  {\left\lVert#1\right\rVert}
\newcommand*\onehalf{\frac{1}{2}}               % fraction 1/2
%------------------------------------Time---------------------------------------
\newcommand*\dlt[1]{\delta #1}
\newcommand*\timestep{\dlt{t}}              % timestep dt
\newcommand*\halfstep                       % half timestep 1/2 dt
  {\onehalf \timestep}
\newcommand*\timeInHalfStep{\left(t + \halfstep\right)}
\newcommand*\timeInFullStep{\left(t + \timestep\right)}
%-------------------------------------------------------------------------------
%%%%%%%%%%%%%%%%%%%%%%%%%%%%%%%%%%%%%%%%%%%%%%%%%%%%%%%%%%%%%%%%%%%%%%%%%%%%%%%%
%					 	                    DEFINITIONS								                     %
%%%%%%%%%%%%%%%%%%%%%%%%%%%%%%%%%%%%%%%%%%%%%%%%%%%%%%%%%%%%%%%%%%%%%%%%%%%%%%%%
%-------------------------------------------------------------------------------
\newcommand*\rattle{\textsf{RATTLE} }       % display name of algorithm
%-------------------------------------------------------------------------------
\newcommand*\rattlePos{\vec{r}}             % position vector
\newcommand*\rattleVel{\vec{v}}             % velocity vector
\newcommand*\rattleAcc{\vec{a}}             % acceleration vector
%-------------------------------------------------------------------------------
\newcommand*\rattleMassNoIndex{m}
\newcommand*\rattleForceNoIndex{\vec{F}}
\newcommand*\rattleConstraintForceNoIndex{\vec{G}}
\newcommand*\rattleConstraintNoIndex{\sigma}
\newcommand*\rattleLagrangeNoIndex{\lambda}
\newcommand*\rattleLagrangeApproxNoIndex{\gamma}
\newcommand*\rattleCorrectiveValueNoIndex{g}
%---------------------Definitions for General Formulation ----------------------
\input{rattle/general/definitions}
%-----------------------------------Indexing------------------------------------
\newcommand*\rattleAtomIndex{a}               % solo atom index
\newcommand*\rattleAtomIndexFirst{a}          % first index in atomic pair
\newcommand*\rattleAtomIndexSecond{b}         % second index in atomic pair
\newcommand*\rattleNAtoms{N}                  % number of atoms
\newcommand*\rattleConstraintSet{K}           % set of constraints
\newcommand*\rattleConstraintIndex            % index for constraints
  {\left(
    \rattleAtomIndexFirst,\,\rattleAtomIndexSecond
  \right) \in \rattleConstraintSet}
\newcommand*\rattleConstraintIndexSimple      % simplified index for constraints
  {\rattleAtomIndexFirst\rattleAtomIndexSecond}
\newcommand*\rattleCorrectionIndex{i}         % index for iterative correction
\newcommand*\rattleCorrectionIndexThis        % current for iterative correction
  {\rattleCorrectionIndex}
\newcommand*\rattleCorrectionIndexNext        % next for iterative correction
    {\rattleCorrectionIndex + 1}
\newcommand*\rattleCorrectionIndexLast{m}     % last step of correction
\newcommand*\rattleNCorrections{M}            % number of iterative corrections
%-----------------------------------Dynamics------------------------------------
\newcommand*\rattleMass                       % solo atomic mass
  {\rattleMassNoIndex_{\rattleAtomIndex}}
\newcommand*\rattleMassFirst                  % first atomic mass in pair
    {\rattleMassNoIndex_{\rattleAtomIndexFirst}}
\newcommand*\rattleMassSecond                 % second atomic mass in pair
    {\rattleMassNoIndex_{\rattleAtomIndexSecond}}
\newcommand*\rattleForce
  {\rattleForceNoIndex_{\rattleAtomIndex}}
\newcommand*\rattleConstraintForce            % solo constraint force
  {\rattleConstraintForceNoIndex_{\rattleAtomIndex}}
\newcommand*\rattleConstraintForceFirst       % first constraint force in pair
    {\rattleConstraintForceNoIndex_{\rattleAtomIndexFirst}}
\newcommand*\rattleConstraintForceSecond      % second constraint force in pair
    {\rattleConstraintForceNoIndex_{\rattleAtomIndexSecond}}
\newcommand*\rattleConstraintForceBond        % constraint force along bond
        {\rattleConstraintForceNoIndex_{\rattleConstraintIndexSimple}}
\newcommand*\rattleConstraint
  {\rattleConstraintNoIndex_{\rattleConstraintIndexSimple}}
  \newcommand*\rattleConstraintDot
    {\dot{\rattleConstraintNoIndex}_{\rattleConstraintIndexSimple}}
\newcommand*\rattleLagrange
    {\rattleLagrangeNoIndex_{\rattleConstraintIndexSimple}}
\newcommand*\rattleLagrangeApprox
        {\rattleLagrangeApproxNoIndex_{\rattleConstraintIndexSimple}}
\newcommand*\rattleCorrectiveValue
  {\rattleCorrectiveValueNoIndex_{\rattleConstraintIndexSimple}}
%-----------------------------IterativeCorrection-------------------------------
\newcommand*\iteration[2]{#1^{(#2)}}            % iterative correction steps
\newcommand*\iterationThis[1]                   % arg^{i}
  {\iteration{#1}{\rattleCorrectionIndexThis}}
\newcommand*\iterationNext[1]                   % arg^{i+1}
  {\iteration{#1}{\rattleCorrectionIndexNext}}
%----------------------------------Vectors--------------------------------------
\newcommand*\rattleAtomPos{\rattlePos_{\rattleAtomIndex}}
\newcommand*\rattleAtomPosFirst{\rattlePos_{\rattleAtomIndexFirst}}
\newcommand*\rattleAtomPosSecond{\rattlePos_{\rattleAtomIndexSecond}}
\newcommand*\rattleAtomVel{\rattleVel_{\rattleAtomIndex}}
\newcommand*\rattleAtomVelFirst{\rattleVel_{\rattleAtomIndexFirst}}
\newcommand*\rattleAtomVelSecond{\rattleVel_{\rattleAtomIndexSecond}}
\newcommand*\rattlePosUnconstrained             % r^{(0)}
  {\iteration{\rattlePos}{0}}
\newcommand*\rattleVelUnconstrained             % v^{(0)}
  {\iteration{\rattleVel}{0}}
\newcommand*\rattleAtomPosUnconstrained         % r^{(0)}
  {\iteration{\rattleAtomPos}{0}}
\newcommand*\rattleAtomVelUnconstrained         % v^{(0)}
  {\iteration{\rattleAtomVel}{0}}
\newcommand*\rattleBondPos{\rattlePos_{\rattleConstraintIndexSimple}}
\newcommand*\rattleBondPosUnit{\hat{\rattlePos}_{\rattleConstraintIndexSimple}}
\newcommand*\rattleBondVel{\rattleVel_{\rattleConstraintIndexSimple}}
\newcommand*\rattleBondVelUnit{\hat{\rattleVel}_{\rattleConstraintIndexSimple}}
\newcommand*\rattleBondPosUnconstrained
  {\iteration{\rattleBondPos}{0}}
\newcommand*\rattleBondVelUnconstrained
  {\iteration{\rattleBondVel}{0}}
\newcommand*\rattleAtomPosUnconstrainedFirst
    {\iteration{\rattleAtomPosFirst}{0}}
\newcommand*\rattleAtomPosUnconstrainedSecond
    {\iteration{\rattleAtomPosSecond}{0}}
\newcommand*\rattleAtomVelUnconstrainedFirst
    {\iteration{\rattleAtomVelFirst}{0}}
\newcommand*\rattleAtomVelUnconstrainedSecond
    {\iteration{\rattleAtomVelSecond}{0}}
%---------------------------------Constants-------------------------------------
\newcommand*\rattleBondlength                   % fixed bond length
  {d_{\rattleConstraintIndexSimple}}
\newcommand*\rattleTol{\xi}                     % d-r/d tolerance
\newcommand*\rattleRRTol{\varepsilon}           % r(t) * r(t+dt) tolerance
\newcommand*\rattleRVTol{\rattleTol}            % v(t)/d tolerance
%-------------------------------------------------------------------------------

%%%%%%%%%%%%%%%%%%%%%%%%%%%%%%%%%%%%%%%%%%%%%%%%%%%%%%%%%%%%%%%%%%%%%%%%%%%%%%%%
%					 	                    DEFINITIONS								                     %
%%%%%%%%%%%%%%%%%%%%%%%%%%%%%%%%%%%%%%%%%%%%%%%%%%%%%%%%%%%%%%%%%%%%%%%%%%%%%%%%
%-------------------------------------------------------------------------------
\newcommand*\rattle{\textsf{RATTLE} }       % display name of algorithm
%-------------------------------------------------------------------------------
\newcommand*\rattlePos{\vec{r}}             % position vector
\newcommand*\rattleVel{\vec{v}}             % velocity vector
\newcommand*\rattleAcc{\vec{a}}             % acceleration vector
%-------------------------------------------------------------------------------
\newcommand*\rattleMassNoIndex{m}
\newcommand*\rattleForceNoIndex{\vec{F}}
\newcommand*\rattleConstraintForceNoIndex{\vec{G}}
\newcommand*\rattleConstraintNoIndex{\sigma}
\newcommand*\rattleLagrangeNoIndex{\lambda}
\newcommand*\rattleLagrangeApproxNoIndex{\gamma}
\newcommand*\rattleCorrectiveValueNoIndex{g}
%---------------------Definitions for General Formulation ----------------------
\input{rattle/general/definitions}
%-----------------------------------Indexing------------------------------------
\newcommand*\rattleAtomIndex{a}               % solo atom index
\newcommand*\rattleAtomIndexFirst{a}          % first index in atomic pair
\newcommand*\rattleAtomIndexSecond{b}         % second index in atomic pair
\newcommand*\rattleNAtoms{N}                  % number of atoms
\newcommand*\rattleConstraintSet{K}           % set of constraints
\newcommand*\rattleConstraintIndex            % index for constraints
  {\left(
    \rattleAtomIndexFirst,\,\rattleAtomIndexSecond
  \right) \in \rattleConstraintSet}
\newcommand*\rattleConstraintIndexSimple      % simplified index for constraints
  {\rattleAtomIndexFirst\rattleAtomIndexSecond}
\newcommand*\rattleCorrectionIndex{i}         % index for iterative correction
\newcommand*\rattleCorrectionIndexThis        % current for iterative correction
  {\rattleCorrectionIndex}
\newcommand*\rattleCorrectionIndexNext        % next for iterative correction
    {\rattleCorrectionIndex + 1}
\newcommand*\rattleCorrectionIndexLast{m}     % last step of correction
\newcommand*\rattleNCorrections{M}            % number of iterative corrections
%-----------------------------------Dynamics------------------------------------
\newcommand*\rattleMass                       % solo atomic mass
  {\rattleMassNoIndex_{\rattleAtomIndex}}
\newcommand*\rattleMassFirst                  % first atomic mass in pair
    {\rattleMassNoIndex_{\rattleAtomIndexFirst}}
\newcommand*\rattleMassSecond                 % second atomic mass in pair
    {\rattleMassNoIndex_{\rattleAtomIndexSecond}}
\newcommand*\rattleForce
  {\rattleForceNoIndex_{\rattleAtomIndex}}
\newcommand*\rattleConstraintForce            % solo constraint force
  {\rattleConstraintForceNoIndex_{\rattleAtomIndex}}
\newcommand*\rattleConstraintForceFirst       % first constraint force in pair
    {\rattleConstraintForceNoIndex_{\rattleAtomIndexFirst}}
\newcommand*\rattleConstraintForceSecond      % second constraint force in pair
    {\rattleConstraintForceNoIndex_{\rattleAtomIndexSecond}}
\newcommand*\rattleConstraintForceBond        % constraint force along bond
        {\rattleConstraintForceNoIndex_{\rattleConstraintIndexSimple}}
\newcommand*\rattleConstraint
  {\rattleConstraintNoIndex_{\rattleConstraintIndexSimple}}
  \newcommand*\rattleConstraintDot
    {\dot{\rattleConstraintNoIndex}_{\rattleConstraintIndexSimple}}
\newcommand*\rattleLagrange
    {\rattleLagrangeNoIndex_{\rattleConstraintIndexSimple}}
\newcommand*\rattleLagrangeApprox
        {\rattleLagrangeApproxNoIndex_{\rattleConstraintIndexSimple}}
\newcommand*\rattleCorrectiveValue
  {\rattleCorrectiveValueNoIndex_{\rattleConstraintIndexSimple}}
%-----------------------------IterativeCorrection-------------------------------
\newcommand*\iteration[2]{#1^{(#2)}}            % iterative correction steps
\newcommand*\iterationThis[1]                   % arg^{i}
  {\iteration{#1}{\rattleCorrectionIndexThis}}
\newcommand*\iterationNext[1]                   % arg^{i+1}
  {\iteration{#1}{\rattleCorrectionIndexNext}}
%----------------------------------Vectors--------------------------------------
\newcommand*\rattleAtomPos{\rattlePos_{\rattleAtomIndex}}
\newcommand*\rattleAtomPosFirst{\rattlePos_{\rattleAtomIndexFirst}}
\newcommand*\rattleAtomPosSecond{\rattlePos_{\rattleAtomIndexSecond}}
\newcommand*\rattleAtomVel{\rattleVel_{\rattleAtomIndex}}
\newcommand*\rattleAtomVelFirst{\rattleVel_{\rattleAtomIndexFirst}}
\newcommand*\rattleAtomVelSecond{\rattleVel_{\rattleAtomIndexSecond}}
\newcommand*\rattlePosUnconstrained             % r^{(0)}
  {\iteration{\rattlePos}{0}}
\newcommand*\rattleVelUnconstrained             % v^{(0)}
  {\iteration{\rattleVel}{0}}
\newcommand*\rattleAtomPosUnconstrained         % r^{(0)}
  {\iteration{\rattleAtomPos}{0}}
\newcommand*\rattleAtomVelUnconstrained         % v^{(0)}
  {\iteration{\rattleAtomVel}{0}}
\newcommand*\rattleBondPos{\rattlePos_{\rattleConstraintIndexSimple}}
\newcommand*\rattleBondPosUnit{\hat{\rattlePos}_{\rattleConstraintIndexSimple}}
\newcommand*\rattleBondVel{\rattleVel_{\rattleConstraintIndexSimple}}
\newcommand*\rattleBondVelUnit{\hat{\rattleVel}_{\rattleConstraintIndexSimple}}
\newcommand*\rattleBondPosUnconstrained
  {\iteration{\rattleBondPos}{0}}
\newcommand*\rattleBondVelUnconstrained
  {\iteration{\rattleBondVel}{0}}
\newcommand*\rattleAtomPosUnconstrainedFirst
    {\iteration{\rattleAtomPosFirst}{0}}
\newcommand*\rattleAtomPosUnconstrainedSecond
    {\iteration{\rattleAtomPosSecond}{0}}
\newcommand*\rattleAtomVelUnconstrainedFirst
    {\iteration{\rattleAtomVelFirst}{0}}
\newcommand*\rattleAtomVelUnconstrainedSecond
    {\iteration{\rattleAtomVelSecond}{0}}
%---------------------------------Constants-------------------------------------
\newcommand*\rattleBondlength                   % fixed bond length
  {d_{\rattleConstraintIndexSimple}}
\newcommand*\rattleTol{\xi}                     % d-r/d tolerance
\newcommand*\rattleRRTol{\varepsilon}           % r(t) * r(t+dt) tolerance
\newcommand*\rattleRVTol{\rattleTol}            % v(t)/d tolerance
%-------------------------------------------------------------------------------

%-------------------------------------------------------------------------------
%\setlength{\parindent}{0pt}						          % do not indent
\numberwithin{equation}{section}                % number eqs within sections
\begin{document}
\lstset{language=Python}
%%%%%%%%%%%%%%%%%%%%%%%%%%%%%%%%%%%%%%%%%%%%%%%%%%%%%%%%%%%%%%%%%%%%%%%%%%%%%%%%
%					 	                        TITLE							                         %
%%%%%%%%%%%%%%%%%%%%%%%%%%%%%%%%%%%%%%%%%%%%%%%%%%%%%%%%%%%%%%%%%%%%%%%%%%%%%%%%
\title{Group Seminar (revised: \revisiondate)}
\thanks{Professor Stratt, Vale Cofer-Shabica, Yan Zhao, Andrew Ton, Mansheej Paul, Evan Coleman}
\author{Artur Avkhadiev}
\email{artur\_avkhadiev@brown.edu}
\affiliation{Department of Physics, Brown University, Providence RI 02912, USA}
\date{\seminardate}
%%%%%%%%%%%%%%%%%%%%%%%%%%%%%%%%%%%%%%%%%%%%%%%%%%%%%%%%%%%%%%%%%%%%%%%%%%%%%%%%
%					 	                       ABSTRACT							                       %
%%%%%%%%%%%%%%%%%%%%%%%%%%%%%%%%%%%%%%%%%%%%%%%%%%%%%%%%%%%%%%%%%%%%%%%%%%%%%%%%
\begin{abstract}
  This report consist of two parts:
  \begin{enumerate}
      \item A discussion of the constraint dynamics algorithm based on velocity Verlet numerical integration scheme, \rattle.
      \item A calculation of the kinetic energy of a freely-jointed polymer chain using the bond-vector representation in the CM frame of the molecule.
    \end{enumerate}
\end{abstract}
\maketitle
\tableofcontents						 % hide table of contents
%%%%%%%%%%%%%%%%%%%%%%%%%%%%%%%%%%%%%%%%%%%%%%%%%%%%%%%%%%%%%%%%%%%%%%%%%%%%%%%%
%					 	                        BODY							                         %
%%%%%%%%%%%%%%%%%%%%%%%%%%%%%%%%%%%%%%%%%%%%%%%%%%%%%%%%%%%%%%%%%%%%%%%%%%%%%%%%
% Add include statements below. All the text should be in the folders.
\documentclass[%
    reprint,
    superscriptaddress,
    %groupedaddress,
    %unsortedaddress,
    %runinaddress,
    %frontmatterverbose,
    %preprint,
    longbibliography,
    bibnotes,
    %showpacs,preprintnumbers,
    %nofootinbib,
    %nobibnotes,
    %bibnotes,
     amsmath,amssymb,
     %aps,
     aip,
     jcp,                                       % J. Chem. Phys
    %pra,
    %prb,
    %rmp,
    %prstab,
    %prstper,
    %floatfix,
    ]{revtex4-1}
%%%%%%%%%%%%%%%%%%%%%%%%%%%%%%%%%%%%%%%%%%%%%%%%%%%%%%%%%%%%%%%%%%%%%%%%%%%%%%%%
%					 	                       LIBRARIES							                     %
%%%%%%%%%%%%%%%%%%%%%%%%%%%%%%%%%%%%%%%%%%%%%%%%%%%%%%%%%%%%%%%%%%%%%%%%%%%%%%%%
% Add whatever libraries you need here. Add a description if you want
\usepackage{graphicx}			 % Include figure files
\usepackage{dcolumn}			 % Align table columns on decimal point
\usepackage{tcolorbox}		 % use tcolorbox environment to box equations
\usepackage[USenglish]{babel}
\usepackage[useregional]{datetime2}
\usepackage[utf8]{inputenc}
\usepackage[T1]{fontenc}
\usepackage{enumerate}
\usepackage{caption}
\usepackage{subcaption}
\usepackage{physics}
\usepackage{hyperref}
\usepackage{chemformula}
\usepackage{booktabs}
\usepackage{makecell}
\usepackage{fullpage}
\usepackage{natbib}
\usepackage{setspace}
\usepackage{amsmath}
\usepackage{bm}              % enchanced bold math symbols
\usepackage{amssymb}
\usepackage{listings} 			 % For code citations
\usepackage{amsfonts}
\usepackage{mathtools}
\usepackage{commath}
\usepackage{fancyhdr}
\usepackage{lastpage}
\usepackage{tikz}
\usepackage{graphicx}
	\graphicspath{ {images/} }
\usepackage{feynmp-auto}
\usepackage[toc,page]{appendix}
%%%%%%%%%%%%%%%%%%%%%%%%%%%%%%%%%%%%%%%%%%%%%%%%%%%%%%%%%%%%%%%%%%%%%%%%%%%%%%%%
%					 	                      NEW COMMANDS							                   %
%%%%%%%%%%%%%%%%%%%%%%%%%%%%%%%%%%%%%%%%%%%%%%%%%%%%%%%%%%%%%%%%%%%%%%%%%%%%%%%%
% Redefine commands if necessary
\renewcommand\vec{\mathbf}                      % use boldface for vectors
% New commands
%------------------------------------Date---------------------------------------
\newcommand{\seminardate}{\DTMdisplaydate{2017}{7}{25}{-1}}
\newcommand{\revisiondate}{\DTMdisplaydate{2017}{7}{26}{-1}}
%-------------------------------------------------------------------------------
\newcommand*\Del{\vec{\nabla}}                  % del operator
\newcommand*\diff{\mathop{}\!\mathrm{d}}		    % differential d
\newcommand*\Diff[1]{\mathop{}\!\mathrm{d^#1}}	% d^(...)
\newcommand*{\scalarnorm}[1]                    % scalar norm |*|
  {\left\lvert#1\right\rvert}
\newcommand*{\vectornorm}[1]                    % vector norm ||*||
  {\left\lVert#1\right\rVert}
\newcommand*\onehalf{\frac{1}{2}}               % fraction 1/2
%------------------------------------Time---------------------------------------
\newcommand*\dlt[1]{\delta #1}
\newcommand*\timestep{\dlt{t}}              % timestep dt
\newcommand*\halfstep                       % half timestep 1/2 dt
  {\onehalf \timestep}
\newcommand*\timeInHalfStep{\left(t + \halfstep\right)}
\newcommand*\timeInFullStep{\left(t + \timestep\right)}
%-------------------------------------------------------------------------------
\input{rattle/definitions}
\input{ke_polymer/definitions}
%-------------------------------------------------------------------------------
%\setlength{\parindent}{0pt}						          % do not indent
\numberwithin{equation}{section}                % number eqs within sections
\begin{document}
\lstset{language=Python}
%%%%%%%%%%%%%%%%%%%%%%%%%%%%%%%%%%%%%%%%%%%%%%%%%%%%%%%%%%%%%%%%%%%%%%%%%%%%%%%%
%					 	                        TITLE							                         %
%%%%%%%%%%%%%%%%%%%%%%%%%%%%%%%%%%%%%%%%%%%%%%%%%%%%%%%%%%%%%%%%%%%%%%%%%%%%%%%%
\title{Group Seminar (revised: \revisiondate)}
\thanks{Professor Stratt, Vale Cofer-Shabica, Yan Zhao, Andrew Ton, Mansheej Paul, Evan Coleman}
\author{Artur Avkhadiev}
\email{artur\_avkhadiev@brown.edu}
\affiliation{Department of Physics, Brown University, Providence RI 02912, USA}
\date{\seminardate}
%%%%%%%%%%%%%%%%%%%%%%%%%%%%%%%%%%%%%%%%%%%%%%%%%%%%%%%%%%%%%%%%%%%%%%%%%%%%%%%%
%					 	                       ABSTRACT							                       %
%%%%%%%%%%%%%%%%%%%%%%%%%%%%%%%%%%%%%%%%%%%%%%%%%%%%%%%%%%%%%%%%%%%%%%%%%%%%%%%%
\begin{abstract}
  This report consist of two parts:
  \begin{enumerate}
      \item A discussion of the constraint dynamics algorithm based on velocity Verlet numerical integration scheme, \rattle.
      \item A calculation of the kinetic energy of a freely-jointed polymer chain using the bond-vector representation in the CM frame of the molecule.
    \end{enumerate}
\end{abstract}
\maketitle
\tableofcontents						 % hide table of contents
%%%%%%%%%%%%%%%%%%%%%%%%%%%%%%%%%%%%%%%%%%%%%%%%%%%%%%%%%%%%%%%%%%%%%%%%%%%%%%%%
%					 	                        BODY							                         %
%%%%%%%%%%%%%%%%%%%%%%%%%%%%%%%%%%%%%%%%%%%%%%%%%%%%%%%%%%%%%%%%%%%%%%%%%%%%%%%%
% Add include statements below. All the text should be in the folders.
\input{rattle/main}
\input{ke_polymer/main}
%-------------------------------------------------------------------------------
%%%%%%%%%%%%%%%%%%%%%%%%%%%%%%%%%%%%%%%%%%%%%%%%%%%%%%%%%%%%%%%%%%%%%%%%%%%%%%%%
%					                        BIBLIOGRAPHY							                   %
%%%%%%%%%%%%%%%%%%%%%%%%%%%%%%%%%%%%%%%%%%%%%%%%%%%%%%%%%%%%%%%%%%%%%%%%%%%%%%%%
% The bibliography file is reference,bib. If you need to add a
% reference, make sure all the entries look good (e.g., ensure
% proper capitalization by putting {} around the letters to be
% capitalized. Reference the books by chapters!
\nocite{*}
\bibliographystyle{apalike}
\bibliography{reference}
%-----------------------------
%%%%%%%%%%%%%%%%%%%%%%%%%%%%%%%%%%%%%%%%%%%%%%%%%%%%%%%%%%%%%%%%%%%%%%%%%%%%%%%%
%					 	                      APPENDICES							                     %
%%%%%%%%%%%%%%%%%%%%%%%%%%%%%%%%%%%%%%%%%%%%%%%%%%%%%%%%%%%%%%%%%%%%%%%%%%%%%%%%
% Add all appendices in appendices/main.tex with include statements.
% see appendices/sub-example-app.tex for an example.
\onecolumngrid
\input{appendices/main}
%-------------------------------------------------------------------------------
\end{document}

\documentclass[%
    reprint,
    superscriptaddress,
    %groupedaddress,
    %unsortedaddress,
    %runinaddress,
    %frontmatterverbose,
    %preprint,
    longbibliography,
    bibnotes,
    %showpacs,preprintnumbers,
    %nofootinbib,
    %nobibnotes,
    %bibnotes,
     amsmath,amssymb,
     %aps,
     aip,
     jcp,                                       % J. Chem. Phys
    %pra,
    %prb,
    %rmp,
    %prstab,
    %prstper,
    %floatfix,
    ]{revtex4-1}
%%%%%%%%%%%%%%%%%%%%%%%%%%%%%%%%%%%%%%%%%%%%%%%%%%%%%%%%%%%%%%%%%%%%%%%%%%%%%%%%
%					 	                       LIBRARIES							                     %
%%%%%%%%%%%%%%%%%%%%%%%%%%%%%%%%%%%%%%%%%%%%%%%%%%%%%%%%%%%%%%%%%%%%%%%%%%%%%%%%
% Add whatever libraries you need here. Add a description if you want
\usepackage{graphicx}			 % Include figure files
\usepackage{dcolumn}			 % Align table columns on decimal point
\usepackage{tcolorbox}		 % use tcolorbox environment to box equations
\usepackage[USenglish]{babel}
\usepackage[useregional]{datetime2}
\usepackage[utf8]{inputenc}
\usepackage[T1]{fontenc}
\usepackage{enumerate}
\usepackage{caption}
\usepackage{subcaption}
\usepackage{physics}
\usepackage{hyperref}
\usepackage{chemformula}
\usepackage{booktabs}
\usepackage{makecell}
\usepackage{fullpage}
\usepackage{natbib}
\usepackage{setspace}
\usepackage{amsmath}
\usepackage{bm}              % enchanced bold math symbols
\usepackage{amssymb}
\usepackage{listings} 			 % For code citations
\usepackage{amsfonts}
\usepackage{mathtools}
\usepackage{commath}
\usepackage{fancyhdr}
\usepackage{lastpage}
\usepackage{tikz}
\usepackage{graphicx}
	\graphicspath{ {images/} }
\usepackage{feynmp-auto}
\usepackage[toc,page]{appendix}
%%%%%%%%%%%%%%%%%%%%%%%%%%%%%%%%%%%%%%%%%%%%%%%%%%%%%%%%%%%%%%%%%%%%%%%%%%%%%%%%
%					 	                      NEW COMMANDS							                   %
%%%%%%%%%%%%%%%%%%%%%%%%%%%%%%%%%%%%%%%%%%%%%%%%%%%%%%%%%%%%%%%%%%%%%%%%%%%%%%%%
% Redefine commands if necessary
\renewcommand\vec{\mathbf}                      % use boldface for vectors
% New commands
%------------------------------------Date---------------------------------------
\newcommand{\seminardate}{\DTMdisplaydate{2017}{7}{25}{-1}}
\newcommand{\revisiondate}{\DTMdisplaydate{2017}{7}{26}{-1}}
%-------------------------------------------------------------------------------
\newcommand*\Del{\vec{\nabla}}                  % del operator
\newcommand*\diff{\mathop{}\!\mathrm{d}}		    % differential d
\newcommand*\Diff[1]{\mathop{}\!\mathrm{d^#1}}	% d^(...)
\newcommand*{\scalarnorm}[1]                    % scalar norm |*|
  {\left\lvert#1\right\rvert}
\newcommand*{\vectornorm}[1]                    % vector norm ||*||
  {\left\lVert#1\right\rVert}
\newcommand*\onehalf{\frac{1}{2}}               % fraction 1/2
%------------------------------------Time---------------------------------------
\newcommand*\dlt[1]{\delta #1}
\newcommand*\timestep{\dlt{t}}              % timestep dt
\newcommand*\halfstep                       % half timestep 1/2 dt
  {\onehalf \timestep}
\newcommand*\timeInHalfStep{\left(t + \halfstep\right)}
\newcommand*\timeInFullStep{\left(t + \timestep\right)}
%-------------------------------------------------------------------------------
\input{rattle/definitions}
\input{ke_polymer/definitions}
%-------------------------------------------------------------------------------
%\setlength{\parindent}{0pt}						          % do not indent
\numberwithin{equation}{section}                % number eqs within sections
\begin{document}
\lstset{language=Python}
%%%%%%%%%%%%%%%%%%%%%%%%%%%%%%%%%%%%%%%%%%%%%%%%%%%%%%%%%%%%%%%%%%%%%%%%%%%%%%%%
%					 	                        TITLE							                         %
%%%%%%%%%%%%%%%%%%%%%%%%%%%%%%%%%%%%%%%%%%%%%%%%%%%%%%%%%%%%%%%%%%%%%%%%%%%%%%%%
\title{Group Seminar (revised: \revisiondate)}
\thanks{Professor Stratt, Vale Cofer-Shabica, Yan Zhao, Andrew Ton, Mansheej Paul, Evan Coleman}
\author{Artur Avkhadiev}
\email{artur\_avkhadiev@brown.edu}
\affiliation{Department of Physics, Brown University, Providence RI 02912, USA}
\date{\seminardate}
%%%%%%%%%%%%%%%%%%%%%%%%%%%%%%%%%%%%%%%%%%%%%%%%%%%%%%%%%%%%%%%%%%%%%%%%%%%%%%%%
%					 	                       ABSTRACT							                       %
%%%%%%%%%%%%%%%%%%%%%%%%%%%%%%%%%%%%%%%%%%%%%%%%%%%%%%%%%%%%%%%%%%%%%%%%%%%%%%%%
\begin{abstract}
  This report consist of two parts:
  \begin{enumerate}
      \item A discussion of the constraint dynamics algorithm based on velocity Verlet numerical integration scheme, \rattle.
      \item A calculation of the kinetic energy of a freely-jointed polymer chain using the bond-vector representation in the CM frame of the molecule.
    \end{enumerate}
\end{abstract}
\maketitle
\tableofcontents						 % hide table of contents
%%%%%%%%%%%%%%%%%%%%%%%%%%%%%%%%%%%%%%%%%%%%%%%%%%%%%%%%%%%%%%%%%%%%%%%%%%%%%%%%
%					 	                        BODY							                         %
%%%%%%%%%%%%%%%%%%%%%%%%%%%%%%%%%%%%%%%%%%%%%%%%%%%%%%%%%%%%%%%%%%%%%%%%%%%%%%%%
% Add include statements below. All the text should be in the folders.
\input{rattle/main}
\input{ke_polymer/main}
%-------------------------------------------------------------------------------
%%%%%%%%%%%%%%%%%%%%%%%%%%%%%%%%%%%%%%%%%%%%%%%%%%%%%%%%%%%%%%%%%%%%%%%%%%%%%%%%
%					                        BIBLIOGRAPHY							                   %
%%%%%%%%%%%%%%%%%%%%%%%%%%%%%%%%%%%%%%%%%%%%%%%%%%%%%%%%%%%%%%%%%%%%%%%%%%%%%%%%
% The bibliography file is reference,bib. If you need to add a
% reference, make sure all the entries look good (e.g., ensure
% proper capitalization by putting {} around the letters to be
% capitalized. Reference the books by chapters!
\nocite{*}
\bibliographystyle{apalike}
\bibliography{reference}
%-----------------------------
%%%%%%%%%%%%%%%%%%%%%%%%%%%%%%%%%%%%%%%%%%%%%%%%%%%%%%%%%%%%%%%%%%%%%%%%%%%%%%%%
%					 	                      APPENDICES							                     %
%%%%%%%%%%%%%%%%%%%%%%%%%%%%%%%%%%%%%%%%%%%%%%%%%%%%%%%%%%%%%%%%%%%%%%%%%%%%%%%%
% Add all appendices in appendices/main.tex with include statements.
% see appendices/sub-example-app.tex for an example.
\onecolumngrid
\input{appendices/main}
%-------------------------------------------------------------------------------
\end{document}

%-------------------------------------------------------------------------------
%%%%%%%%%%%%%%%%%%%%%%%%%%%%%%%%%%%%%%%%%%%%%%%%%%%%%%%%%%%%%%%%%%%%%%%%%%%%%%%%
%					                        BIBLIOGRAPHY							                   %
%%%%%%%%%%%%%%%%%%%%%%%%%%%%%%%%%%%%%%%%%%%%%%%%%%%%%%%%%%%%%%%%%%%%%%%%%%%%%%%%
% The bibliography file is reference,bib. If you need to add a
% reference, make sure all the entries look good (e.g., ensure
% proper capitalization by putting {} around the letters to be
% capitalized. Reference the books by chapters!
\nocite{*}
\bibliographystyle{apalike}
\bibliography{reference}
%-----------------------------
%%%%%%%%%%%%%%%%%%%%%%%%%%%%%%%%%%%%%%%%%%%%%%%%%%%%%%%%%%%%%%%%%%%%%%%%%%%%%%%%
%					 	                      APPENDICES							                     %
%%%%%%%%%%%%%%%%%%%%%%%%%%%%%%%%%%%%%%%%%%%%%%%%%%%%%%%%%%%%%%%%%%%%%%%%%%%%%%%%
% Add all appendices in appendices/main.tex with include statements.
% see appendices/sub-example-app.tex for an example.
\onecolumngrid
\documentclass[%
    reprint,
    superscriptaddress,
    %groupedaddress,
    %unsortedaddress,
    %runinaddress,
    %frontmatterverbose,
    %preprint,
    longbibliography,
    bibnotes,
    %showpacs,preprintnumbers,
    %nofootinbib,
    %nobibnotes,
    %bibnotes,
     amsmath,amssymb,
     %aps,
     aip,
     jcp,                                       % J. Chem. Phys
    %pra,
    %prb,
    %rmp,
    %prstab,
    %prstper,
    %floatfix,
    ]{revtex4-1}
%%%%%%%%%%%%%%%%%%%%%%%%%%%%%%%%%%%%%%%%%%%%%%%%%%%%%%%%%%%%%%%%%%%%%%%%%%%%%%%%
%					 	                       LIBRARIES							                     %
%%%%%%%%%%%%%%%%%%%%%%%%%%%%%%%%%%%%%%%%%%%%%%%%%%%%%%%%%%%%%%%%%%%%%%%%%%%%%%%%
% Add whatever libraries you need here. Add a description if you want
\usepackage{graphicx}			 % Include figure files
\usepackage{dcolumn}			 % Align table columns on decimal point
\usepackage{tcolorbox}		 % use tcolorbox environment to box equations
\usepackage[USenglish]{babel}
\usepackage[useregional]{datetime2}
\usepackage[utf8]{inputenc}
\usepackage[T1]{fontenc}
\usepackage{enumerate}
\usepackage{caption}
\usepackage{subcaption}
\usepackage{physics}
\usepackage{hyperref}
\usepackage{chemformula}
\usepackage{booktabs}
\usepackage{makecell}
\usepackage{fullpage}
\usepackage{natbib}
\usepackage{setspace}
\usepackage{amsmath}
\usepackage{bm}              % enchanced bold math symbols
\usepackage{amssymb}
\usepackage{listings} 			 % For code citations
\usepackage{amsfonts}
\usepackage{mathtools}
\usepackage{commath}
\usepackage{fancyhdr}
\usepackage{lastpage}
\usepackage{tikz}
\usepackage{graphicx}
	\graphicspath{ {images/} }
\usepackage{feynmp-auto}
\usepackage[toc,page]{appendix}
%%%%%%%%%%%%%%%%%%%%%%%%%%%%%%%%%%%%%%%%%%%%%%%%%%%%%%%%%%%%%%%%%%%%%%%%%%%%%%%%
%					 	                      NEW COMMANDS							                   %
%%%%%%%%%%%%%%%%%%%%%%%%%%%%%%%%%%%%%%%%%%%%%%%%%%%%%%%%%%%%%%%%%%%%%%%%%%%%%%%%
% Redefine commands if necessary
\renewcommand\vec{\mathbf}                      % use boldface for vectors
% New commands
%------------------------------------Date---------------------------------------
\newcommand{\seminardate}{\DTMdisplaydate{2017}{7}{25}{-1}}
\newcommand{\revisiondate}{\DTMdisplaydate{2017}{7}{26}{-1}}
%-------------------------------------------------------------------------------
\newcommand*\Del{\vec{\nabla}}                  % del operator
\newcommand*\diff{\mathop{}\!\mathrm{d}}		    % differential d
\newcommand*\Diff[1]{\mathop{}\!\mathrm{d^#1}}	% d^(...)
\newcommand*{\scalarnorm}[1]                    % scalar norm |*|
  {\left\lvert#1\right\rvert}
\newcommand*{\vectornorm}[1]                    % vector norm ||*||
  {\left\lVert#1\right\rVert}
\newcommand*\onehalf{\frac{1}{2}}               % fraction 1/2
%------------------------------------Time---------------------------------------
\newcommand*\dlt[1]{\delta #1}
\newcommand*\timestep{\dlt{t}}              % timestep dt
\newcommand*\halfstep                       % half timestep 1/2 dt
  {\onehalf \timestep}
\newcommand*\timeInHalfStep{\left(t + \halfstep\right)}
\newcommand*\timeInFullStep{\left(t + \timestep\right)}
%-------------------------------------------------------------------------------
\input{rattle/definitions}
\input{ke_polymer/definitions}
%-------------------------------------------------------------------------------
%\setlength{\parindent}{0pt}						          % do not indent
\numberwithin{equation}{section}                % number eqs within sections
\begin{document}
\lstset{language=Python}
%%%%%%%%%%%%%%%%%%%%%%%%%%%%%%%%%%%%%%%%%%%%%%%%%%%%%%%%%%%%%%%%%%%%%%%%%%%%%%%%
%					 	                        TITLE							                         %
%%%%%%%%%%%%%%%%%%%%%%%%%%%%%%%%%%%%%%%%%%%%%%%%%%%%%%%%%%%%%%%%%%%%%%%%%%%%%%%%
\title{Group Seminar (revised: \revisiondate)}
\thanks{Professor Stratt, Vale Cofer-Shabica, Yan Zhao, Andrew Ton, Mansheej Paul, Evan Coleman}
\author{Artur Avkhadiev}
\email{artur\_avkhadiev@brown.edu}
\affiliation{Department of Physics, Brown University, Providence RI 02912, USA}
\date{\seminardate}
%%%%%%%%%%%%%%%%%%%%%%%%%%%%%%%%%%%%%%%%%%%%%%%%%%%%%%%%%%%%%%%%%%%%%%%%%%%%%%%%
%					 	                       ABSTRACT							                       %
%%%%%%%%%%%%%%%%%%%%%%%%%%%%%%%%%%%%%%%%%%%%%%%%%%%%%%%%%%%%%%%%%%%%%%%%%%%%%%%%
\begin{abstract}
  This report consist of two parts:
  \begin{enumerate}
      \item A discussion of the constraint dynamics algorithm based on velocity Verlet numerical integration scheme, \rattle.
      \item A calculation of the kinetic energy of a freely-jointed polymer chain using the bond-vector representation in the CM frame of the molecule.
    \end{enumerate}
\end{abstract}
\maketitle
\tableofcontents						 % hide table of contents
%%%%%%%%%%%%%%%%%%%%%%%%%%%%%%%%%%%%%%%%%%%%%%%%%%%%%%%%%%%%%%%%%%%%%%%%%%%%%%%%
%					 	                        BODY							                         %
%%%%%%%%%%%%%%%%%%%%%%%%%%%%%%%%%%%%%%%%%%%%%%%%%%%%%%%%%%%%%%%%%%%%%%%%%%%%%%%%
% Add include statements below. All the text should be in the folders.
\input{rattle/main}
\input{ke_polymer/main}
%-------------------------------------------------------------------------------
%%%%%%%%%%%%%%%%%%%%%%%%%%%%%%%%%%%%%%%%%%%%%%%%%%%%%%%%%%%%%%%%%%%%%%%%%%%%%%%%
%					                        BIBLIOGRAPHY							                   %
%%%%%%%%%%%%%%%%%%%%%%%%%%%%%%%%%%%%%%%%%%%%%%%%%%%%%%%%%%%%%%%%%%%%%%%%%%%%%%%%
% The bibliography file is reference,bib. If you need to add a
% reference, make sure all the entries look good (e.g., ensure
% proper capitalization by putting {} around the letters to be
% capitalized. Reference the books by chapters!
\nocite{*}
\bibliographystyle{apalike}
\bibliography{reference}
%-----------------------------
%%%%%%%%%%%%%%%%%%%%%%%%%%%%%%%%%%%%%%%%%%%%%%%%%%%%%%%%%%%%%%%%%%%%%%%%%%%%%%%%
%					 	                      APPENDICES							                     %
%%%%%%%%%%%%%%%%%%%%%%%%%%%%%%%%%%%%%%%%%%%%%%%%%%%%%%%%%%%%%%%%%%%%%%%%%%%%%%%%
% Add all appendices in appendices/main.tex with include statements.
% see appendices/sub-example-app.tex for an example.
\onecolumngrid
\input{appendices/main}
%-------------------------------------------------------------------------------
\end{document}

%-------------------------------------------------------------------------------
\end{document}

\documentclass[%
    reprint,
    superscriptaddress,
    %groupedaddress,
    %unsortedaddress,
    %runinaddress,
    %frontmatterverbose,
    %preprint,
    longbibliography,
    bibnotes,
    %showpacs,preprintnumbers,
    %nofootinbib,
    %nobibnotes,
    %bibnotes,
     amsmath,amssymb,
     %aps,
     aip,
     jcp,                                       % J. Chem. Phys
    %pra,
    %prb,
    %rmp,
    %prstab,
    %prstper,
    %floatfix,
    ]{revtex4-1}
%%%%%%%%%%%%%%%%%%%%%%%%%%%%%%%%%%%%%%%%%%%%%%%%%%%%%%%%%%%%%%%%%%%%%%%%%%%%%%%%
%					 	                       LIBRARIES							                     %
%%%%%%%%%%%%%%%%%%%%%%%%%%%%%%%%%%%%%%%%%%%%%%%%%%%%%%%%%%%%%%%%%%%%%%%%%%%%%%%%
% Add whatever libraries you need here. Add a description if you want
\usepackage{graphicx}			 % Include figure files
\usepackage{dcolumn}			 % Align table columns on decimal point
\usepackage{tcolorbox}		 % use tcolorbox environment to box equations
\usepackage[USenglish]{babel}
\usepackage[useregional]{datetime2}
\usepackage[utf8]{inputenc}
\usepackage[T1]{fontenc}
\usepackage{enumerate}
\usepackage{caption}
\usepackage{subcaption}
\usepackage{physics}
\usepackage{hyperref}
\usepackage{chemformula}
\usepackage{booktabs}
\usepackage{makecell}
\usepackage{fullpage}
\usepackage{natbib}
\usepackage{setspace}
\usepackage{amsmath}
\usepackage{bm}              % enchanced bold math symbols
\usepackage{amssymb}
\usepackage{listings} 			 % For code citations
\usepackage{amsfonts}
\usepackage{mathtools}
\usepackage{commath}
\usepackage{fancyhdr}
\usepackage{lastpage}
\usepackage{tikz}
\usepackage{graphicx}
	\graphicspath{ {images/} }
\usepackage{feynmp-auto}
\usepackage[toc,page]{appendix}
%%%%%%%%%%%%%%%%%%%%%%%%%%%%%%%%%%%%%%%%%%%%%%%%%%%%%%%%%%%%%%%%%%%%%%%%%%%%%%%%
%					 	                      NEW COMMANDS							                   %
%%%%%%%%%%%%%%%%%%%%%%%%%%%%%%%%%%%%%%%%%%%%%%%%%%%%%%%%%%%%%%%%%%%%%%%%%%%%%%%%
% Redefine commands if necessary
\renewcommand\vec{\mathbf}                      % use boldface for vectors
% New commands
%------------------------------------Date---------------------------------------
\newcommand{\seminardate}{\DTMdisplaydate{2017}{7}{25}{-1}}
\newcommand{\revisiondate}{\DTMdisplaydate{2017}{7}{26}{-1}}
%-------------------------------------------------------------------------------
\newcommand*\Del{\vec{\nabla}}                  % del operator
\newcommand*\diff{\mathop{}\!\mathrm{d}}		    % differential d
\newcommand*\Diff[1]{\mathop{}\!\mathrm{d^#1}}	% d^(...)
\newcommand*{\scalarnorm}[1]                    % scalar norm |*|
  {\left\lvert#1\right\rvert}
\newcommand*{\vectornorm}[1]                    % vector norm ||*||
  {\left\lVert#1\right\rVert}
\newcommand*\onehalf{\frac{1}{2}}               % fraction 1/2
%------------------------------------Time---------------------------------------
\newcommand*\dlt[1]{\delta #1}
\newcommand*\timestep{\dlt{t}}              % timestep dt
\newcommand*\halfstep                       % half timestep 1/2 dt
  {\onehalf \timestep}
\newcommand*\timeInHalfStep{\left(t + \halfstep\right)}
\newcommand*\timeInFullStep{\left(t + \timestep\right)}
%-------------------------------------------------------------------------------
%%%%%%%%%%%%%%%%%%%%%%%%%%%%%%%%%%%%%%%%%%%%%%%%%%%%%%%%%%%%%%%%%%%%%%%%%%%%%%%%
%					 	                    DEFINITIONS								                     %
%%%%%%%%%%%%%%%%%%%%%%%%%%%%%%%%%%%%%%%%%%%%%%%%%%%%%%%%%%%%%%%%%%%%%%%%%%%%%%%%
%-------------------------------------------------------------------------------
\newcommand*\rattle{\textsf{RATTLE} }       % display name of algorithm
%-------------------------------------------------------------------------------
\newcommand*\rattlePos{\vec{r}}             % position vector
\newcommand*\rattleVel{\vec{v}}             % velocity vector
\newcommand*\rattleAcc{\vec{a}}             % acceleration vector
%-------------------------------------------------------------------------------
\newcommand*\rattleMassNoIndex{m}
\newcommand*\rattleForceNoIndex{\vec{F}}
\newcommand*\rattleConstraintForceNoIndex{\vec{G}}
\newcommand*\rattleConstraintNoIndex{\sigma}
\newcommand*\rattleLagrangeNoIndex{\lambda}
\newcommand*\rattleLagrangeApproxNoIndex{\gamma}
\newcommand*\rattleCorrectiveValueNoIndex{g}
%---------------------Definitions for General Formulation ----------------------
\input{rattle/general/definitions}
%-----------------------------------Indexing------------------------------------
\newcommand*\rattleAtomIndex{a}               % solo atom index
\newcommand*\rattleAtomIndexFirst{a}          % first index in atomic pair
\newcommand*\rattleAtomIndexSecond{b}         % second index in atomic pair
\newcommand*\rattleNAtoms{N}                  % number of atoms
\newcommand*\rattleConstraintSet{K}           % set of constraints
\newcommand*\rattleConstraintIndex            % index for constraints
  {\left(
    \rattleAtomIndexFirst,\,\rattleAtomIndexSecond
  \right) \in \rattleConstraintSet}
\newcommand*\rattleConstraintIndexSimple      % simplified index for constraints
  {\rattleAtomIndexFirst\rattleAtomIndexSecond}
\newcommand*\rattleCorrectionIndex{i}         % index for iterative correction
\newcommand*\rattleCorrectionIndexThis        % current for iterative correction
  {\rattleCorrectionIndex}
\newcommand*\rattleCorrectionIndexNext        % next for iterative correction
    {\rattleCorrectionIndex + 1}
\newcommand*\rattleCorrectionIndexLast{m}     % last step of correction
\newcommand*\rattleNCorrections{M}            % number of iterative corrections
%-----------------------------------Dynamics------------------------------------
\newcommand*\rattleMass                       % solo atomic mass
  {\rattleMassNoIndex_{\rattleAtomIndex}}
\newcommand*\rattleMassFirst                  % first atomic mass in pair
    {\rattleMassNoIndex_{\rattleAtomIndexFirst}}
\newcommand*\rattleMassSecond                 % second atomic mass in pair
    {\rattleMassNoIndex_{\rattleAtomIndexSecond}}
\newcommand*\rattleForce
  {\rattleForceNoIndex_{\rattleAtomIndex}}
\newcommand*\rattleConstraintForce            % solo constraint force
  {\rattleConstraintForceNoIndex_{\rattleAtomIndex}}
\newcommand*\rattleConstraintForceFirst       % first constraint force in pair
    {\rattleConstraintForceNoIndex_{\rattleAtomIndexFirst}}
\newcommand*\rattleConstraintForceSecond      % second constraint force in pair
    {\rattleConstraintForceNoIndex_{\rattleAtomIndexSecond}}
\newcommand*\rattleConstraintForceBond        % constraint force along bond
        {\rattleConstraintForceNoIndex_{\rattleConstraintIndexSimple}}
\newcommand*\rattleConstraint
  {\rattleConstraintNoIndex_{\rattleConstraintIndexSimple}}
  \newcommand*\rattleConstraintDot
    {\dot{\rattleConstraintNoIndex}_{\rattleConstraintIndexSimple}}
\newcommand*\rattleLagrange
    {\rattleLagrangeNoIndex_{\rattleConstraintIndexSimple}}
\newcommand*\rattleLagrangeApprox
        {\rattleLagrangeApproxNoIndex_{\rattleConstraintIndexSimple}}
\newcommand*\rattleCorrectiveValue
  {\rattleCorrectiveValueNoIndex_{\rattleConstraintIndexSimple}}
%-----------------------------IterativeCorrection-------------------------------
\newcommand*\iteration[2]{#1^{(#2)}}            % iterative correction steps
\newcommand*\iterationThis[1]                   % arg^{i}
  {\iteration{#1}{\rattleCorrectionIndexThis}}
\newcommand*\iterationNext[1]                   % arg^{i+1}
  {\iteration{#1}{\rattleCorrectionIndexNext}}
%----------------------------------Vectors--------------------------------------
\newcommand*\rattleAtomPos{\rattlePos_{\rattleAtomIndex}}
\newcommand*\rattleAtomPosFirst{\rattlePos_{\rattleAtomIndexFirst}}
\newcommand*\rattleAtomPosSecond{\rattlePos_{\rattleAtomIndexSecond}}
\newcommand*\rattleAtomVel{\rattleVel_{\rattleAtomIndex}}
\newcommand*\rattleAtomVelFirst{\rattleVel_{\rattleAtomIndexFirst}}
\newcommand*\rattleAtomVelSecond{\rattleVel_{\rattleAtomIndexSecond}}
\newcommand*\rattlePosUnconstrained             % r^{(0)}
  {\iteration{\rattlePos}{0}}
\newcommand*\rattleVelUnconstrained             % v^{(0)}
  {\iteration{\rattleVel}{0}}
\newcommand*\rattleAtomPosUnconstrained         % r^{(0)}
  {\iteration{\rattleAtomPos}{0}}
\newcommand*\rattleAtomVelUnconstrained         % v^{(0)}
  {\iteration{\rattleAtomVel}{0}}
\newcommand*\rattleBondPos{\rattlePos_{\rattleConstraintIndexSimple}}
\newcommand*\rattleBondPosUnit{\hat{\rattlePos}_{\rattleConstraintIndexSimple}}
\newcommand*\rattleBondVel{\rattleVel_{\rattleConstraintIndexSimple}}
\newcommand*\rattleBondVelUnit{\hat{\rattleVel}_{\rattleConstraintIndexSimple}}
\newcommand*\rattleBondPosUnconstrained
  {\iteration{\rattleBondPos}{0}}
\newcommand*\rattleBondVelUnconstrained
  {\iteration{\rattleBondVel}{0}}
\newcommand*\rattleAtomPosUnconstrainedFirst
    {\iteration{\rattleAtomPosFirst}{0}}
\newcommand*\rattleAtomPosUnconstrainedSecond
    {\iteration{\rattleAtomPosSecond}{0}}
\newcommand*\rattleAtomVelUnconstrainedFirst
    {\iteration{\rattleAtomVelFirst}{0}}
\newcommand*\rattleAtomVelUnconstrainedSecond
    {\iteration{\rattleAtomVelSecond}{0}}
%---------------------------------Constants-------------------------------------
\newcommand*\rattleBondlength                   % fixed bond length
  {d_{\rattleConstraintIndexSimple}}
\newcommand*\rattleTol{\xi}                     % d-r/d tolerance
\newcommand*\rattleRRTol{\varepsilon}           % r(t) * r(t+dt) tolerance
\newcommand*\rattleRVTol{\rattleTol}            % v(t)/d tolerance
%-------------------------------------------------------------------------------

%%%%%%%%%%%%%%%%%%%%%%%%%%%%%%%%%%%%%%%%%%%%%%%%%%%%%%%%%%%%%%%%%%%%%%%%%%%%%%%%
%					 	                    DEFINITIONS								                     %
%%%%%%%%%%%%%%%%%%%%%%%%%%%%%%%%%%%%%%%%%%%%%%%%%%%%%%%%%%%%%%%%%%%%%%%%%%%%%%%%
%-------------------------------------------------------------------------------
\newcommand*\rattle{\textsf{RATTLE} }       % display name of algorithm
%-------------------------------------------------------------------------------
\newcommand*\rattlePos{\vec{r}}             % position vector
\newcommand*\rattleVel{\vec{v}}             % velocity vector
\newcommand*\rattleAcc{\vec{a}}             % acceleration vector
%-------------------------------------------------------------------------------
\newcommand*\rattleMassNoIndex{m}
\newcommand*\rattleForceNoIndex{\vec{F}}
\newcommand*\rattleConstraintForceNoIndex{\vec{G}}
\newcommand*\rattleConstraintNoIndex{\sigma}
\newcommand*\rattleLagrangeNoIndex{\lambda}
\newcommand*\rattleLagrangeApproxNoIndex{\gamma}
\newcommand*\rattleCorrectiveValueNoIndex{g}
%---------------------Definitions for General Formulation ----------------------
\input{rattle/general/definitions}
%-----------------------------------Indexing------------------------------------
\newcommand*\rattleAtomIndex{a}               % solo atom index
\newcommand*\rattleAtomIndexFirst{a}          % first index in atomic pair
\newcommand*\rattleAtomIndexSecond{b}         % second index in atomic pair
\newcommand*\rattleNAtoms{N}                  % number of atoms
\newcommand*\rattleConstraintSet{K}           % set of constraints
\newcommand*\rattleConstraintIndex            % index for constraints
  {\left(
    \rattleAtomIndexFirst,\,\rattleAtomIndexSecond
  \right) \in \rattleConstraintSet}
\newcommand*\rattleConstraintIndexSimple      % simplified index for constraints
  {\rattleAtomIndexFirst\rattleAtomIndexSecond}
\newcommand*\rattleCorrectionIndex{i}         % index for iterative correction
\newcommand*\rattleCorrectionIndexThis        % current for iterative correction
  {\rattleCorrectionIndex}
\newcommand*\rattleCorrectionIndexNext        % next for iterative correction
    {\rattleCorrectionIndex + 1}
\newcommand*\rattleCorrectionIndexLast{m}     % last step of correction
\newcommand*\rattleNCorrections{M}            % number of iterative corrections
%-----------------------------------Dynamics------------------------------------
\newcommand*\rattleMass                       % solo atomic mass
  {\rattleMassNoIndex_{\rattleAtomIndex}}
\newcommand*\rattleMassFirst                  % first atomic mass in pair
    {\rattleMassNoIndex_{\rattleAtomIndexFirst}}
\newcommand*\rattleMassSecond                 % second atomic mass in pair
    {\rattleMassNoIndex_{\rattleAtomIndexSecond}}
\newcommand*\rattleForce
  {\rattleForceNoIndex_{\rattleAtomIndex}}
\newcommand*\rattleConstraintForce            % solo constraint force
  {\rattleConstraintForceNoIndex_{\rattleAtomIndex}}
\newcommand*\rattleConstraintForceFirst       % first constraint force in pair
    {\rattleConstraintForceNoIndex_{\rattleAtomIndexFirst}}
\newcommand*\rattleConstraintForceSecond      % second constraint force in pair
    {\rattleConstraintForceNoIndex_{\rattleAtomIndexSecond}}
\newcommand*\rattleConstraintForceBond        % constraint force along bond
        {\rattleConstraintForceNoIndex_{\rattleConstraintIndexSimple}}
\newcommand*\rattleConstraint
  {\rattleConstraintNoIndex_{\rattleConstraintIndexSimple}}
  \newcommand*\rattleConstraintDot
    {\dot{\rattleConstraintNoIndex}_{\rattleConstraintIndexSimple}}
\newcommand*\rattleLagrange
    {\rattleLagrangeNoIndex_{\rattleConstraintIndexSimple}}
\newcommand*\rattleLagrangeApprox
        {\rattleLagrangeApproxNoIndex_{\rattleConstraintIndexSimple}}
\newcommand*\rattleCorrectiveValue
  {\rattleCorrectiveValueNoIndex_{\rattleConstraintIndexSimple}}
%-----------------------------IterativeCorrection-------------------------------
\newcommand*\iteration[2]{#1^{(#2)}}            % iterative correction steps
\newcommand*\iterationThis[1]                   % arg^{i}
  {\iteration{#1}{\rattleCorrectionIndexThis}}
\newcommand*\iterationNext[1]                   % arg^{i+1}
  {\iteration{#1}{\rattleCorrectionIndexNext}}
%----------------------------------Vectors--------------------------------------
\newcommand*\rattleAtomPos{\rattlePos_{\rattleAtomIndex}}
\newcommand*\rattleAtomPosFirst{\rattlePos_{\rattleAtomIndexFirst}}
\newcommand*\rattleAtomPosSecond{\rattlePos_{\rattleAtomIndexSecond}}
\newcommand*\rattleAtomVel{\rattleVel_{\rattleAtomIndex}}
\newcommand*\rattleAtomVelFirst{\rattleVel_{\rattleAtomIndexFirst}}
\newcommand*\rattleAtomVelSecond{\rattleVel_{\rattleAtomIndexSecond}}
\newcommand*\rattlePosUnconstrained             % r^{(0)}
  {\iteration{\rattlePos}{0}}
\newcommand*\rattleVelUnconstrained             % v^{(0)}
  {\iteration{\rattleVel}{0}}
\newcommand*\rattleAtomPosUnconstrained         % r^{(0)}
  {\iteration{\rattleAtomPos}{0}}
\newcommand*\rattleAtomVelUnconstrained         % v^{(0)}
  {\iteration{\rattleAtomVel}{0}}
\newcommand*\rattleBondPos{\rattlePos_{\rattleConstraintIndexSimple}}
\newcommand*\rattleBondPosUnit{\hat{\rattlePos}_{\rattleConstraintIndexSimple}}
\newcommand*\rattleBondVel{\rattleVel_{\rattleConstraintIndexSimple}}
\newcommand*\rattleBondVelUnit{\hat{\rattleVel}_{\rattleConstraintIndexSimple}}
\newcommand*\rattleBondPosUnconstrained
  {\iteration{\rattleBondPos}{0}}
\newcommand*\rattleBondVelUnconstrained
  {\iteration{\rattleBondVel}{0}}
\newcommand*\rattleAtomPosUnconstrainedFirst
    {\iteration{\rattleAtomPosFirst}{0}}
\newcommand*\rattleAtomPosUnconstrainedSecond
    {\iteration{\rattleAtomPosSecond}{0}}
\newcommand*\rattleAtomVelUnconstrainedFirst
    {\iteration{\rattleAtomVelFirst}{0}}
\newcommand*\rattleAtomVelUnconstrainedSecond
    {\iteration{\rattleAtomVelSecond}{0}}
%---------------------------------Constants-------------------------------------
\newcommand*\rattleBondlength                   % fixed bond length
  {d_{\rattleConstraintIndexSimple}}
\newcommand*\rattleTol{\xi}                     % d-r/d tolerance
\newcommand*\rattleRRTol{\varepsilon}           % r(t) * r(t+dt) tolerance
\newcommand*\rattleRVTol{\rattleTol}            % v(t)/d tolerance
%-------------------------------------------------------------------------------

%-------------------------------------------------------------------------------
%\setlength{\parindent}{0pt}						          % do not indent
\numberwithin{equation}{section}                % number eqs within sections
\begin{document}
\lstset{language=Python}
%%%%%%%%%%%%%%%%%%%%%%%%%%%%%%%%%%%%%%%%%%%%%%%%%%%%%%%%%%%%%%%%%%%%%%%%%%%%%%%%
%					 	                        TITLE							                         %
%%%%%%%%%%%%%%%%%%%%%%%%%%%%%%%%%%%%%%%%%%%%%%%%%%%%%%%%%%%%%%%%%%%%%%%%%%%%%%%%
\title{Group Seminar (revised: \revisiondate)}
\thanks{Professor Stratt, Vale Cofer-Shabica, Yan Zhao, Andrew Ton, Mansheej Paul, Evan Coleman}
\author{Artur Avkhadiev}
\email{artur\_avkhadiev@brown.edu}
\affiliation{Department of Physics, Brown University, Providence RI 02912, USA}
\date{\seminardate}
%%%%%%%%%%%%%%%%%%%%%%%%%%%%%%%%%%%%%%%%%%%%%%%%%%%%%%%%%%%%%%%%%%%%%%%%%%%%%%%%
%					 	                       ABSTRACT							                       %
%%%%%%%%%%%%%%%%%%%%%%%%%%%%%%%%%%%%%%%%%%%%%%%%%%%%%%%%%%%%%%%%%%%%%%%%%%%%%%%%
\begin{abstract}
  This report consist of two parts:
  \begin{enumerate}
      \item A discussion of the constraint dynamics algorithm based on velocity Verlet numerical integration scheme, \rattle.
      \item A calculation of the kinetic energy of a freely-jointed polymer chain using the bond-vector representation in the CM frame of the molecule.
    \end{enumerate}
\end{abstract}
\maketitle
\tableofcontents						 % hide table of contents
%%%%%%%%%%%%%%%%%%%%%%%%%%%%%%%%%%%%%%%%%%%%%%%%%%%%%%%%%%%%%%%%%%%%%%%%%%%%%%%%
%					 	                        BODY							                         %
%%%%%%%%%%%%%%%%%%%%%%%%%%%%%%%%%%%%%%%%%%%%%%%%%%%%%%%%%%%%%%%%%%%%%%%%%%%%%%%%
% Add include statements below. All the text should be in the folders.
\documentclass[%
    reprint,
    superscriptaddress,
    %groupedaddress,
    %unsortedaddress,
    %runinaddress,
    %frontmatterverbose,
    %preprint,
    longbibliography,
    bibnotes,
    %showpacs,preprintnumbers,
    %nofootinbib,
    %nobibnotes,
    %bibnotes,
     amsmath,amssymb,
     %aps,
     aip,
     jcp,                                       % J. Chem. Phys
    %pra,
    %prb,
    %rmp,
    %prstab,
    %prstper,
    %floatfix,
    ]{revtex4-1}
%%%%%%%%%%%%%%%%%%%%%%%%%%%%%%%%%%%%%%%%%%%%%%%%%%%%%%%%%%%%%%%%%%%%%%%%%%%%%%%%
%					 	                       LIBRARIES							                     %
%%%%%%%%%%%%%%%%%%%%%%%%%%%%%%%%%%%%%%%%%%%%%%%%%%%%%%%%%%%%%%%%%%%%%%%%%%%%%%%%
% Add whatever libraries you need here. Add a description if you want
\usepackage{graphicx}			 % Include figure files
\usepackage{dcolumn}			 % Align table columns on decimal point
\usepackage{tcolorbox}		 % use tcolorbox environment to box equations
\usepackage[USenglish]{babel}
\usepackage[useregional]{datetime2}
\usepackage[utf8]{inputenc}
\usepackage[T1]{fontenc}
\usepackage{enumerate}
\usepackage{caption}
\usepackage{subcaption}
\usepackage{physics}
\usepackage{hyperref}
\usepackage{chemformula}
\usepackage{booktabs}
\usepackage{makecell}
\usepackage{fullpage}
\usepackage{natbib}
\usepackage{setspace}
\usepackage{amsmath}
\usepackage{bm}              % enchanced bold math symbols
\usepackage{amssymb}
\usepackage{listings} 			 % For code citations
\usepackage{amsfonts}
\usepackage{mathtools}
\usepackage{commath}
\usepackage{fancyhdr}
\usepackage{lastpage}
\usepackage{tikz}
\usepackage{graphicx}
	\graphicspath{ {images/} }
\usepackage{feynmp-auto}
\usepackage[toc,page]{appendix}
%%%%%%%%%%%%%%%%%%%%%%%%%%%%%%%%%%%%%%%%%%%%%%%%%%%%%%%%%%%%%%%%%%%%%%%%%%%%%%%%
%					 	                      NEW COMMANDS							                   %
%%%%%%%%%%%%%%%%%%%%%%%%%%%%%%%%%%%%%%%%%%%%%%%%%%%%%%%%%%%%%%%%%%%%%%%%%%%%%%%%
% Redefine commands if necessary
\renewcommand\vec{\mathbf}                      % use boldface for vectors
% New commands
%------------------------------------Date---------------------------------------
\newcommand{\seminardate}{\DTMdisplaydate{2017}{7}{25}{-1}}
\newcommand{\revisiondate}{\DTMdisplaydate{2017}{7}{26}{-1}}
%-------------------------------------------------------------------------------
\newcommand*\Del{\vec{\nabla}}                  % del operator
\newcommand*\diff{\mathop{}\!\mathrm{d}}		    % differential d
\newcommand*\Diff[1]{\mathop{}\!\mathrm{d^#1}}	% d^(...)
\newcommand*{\scalarnorm}[1]                    % scalar norm |*|
  {\left\lvert#1\right\rvert}
\newcommand*{\vectornorm}[1]                    % vector norm ||*||
  {\left\lVert#1\right\rVert}
\newcommand*\onehalf{\frac{1}{2}}               % fraction 1/2
%------------------------------------Time---------------------------------------
\newcommand*\dlt[1]{\delta #1}
\newcommand*\timestep{\dlt{t}}              % timestep dt
\newcommand*\halfstep                       % half timestep 1/2 dt
  {\onehalf \timestep}
\newcommand*\timeInHalfStep{\left(t + \halfstep\right)}
\newcommand*\timeInFullStep{\left(t + \timestep\right)}
%-------------------------------------------------------------------------------
\input{rattle/definitions}
\input{ke_polymer/definitions}
%-------------------------------------------------------------------------------
%\setlength{\parindent}{0pt}						          % do not indent
\numberwithin{equation}{section}                % number eqs within sections
\begin{document}
\lstset{language=Python}
%%%%%%%%%%%%%%%%%%%%%%%%%%%%%%%%%%%%%%%%%%%%%%%%%%%%%%%%%%%%%%%%%%%%%%%%%%%%%%%%
%					 	                        TITLE							                         %
%%%%%%%%%%%%%%%%%%%%%%%%%%%%%%%%%%%%%%%%%%%%%%%%%%%%%%%%%%%%%%%%%%%%%%%%%%%%%%%%
\title{Group Seminar (revised: \revisiondate)}
\thanks{Professor Stratt, Vale Cofer-Shabica, Yan Zhao, Andrew Ton, Mansheej Paul, Evan Coleman}
\author{Artur Avkhadiev}
\email{artur\_avkhadiev@brown.edu}
\affiliation{Department of Physics, Brown University, Providence RI 02912, USA}
\date{\seminardate}
%%%%%%%%%%%%%%%%%%%%%%%%%%%%%%%%%%%%%%%%%%%%%%%%%%%%%%%%%%%%%%%%%%%%%%%%%%%%%%%%
%					 	                       ABSTRACT							                       %
%%%%%%%%%%%%%%%%%%%%%%%%%%%%%%%%%%%%%%%%%%%%%%%%%%%%%%%%%%%%%%%%%%%%%%%%%%%%%%%%
\begin{abstract}
  This report consist of two parts:
  \begin{enumerate}
      \item A discussion of the constraint dynamics algorithm based on velocity Verlet numerical integration scheme, \rattle.
      \item A calculation of the kinetic energy of a freely-jointed polymer chain using the bond-vector representation in the CM frame of the molecule.
    \end{enumerate}
\end{abstract}
\maketitle
\tableofcontents						 % hide table of contents
%%%%%%%%%%%%%%%%%%%%%%%%%%%%%%%%%%%%%%%%%%%%%%%%%%%%%%%%%%%%%%%%%%%%%%%%%%%%%%%%
%					 	                        BODY							                         %
%%%%%%%%%%%%%%%%%%%%%%%%%%%%%%%%%%%%%%%%%%%%%%%%%%%%%%%%%%%%%%%%%%%%%%%%%%%%%%%%
% Add include statements below. All the text should be in the folders.
\input{rattle/main}
\input{ke_polymer/main}
%-------------------------------------------------------------------------------
%%%%%%%%%%%%%%%%%%%%%%%%%%%%%%%%%%%%%%%%%%%%%%%%%%%%%%%%%%%%%%%%%%%%%%%%%%%%%%%%
%					                        BIBLIOGRAPHY							                   %
%%%%%%%%%%%%%%%%%%%%%%%%%%%%%%%%%%%%%%%%%%%%%%%%%%%%%%%%%%%%%%%%%%%%%%%%%%%%%%%%
% The bibliography file is reference,bib. If you need to add a
% reference, make sure all the entries look good (e.g., ensure
% proper capitalization by putting {} around the letters to be
% capitalized. Reference the books by chapters!
\nocite{*}
\bibliographystyle{apalike}
\bibliography{reference}
%-----------------------------
%%%%%%%%%%%%%%%%%%%%%%%%%%%%%%%%%%%%%%%%%%%%%%%%%%%%%%%%%%%%%%%%%%%%%%%%%%%%%%%%
%					 	                      APPENDICES							                     %
%%%%%%%%%%%%%%%%%%%%%%%%%%%%%%%%%%%%%%%%%%%%%%%%%%%%%%%%%%%%%%%%%%%%%%%%%%%%%%%%
% Add all appendices in appendices/main.tex with include statements.
% see appendices/sub-example-app.tex for an example.
\onecolumngrid
\input{appendices/main}
%-------------------------------------------------------------------------------
\end{document}

\documentclass[%
    reprint,
    superscriptaddress,
    %groupedaddress,
    %unsortedaddress,
    %runinaddress,
    %frontmatterverbose,
    %preprint,
    longbibliography,
    bibnotes,
    %showpacs,preprintnumbers,
    %nofootinbib,
    %nobibnotes,
    %bibnotes,
     amsmath,amssymb,
     %aps,
     aip,
     jcp,                                       % J. Chem. Phys
    %pra,
    %prb,
    %rmp,
    %prstab,
    %prstper,
    %floatfix,
    ]{revtex4-1}
%%%%%%%%%%%%%%%%%%%%%%%%%%%%%%%%%%%%%%%%%%%%%%%%%%%%%%%%%%%%%%%%%%%%%%%%%%%%%%%%
%					 	                       LIBRARIES							                     %
%%%%%%%%%%%%%%%%%%%%%%%%%%%%%%%%%%%%%%%%%%%%%%%%%%%%%%%%%%%%%%%%%%%%%%%%%%%%%%%%
% Add whatever libraries you need here. Add a description if you want
\usepackage{graphicx}			 % Include figure files
\usepackage{dcolumn}			 % Align table columns on decimal point
\usepackage{tcolorbox}		 % use tcolorbox environment to box equations
\usepackage[USenglish]{babel}
\usepackage[useregional]{datetime2}
\usepackage[utf8]{inputenc}
\usepackage[T1]{fontenc}
\usepackage{enumerate}
\usepackage{caption}
\usepackage{subcaption}
\usepackage{physics}
\usepackage{hyperref}
\usepackage{chemformula}
\usepackage{booktabs}
\usepackage{makecell}
\usepackage{fullpage}
\usepackage{natbib}
\usepackage{setspace}
\usepackage{amsmath}
\usepackage{bm}              % enchanced bold math symbols
\usepackage{amssymb}
\usepackage{listings} 			 % For code citations
\usepackage{amsfonts}
\usepackage{mathtools}
\usepackage{commath}
\usepackage{fancyhdr}
\usepackage{lastpage}
\usepackage{tikz}
\usepackage{graphicx}
	\graphicspath{ {images/} }
\usepackage{feynmp-auto}
\usepackage[toc,page]{appendix}
%%%%%%%%%%%%%%%%%%%%%%%%%%%%%%%%%%%%%%%%%%%%%%%%%%%%%%%%%%%%%%%%%%%%%%%%%%%%%%%%
%					 	                      NEW COMMANDS							                   %
%%%%%%%%%%%%%%%%%%%%%%%%%%%%%%%%%%%%%%%%%%%%%%%%%%%%%%%%%%%%%%%%%%%%%%%%%%%%%%%%
% Redefine commands if necessary
\renewcommand\vec{\mathbf}                      % use boldface for vectors
% New commands
%------------------------------------Date---------------------------------------
\newcommand{\seminardate}{\DTMdisplaydate{2017}{7}{25}{-1}}
\newcommand{\revisiondate}{\DTMdisplaydate{2017}{7}{26}{-1}}
%-------------------------------------------------------------------------------
\newcommand*\Del{\vec{\nabla}}                  % del operator
\newcommand*\diff{\mathop{}\!\mathrm{d}}		    % differential d
\newcommand*\Diff[1]{\mathop{}\!\mathrm{d^#1}}	% d^(...)
\newcommand*{\scalarnorm}[1]                    % scalar norm |*|
  {\left\lvert#1\right\rvert}
\newcommand*{\vectornorm}[1]                    % vector norm ||*||
  {\left\lVert#1\right\rVert}
\newcommand*\onehalf{\frac{1}{2}}               % fraction 1/2
%------------------------------------Time---------------------------------------
\newcommand*\dlt[1]{\delta #1}
\newcommand*\timestep{\dlt{t}}              % timestep dt
\newcommand*\halfstep                       % half timestep 1/2 dt
  {\onehalf \timestep}
\newcommand*\timeInHalfStep{\left(t + \halfstep\right)}
\newcommand*\timeInFullStep{\left(t + \timestep\right)}
%-------------------------------------------------------------------------------
\input{rattle/definitions}
\input{ke_polymer/definitions}
%-------------------------------------------------------------------------------
%\setlength{\parindent}{0pt}						          % do not indent
\numberwithin{equation}{section}                % number eqs within sections
\begin{document}
\lstset{language=Python}
%%%%%%%%%%%%%%%%%%%%%%%%%%%%%%%%%%%%%%%%%%%%%%%%%%%%%%%%%%%%%%%%%%%%%%%%%%%%%%%%
%					 	                        TITLE							                         %
%%%%%%%%%%%%%%%%%%%%%%%%%%%%%%%%%%%%%%%%%%%%%%%%%%%%%%%%%%%%%%%%%%%%%%%%%%%%%%%%
\title{Group Seminar (revised: \revisiondate)}
\thanks{Professor Stratt, Vale Cofer-Shabica, Yan Zhao, Andrew Ton, Mansheej Paul, Evan Coleman}
\author{Artur Avkhadiev}
\email{artur\_avkhadiev@brown.edu}
\affiliation{Department of Physics, Brown University, Providence RI 02912, USA}
\date{\seminardate}
%%%%%%%%%%%%%%%%%%%%%%%%%%%%%%%%%%%%%%%%%%%%%%%%%%%%%%%%%%%%%%%%%%%%%%%%%%%%%%%%
%					 	                       ABSTRACT							                       %
%%%%%%%%%%%%%%%%%%%%%%%%%%%%%%%%%%%%%%%%%%%%%%%%%%%%%%%%%%%%%%%%%%%%%%%%%%%%%%%%
\begin{abstract}
  This report consist of two parts:
  \begin{enumerate}
      \item A discussion of the constraint dynamics algorithm based on velocity Verlet numerical integration scheme, \rattle.
      \item A calculation of the kinetic energy of a freely-jointed polymer chain using the bond-vector representation in the CM frame of the molecule.
    \end{enumerate}
\end{abstract}
\maketitle
\tableofcontents						 % hide table of contents
%%%%%%%%%%%%%%%%%%%%%%%%%%%%%%%%%%%%%%%%%%%%%%%%%%%%%%%%%%%%%%%%%%%%%%%%%%%%%%%%
%					 	                        BODY							                         %
%%%%%%%%%%%%%%%%%%%%%%%%%%%%%%%%%%%%%%%%%%%%%%%%%%%%%%%%%%%%%%%%%%%%%%%%%%%%%%%%
% Add include statements below. All the text should be in the folders.
\input{rattle/main}
\input{ke_polymer/main}
%-------------------------------------------------------------------------------
%%%%%%%%%%%%%%%%%%%%%%%%%%%%%%%%%%%%%%%%%%%%%%%%%%%%%%%%%%%%%%%%%%%%%%%%%%%%%%%%
%					                        BIBLIOGRAPHY							                   %
%%%%%%%%%%%%%%%%%%%%%%%%%%%%%%%%%%%%%%%%%%%%%%%%%%%%%%%%%%%%%%%%%%%%%%%%%%%%%%%%
% The bibliography file is reference,bib. If you need to add a
% reference, make sure all the entries look good (e.g., ensure
% proper capitalization by putting {} around the letters to be
% capitalized. Reference the books by chapters!
\nocite{*}
\bibliographystyle{apalike}
\bibliography{reference}
%-----------------------------
%%%%%%%%%%%%%%%%%%%%%%%%%%%%%%%%%%%%%%%%%%%%%%%%%%%%%%%%%%%%%%%%%%%%%%%%%%%%%%%%
%					 	                      APPENDICES							                     %
%%%%%%%%%%%%%%%%%%%%%%%%%%%%%%%%%%%%%%%%%%%%%%%%%%%%%%%%%%%%%%%%%%%%%%%%%%%%%%%%
% Add all appendices in appendices/main.tex with include statements.
% see appendices/sub-example-app.tex for an example.
\onecolumngrid
\input{appendices/main}
%-------------------------------------------------------------------------------
\end{document}

%-------------------------------------------------------------------------------
%%%%%%%%%%%%%%%%%%%%%%%%%%%%%%%%%%%%%%%%%%%%%%%%%%%%%%%%%%%%%%%%%%%%%%%%%%%%%%%%
%					                        BIBLIOGRAPHY							                   %
%%%%%%%%%%%%%%%%%%%%%%%%%%%%%%%%%%%%%%%%%%%%%%%%%%%%%%%%%%%%%%%%%%%%%%%%%%%%%%%%
% The bibliography file is reference,bib. If you need to add a
% reference, make sure all the entries look good (e.g., ensure
% proper capitalization by putting {} around the letters to be
% capitalized. Reference the books by chapters!
\nocite{*}
\bibliographystyle{apalike}
\bibliography{reference}
%-----------------------------
%%%%%%%%%%%%%%%%%%%%%%%%%%%%%%%%%%%%%%%%%%%%%%%%%%%%%%%%%%%%%%%%%%%%%%%%%%%%%%%%
%					 	                      APPENDICES							                     %
%%%%%%%%%%%%%%%%%%%%%%%%%%%%%%%%%%%%%%%%%%%%%%%%%%%%%%%%%%%%%%%%%%%%%%%%%%%%%%%%
% Add all appendices in appendices/main.tex with include statements.
% see appendices/sub-example-app.tex for an example.
\onecolumngrid
\documentclass[%
    reprint,
    superscriptaddress,
    %groupedaddress,
    %unsortedaddress,
    %runinaddress,
    %frontmatterverbose,
    %preprint,
    longbibliography,
    bibnotes,
    %showpacs,preprintnumbers,
    %nofootinbib,
    %nobibnotes,
    %bibnotes,
     amsmath,amssymb,
     %aps,
     aip,
     jcp,                                       % J. Chem. Phys
    %pra,
    %prb,
    %rmp,
    %prstab,
    %prstper,
    %floatfix,
    ]{revtex4-1}
%%%%%%%%%%%%%%%%%%%%%%%%%%%%%%%%%%%%%%%%%%%%%%%%%%%%%%%%%%%%%%%%%%%%%%%%%%%%%%%%
%					 	                       LIBRARIES							                     %
%%%%%%%%%%%%%%%%%%%%%%%%%%%%%%%%%%%%%%%%%%%%%%%%%%%%%%%%%%%%%%%%%%%%%%%%%%%%%%%%
% Add whatever libraries you need here. Add a description if you want
\usepackage{graphicx}			 % Include figure files
\usepackage{dcolumn}			 % Align table columns on decimal point
\usepackage{tcolorbox}		 % use tcolorbox environment to box equations
\usepackage[USenglish]{babel}
\usepackage[useregional]{datetime2}
\usepackage[utf8]{inputenc}
\usepackage[T1]{fontenc}
\usepackage{enumerate}
\usepackage{caption}
\usepackage{subcaption}
\usepackage{physics}
\usepackage{hyperref}
\usepackage{chemformula}
\usepackage{booktabs}
\usepackage{makecell}
\usepackage{fullpage}
\usepackage{natbib}
\usepackage{setspace}
\usepackage{amsmath}
\usepackage{bm}              % enchanced bold math symbols
\usepackage{amssymb}
\usepackage{listings} 			 % For code citations
\usepackage{amsfonts}
\usepackage{mathtools}
\usepackage{commath}
\usepackage{fancyhdr}
\usepackage{lastpage}
\usepackage{tikz}
\usepackage{graphicx}
	\graphicspath{ {images/} }
\usepackage{feynmp-auto}
\usepackage[toc,page]{appendix}
%%%%%%%%%%%%%%%%%%%%%%%%%%%%%%%%%%%%%%%%%%%%%%%%%%%%%%%%%%%%%%%%%%%%%%%%%%%%%%%%
%					 	                      NEW COMMANDS							                   %
%%%%%%%%%%%%%%%%%%%%%%%%%%%%%%%%%%%%%%%%%%%%%%%%%%%%%%%%%%%%%%%%%%%%%%%%%%%%%%%%
% Redefine commands if necessary
\renewcommand\vec{\mathbf}                      % use boldface for vectors
% New commands
%------------------------------------Date---------------------------------------
\newcommand{\seminardate}{\DTMdisplaydate{2017}{7}{25}{-1}}
\newcommand{\revisiondate}{\DTMdisplaydate{2017}{7}{26}{-1}}
%-------------------------------------------------------------------------------
\newcommand*\Del{\vec{\nabla}}                  % del operator
\newcommand*\diff{\mathop{}\!\mathrm{d}}		    % differential d
\newcommand*\Diff[1]{\mathop{}\!\mathrm{d^#1}}	% d^(...)
\newcommand*{\scalarnorm}[1]                    % scalar norm |*|
  {\left\lvert#1\right\rvert}
\newcommand*{\vectornorm}[1]                    % vector norm ||*||
  {\left\lVert#1\right\rVert}
\newcommand*\onehalf{\frac{1}{2}}               % fraction 1/2
%------------------------------------Time---------------------------------------
\newcommand*\dlt[1]{\delta #1}
\newcommand*\timestep{\dlt{t}}              % timestep dt
\newcommand*\halfstep                       % half timestep 1/2 dt
  {\onehalf \timestep}
\newcommand*\timeInHalfStep{\left(t + \halfstep\right)}
\newcommand*\timeInFullStep{\left(t + \timestep\right)}
%-------------------------------------------------------------------------------
\input{rattle/definitions}
\input{ke_polymer/definitions}
%-------------------------------------------------------------------------------
%\setlength{\parindent}{0pt}						          % do not indent
\numberwithin{equation}{section}                % number eqs within sections
\begin{document}
\lstset{language=Python}
%%%%%%%%%%%%%%%%%%%%%%%%%%%%%%%%%%%%%%%%%%%%%%%%%%%%%%%%%%%%%%%%%%%%%%%%%%%%%%%%
%					 	                        TITLE							                         %
%%%%%%%%%%%%%%%%%%%%%%%%%%%%%%%%%%%%%%%%%%%%%%%%%%%%%%%%%%%%%%%%%%%%%%%%%%%%%%%%
\title{Group Seminar (revised: \revisiondate)}
\thanks{Professor Stratt, Vale Cofer-Shabica, Yan Zhao, Andrew Ton, Mansheej Paul, Evan Coleman}
\author{Artur Avkhadiev}
\email{artur\_avkhadiev@brown.edu}
\affiliation{Department of Physics, Brown University, Providence RI 02912, USA}
\date{\seminardate}
%%%%%%%%%%%%%%%%%%%%%%%%%%%%%%%%%%%%%%%%%%%%%%%%%%%%%%%%%%%%%%%%%%%%%%%%%%%%%%%%
%					 	                       ABSTRACT							                       %
%%%%%%%%%%%%%%%%%%%%%%%%%%%%%%%%%%%%%%%%%%%%%%%%%%%%%%%%%%%%%%%%%%%%%%%%%%%%%%%%
\begin{abstract}
  This report consist of two parts:
  \begin{enumerate}
      \item A discussion of the constraint dynamics algorithm based on velocity Verlet numerical integration scheme, \rattle.
      \item A calculation of the kinetic energy of a freely-jointed polymer chain using the bond-vector representation in the CM frame of the molecule.
    \end{enumerate}
\end{abstract}
\maketitle
\tableofcontents						 % hide table of contents
%%%%%%%%%%%%%%%%%%%%%%%%%%%%%%%%%%%%%%%%%%%%%%%%%%%%%%%%%%%%%%%%%%%%%%%%%%%%%%%%
%					 	                        BODY							                         %
%%%%%%%%%%%%%%%%%%%%%%%%%%%%%%%%%%%%%%%%%%%%%%%%%%%%%%%%%%%%%%%%%%%%%%%%%%%%%%%%
% Add include statements below. All the text should be in the folders.
\input{rattle/main}
\input{ke_polymer/main}
%-------------------------------------------------------------------------------
%%%%%%%%%%%%%%%%%%%%%%%%%%%%%%%%%%%%%%%%%%%%%%%%%%%%%%%%%%%%%%%%%%%%%%%%%%%%%%%%
%					                        BIBLIOGRAPHY							                   %
%%%%%%%%%%%%%%%%%%%%%%%%%%%%%%%%%%%%%%%%%%%%%%%%%%%%%%%%%%%%%%%%%%%%%%%%%%%%%%%%
% The bibliography file is reference,bib. If you need to add a
% reference, make sure all the entries look good (e.g., ensure
% proper capitalization by putting {} around the letters to be
% capitalized. Reference the books by chapters!
\nocite{*}
\bibliographystyle{apalike}
\bibliography{reference}
%-----------------------------
%%%%%%%%%%%%%%%%%%%%%%%%%%%%%%%%%%%%%%%%%%%%%%%%%%%%%%%%%%%%%%%%%%%%%%%%%%%%%%%%
%					 	                      APPENDICES							                     %
%%%%%%%%%%%%%%%%%%%%%%%%%%%%%%%%%%%%%%%%%%%%%%%%%%%%%%%%%%%%%%%%%%%%%%%%%%%%%%%%
% Add all appendices in appendices/main.tex with include statements.
% see appendices/sub-example-app.tex for an example.
\onecolumngrid
\input{appendices/main}
%-------------------------------------------------------------------------------
\end{document}

%-------------------------------------------------------------------------------
\end{document}

%-------------------------------------------------------------------------------
%%%%%%%%%%%%%%%%%%%%%%%%%%%%%%%%%%%%%%%%%%%%%%%%%%%%%%%%%%%%%%%%%%%%%%%%%%%%%%%%
%					                        BIBLIOGRAPHY							                   %
%%%%%%%%%%%%%%%%%%%%%%%%%%%%%%%%%%%%%%%%%%%%%%%%%%%%%%%%%%%%%%%%%%%%%%%%%%%%%%%%
% The bibliography file is reference,bib. If you need to add a
% reference, make sure all the entries look good (e.g., ensure
% proper capitalization by putting {} around the letters to be
% capitalized. Reference the books by chapters!
\nocite{*}
\bibliographystyle{apalike}
\bibliography{reference}
%-----------------------------
%%%%%%%%%%%%%%%%%%%%%%%%%%%%%%%%%%%%%%%%%%%%%%%%%%%%%%%%%%%%%%%%%%%%%%%%%%%%%%%%
%					 	                      APPENDICES							                     %
%%%%%%%%%%%%%%%%%%%%%%%%%%%%%%%%%%%%%%%%%%%%%%%%%%%%%%%%%%%%%%%%%%%%%%%%%%%%%%%%
% Add all appendices in appendices/main.tex with include statements.
% see appendices/sub-example-app.tex for an example.
\onecolumngrid
\documentclass[%
    reprint,
    superscriptaddress,
    %groupedaddress,
    %unsortedaddress,
    %runinaddress,
    %frontmatterverbose,
    %preprint,
    longbibliography,
    bibnotes,
    %showpacs,preprintnumbers,
    %nofootinbib,
    %nobibnotes,
    %bibnotes,
     amsmath,amssymb,
     %aps,
     aip,
     jcp,                                       % J. Chem. Phys
    %pra,
    %prb,
    %rmp,
    %prstab,
    %prstper,
    %floatfix,
    ]{revtex4-1}
%%%%%%%%%%%%%%%%%%%%%%%%%%%%%%%%%%%%%%%%%%%%%%%%%%%%%%%%%%%%%%%%%%%%%%%%%%%%%%%%
%					 	                       LIBRARIES							                     %
%%%%%%%%%%%%%%%%%%%%%%%%%%%%%%%%%%%%%%%%%%%%%%%%%%%%%%%%%%%%%%%%%%%%%%%%%%%%%%%%
% Add whatever libraries you need here. Add a description if you want
\usepackage{graphicx}			 % Include figure files
\usepackage{dcolumn}			 % Align table columns on decimal point
\usepackage{tcolorbox}		 % use tcolorbox environment to box equations
\usepackage[USenglish]{babel}
\usepackage[useregional]{datetime2}
\usepackage[utf8]{inputenc}
\usepackage[T1]{fontenc}
\usepackage{enumerate}
\usepackage{caption}
\usepackage{subcaption}
\usepackage{physics}
\usepackage{hyperref}
\usepackage{chemformula}
\usepackage{booktabs}
\usepackage{makecell}
\usepackage{fullpage}
\usepackage{natbib}
\usepackage{setspace}
\usepackage{amsmath}
\usepackage{bm}              % enchanced bold math symbols
\usepackage{amssymb}
\usepackage{listings} 			 % For code citations
\usepackage{amsfonts}
\usepackage{mathtools}
\usepackage{commath}
\usepackage{fancyhdr}
\usepackage{lastpage}
\usepackage{tikz}
\usepackage{graphicx}
	\graphicspath{ {images/} }
\usepackage{feynmp-auto}
\usepackage[toc,page]{appendix}
%%%%%%%%%%%%%%%%%%%%%%%%%%%%%%%%%%%%%%%%%%%%%%%%%%%%%%%%%%%%%%%%%%%%%%%%%%%%%%%%
%					 	                      NEW COMMANDS							                   %
%%%%%%%%%%%%%%%%%%%%%%%%%%%%%%%%%%%%%%%%%%%%%%%%%%%%%%%%%%%%%%%%%%%%%%%%%%%%%%%%
% Redefine commands if necessary
\renewcommand\vec{\mathbf}                      % use boldface for vectors
% New commands
%------------------------------------Date---------------------------------------
\newcommand{\seminardate}{\DTMdisplaydate{2017}{7}{25}{-1}}
\newcommand{\revisiondate}{\DTMdisplaydate{2017}{7}{26}{-1}}
%-------------------------------------------------------------------------------
\newcommand*\Del{\vec{\nabla}}                  % del operator
\newcommand*\diff{\mathop{}\!\mathrm{d}}		    % differential d
\newcommand*\Diff[1]{\mathop{}\!\mathrm{d^#1}}	% d^(...)
\newcommand*{\scalarnorm}[1]                    % scalar norm |*|
  {\left\lvert#1\right\rvert}
\newcommand*{\vectornorm}[1]                    % vector norm ||*||
  {\left\lVert#1\right\rVert}
\newcommand*\onehalf{\frac{1}{2}}               % fraction 1/2
%------------------------------------Time---------------------------------------
\newcommand*\dlt[1]{\delta #1}
\newcommand*\timestep{\dlt{t}}              % timestep dt
\newcommand*\halfstep                       % half timestep 1/2 dt
  {\onehalf \timestep}
\newcommand*\timeInHalfStep{\left(t + \halfstep\right)}
\newcommand*\timeInFullStep{\left(t + \timestep\right)}
%-------------------------------------------------------------------------------
%%%%%%%%%%%%%%%%%%%%%%%%%%%%%%%%%%%%%%%%%%%%%%%%%%%%%%%%%%%%%%%%%%%%%%%%%%%%%%%%
%					 	                    DEFINITIONS								                     %
%%%%%%%%%%%%%%%%%%%%%%%%%%%%%%%%%%%%%%%%%%%%%%%%%%%%%%%%%%%%%%%%%%%%%%%%%%%%%%%%
%-------------------------------------------------------------------------------
\newcommand*\rattle{\textsf{RATTLE} }       % display name of algorithm
%-------------------------------------------------------------------------------
\newcommand*\rattlePos{\vec{r}}             % position vector
\newcommand*\rattleVel{\vec{v}}             % velocity vector
\newcommand*\rattleAcc{\vec{a}}             % acceleration vector
%-------------------------------------------------------------------------------
\newcommand*\rattleMassNoIndex{m}
\newcommand*\rattleForceNoIndex{\vec{F}}
\newcommand*\rattleConstraintForceNoIndex{\vec{G}}
\newcommand*\rattleConstraintNoIndex{\sigma}
\newcommand*\rattleLagrangeNoIndex{\lambda}
\newcommand*\rattleLagrangeApproxNoIndex{\gamma}
\newcommand*\rattleCorrectiveValueNoIndex{g}
%---------------------Definitions for General Formulation ----------------------
\input{rattle/general/definitions}
%-----------------------------------Indexing------------------------------------
\newcommand*\rattleAtomIndex{a}               % solo atom index
\newcommand*\rattleAtomIndexFirst{a}          % first index in atomic pair
\newcommand*\rattleAtomIndexSecond{b}         % second index in atomic pair
\newcommand*\rattleNAtoms{N}                  % number of atoms
\newcommand*\rattleConstraintSet{K}           % set of constraints
\newcommand*\rattleConstraintIndex            % index for constraints
  {\left(
    \rattleAtomIndexFirst,\,\rattleAtomIndexSecond
  \right) \in \rattleConstraintSet}
\newcommand*\rattleConstraintIndexSimple      % simplified index for constraints
  {\rattleAtomIndexFirst\rattleAtomIndexSecond}
\newcommand*\rattleCorrectionIndex{i}         % index for iterative correction
\newcommand*\rattleCorrectionIndexThis        % current for iterative correction
  {\rattleCorrectionIndex}
\newcommand*\rattleCorrectionIndexNext        % next for iterative correction
    {\rattleCorrectionIndex + 1}
\newcommand*\rattleCorrectionIndexLast{m}     % last step of correction
\newcommand*\rattleNCorrections{M}            % number of iterative corrections
%-----------------------------------Dynamics------------------------------------
\newcommand*\rattleMass                       % solo atomic mass
  {\rattleMassNoIndex_{\rattleAtomIndex}}
\newcommand*\rattleMassFirst                  % first atomic mass in pair
    {\rattleMassNoIndex_{\rattleAtomIndexFirst}}
\newcommand*\rattleMassSecond                 % second atomic mass in pair
    {\rattleMassNoIndex_{\rattleAtomIndexSecond}}
\newcommand*\rattleForce
  {\rattleForceNoIndex_{\rattleAtomIndex}}
\newcommand*\rattleConstraintForce            % solo constraint force
  {\rattleConstraintForceNoIndex_{\rattleAtomIndex}}
\newcommand*\rattleConstraintForceFirst       % first constraint force in pair
    {\rattleConstraintForceNoIndex_{\rattleAtomIndexFirst}}
\newcommand*\rattleConstraintForceSecond      % second constraint force in pair
    {\rattleConstraintForceNoIndex_{\rattleAtomIndexSecond}}
\newcommand*\rattleConstraintForceBond        % constraint force along bond
        {\rattleConstraintForceNoIndex_{\rattleConstraintIndexSimple}}
\newcommand*\rattleConstraint
  {\rattleConstraintNoIndex_{\rattleConstraintIndexSimple}}
  \newcommand*\rattleConstraintDot
    {\dot{\rattleConstraintNoIndex}_{\rattleConstraintIndexSimple}}
\newcommand*\rattleLagrange
    {\rattleLagrangeNoIndex_{\rattleConstraintIndexSimple}}
\newcommand*\rattleLagrangeApprox
        {\rattleLagrangeApproxNoIndex_{\rattleConstraintIndexSimple}}
\newcommand*\rattleCorrectiveValue
  {\rattleCorrectiveValueNoIndex_{\rattleConstraintIndexSimple}}
%-----------------------------IterativeCorrection-------------------------------
\newcommand*\iteration[2]{#1^{(#2)}}            % iterative correction steps
\newcommand*\iterationThis[1]                   % arg^{i}
  {\iteration{#1}{\rattleCorrectionIndexThis}}
\newcommand*\iterationNext[1]                   % arg^{i+1}
  {\iteration{#1}{\rattleCorrectionIndexNext}}
%----------------------------------Vectors--------------------------------------
\newcommand*\rattleAtomPos{\rattlePos_{\rattleAtomIndex}}
\newcommand*\rattleAtomPosFirst{\rattlePos_{\rattleAtomIndexFirst}}
\newcommand*\rattleAtomPosSecond{\rattlePos_{\rattleAtomIndexSecond}}
\newcommand*\rattleAtomVel{\rattleVel_{\rattleAtomIndex}}
\newcommand*\rattleAtomVelFirst{\rattleVel_{\rattleAtomIndexFirst}}
\newcommand*\rattleAtomVelSecond{\rattleVel_{\rattleAtomIndexSecond}}
\newcommand*\rattlePosUnconstrained             % r^{(0)}
  {\iteration{\rattlePos}{0}}
\newcommand*\rattleVelUnconstrained             % v^{(0)}
  {\iteration{\rattleVel}{0}}
\newcommand*\rattleAtomPosUnconstrained         % r^{(0)}
  {\iteration{\rattleAtomPos}{0}}
\newcommand*\rattleAtomVelUnconstrained         % v^{(0)}
  {\iteration{\rattleAtomVel}{0}}
\newcommand*\rattleBondPos{\rattlePos_{\rattleConstraintIndexSimple}}
\newcommand*\rattleBondPosUnit{\hat{\rattlePos}_{\rattleConstraintIndexSimple}}
\newcommand*\rattleBondVel{\rattleVel_{\rattleConstraintIndexSimple}}
\newcommand*\rattleBondVelUnit{\hat{\rattleVel}_{\rattleConstraintIndexSimple}}
\newcommand*\rattleBondPosUnconstrained
  {\iteration{\rattleBondPos}{0}}
\newcommand*\rattleBondVelUnconstrained
  {\iteration{\rattleBondVel}{0}}
\newcommand*\rattleAtomPosUnconstrainedFirst
    {\iteration{\rattleAtomPosFirst}{0}}
\newcommand*\rattleAtomPosUnconstrainedSecond
    {\iteration{\rattleAtomPosSecond}{0}}
\newcommand*\rattleAtomVelUnconstrainedFirst
    {\iteration{\rattleAtomVelFirst}{0}}
\newcommand*\rattleAtomVelUnconstrainedSecond
    {\iteration{\rattleAtomVelSecond}{0}}
%---------------------------------Constants-------------------------------------
\newcommand*\rattleBondlength                   % fixed bond length
  {d_{\rattleConstraintIndexSimple}}
\newcommand*\rattleTol{\xi}                     % d-r/d tolerance
\newcommand*\rattleRRTol{\varepsilon}           % r(t) * r(t+dt) tolerance
\newcommand*\rattleRVTol{\rattleTol}            % v(t)/d tolerance
%-------------------------------------------------------------------------------

%%%%%%%%%%%%%%%%%%%%%%%%%%%%%%%%%%%%%%%%%%%%%%%%%%%%%%%%%%%%%%%%%%%%%%%%%%%%%%%%
%					 	                    DEFINITIONS								                     %
%%%%%%%%%%%%%%%%%%%%%%%%%%%%%%%%%%%%%%%%%%%%%%%%%%%%%%%%%%%%%%%%%%%%%%%%%%%%%%%%
%-------------------------------------------------------------------------------
\newcommand*\rattle{\textsf{RATTLE} }       % display name of algorithm
%-------------------------------------------------------------------------------
\newcommand*\rattlePos{\vec{r}}             % position vector
\newcommand*\rattleVel{\vec{v}}             % velocity vector
\newcommand*\rattleAcc{\vec{a}}             % acceleration vector
%-------------------------------------------------------------------------------
\newcommand*\rattleMassNoIndex{m}
\newcommand*\rattleForceNoIndex{\vec{F}}
\newcommand*\rattleConstraintForceNoIndex{\vec{G}}
\newcommand*\rattleConstraintNoIndex{\sigma}
\newcommand*\rattleLagrangeNoIndex{\lambda}
\newcommand*\rattleLagrangeApproxNoIndex{\gamma}
\newcommand*\rattleCorrectiveValueNoIndex{g}
%---------------------Definitions for General Formulation ----------------------
\input{rattle/general/definitions}
%-----------------------------------Indexing------------------------------------
\newcommand*\rattleAtomIndex{a}               % solo atom index
\newcommand*\rattleAtomIndexFirst{a}          % first index in atomic pair
\newcommand*\rattleAtomIndexSecond{b}         % second index in atomic pair
\newcommand*\rattleNAtoms{N}                  % number of atoms
\newcommand*\rattleConstraintSet{K}           % set of constraints
\newcommand*\rattleConstraintIndex            % index for constraints
  {\left(
    \rattleAtomIndexFirst,\,\rattleAtomIndexSecond
  \right) \in \rattleConstraintSet}
\newcommand*\rattleConstraintIndexSimple      % simplified index for constraints
  {\rattleAtomIndexFirst\rattleAtomIndexSecond}
\newcommand*\rattleCorrectionIndex{i}         % index for iterative correction
\newcommand*\rattleCorrectionIndexThis        % current for iterative correction
  {\rattleCorrectionIndex}
\newcommand*\rattleCorrectionIndexNext        % next for iterative correction
    {\rattleCorrectionIndex + 1}
\newcommand*\rattleCorrectionIndexLast{m}     % last step of correction
\newcommand*\rattleNCorrections{M}            % number of iterative corrections
%-----------------------------------Dynamics------------------------------------
\newcommand*\rattleMass                       % solo atomic mass
  {\rattleMassNoIndex_{\rattleAtomIndex}}
\newcommand*\rattleMassFirst                  % first atomic mass in pair
    {\rattleMassNoIndex_{\rattleAtomIndexFirst}}
\newcommand*\rattleMassSecond                 % second atomic mass in pair
    {\rattleMassNoIndex_{\rattleAtomIndexSecond}}
\newcommand*\rattleForce
  {\rattleForceNoIndex_{\rattleAtomIndex}}
\newcommand*\rattleConstraintForce            % solo constraint force
  {\rattleConstraintForceNoIndex_{\rattleAtomIndex}}
\newcommand*\rattleConstraintForceFirst       % first constraint force in pair
    {\rattleConstraintForceNoIndex_{\rattleAtomIndexFirst}}
\newcommand*\rattleConstraintForceSecond      % second constraint force in pair
    {\rattleConstraintForceNoIndex_{\rattleAtomIndexSecond}}
\newcommand*\rattleConstraintForceBond        % constraint force along bond
        {\rattleConstraintForceNoIndex_{\rattleConstraintIndexSimple}}
\newcommand*\rattleConstraint
  {\rattleConstraintNoIndex_{\rattleConstraintIndexSimple}}
  \newcommand*\rattleConstraintDot
    {\dot{\rattleConstraintNoIndex}_{\rattleConstraintIndexSimple}}
\newcommand*\rattleLagrange
    {\rattleLagrangeNoIndex_{\rattleConstraintIndexSimple}}
\newcommand*\rattleLagrangeApprox
        {\rattleLagrangeApproxNoIndex_{\rattleConstraintIndexSimple}}
\newcommand*\rattleCorrectiveValue
  {\rattleCorrectiveValueNoIndex_{\rattleConstraintIndexSimple}}
%-----------------------------IterativeCorrection-------------------------------
\newcommand*\iteration[2]{#1^{(#2)}}            % iterative correction steps
\newcommand*\iterationThis[1]                   % arg^{i}
  {\iteration{#1}{\rattleCorrectionIndexThis}}
\newcommand*\iterationNext[1]                   % arg^{i+1}
  {\iteration{#1}{\rattleCorrectionIndexNext}}
%----------------------------------Vectors--------------------------------------
\newcommand*\rattleAtomPos{\rattlePos_{\rattleAtomIndex}}
\newcommand*\rattleAtomPosFirst{\rattlePos_{\rattleAtomIndexFirst}}
\newcommand*\rattleAtomPosSecond{\rattlePos_{\rattleAtomIndexSecond}}
\newcommand*\rattleAtomVel{\rattleVel_{\rattleAtomIndex}}
\newcommand*\rattleAtomVelFirst{\rattleVel_{\rattleAtomIndexFirst}}
\newcommand*\rattleAtomVelSecond{\rattleVel_{\rattleAtomIndexSecond}}
\newcommand*\rattlePosUnconstrained             % r^{(0)}
  {\iteration{\rattlePos}{0}}
\newcommand*\rattleVelUnconstrained             % v^{(0)}
  {\iteration{\rattleVel}{0}}
\newcommand*\rattleAtomPosUnconstrained         % r^{(0)}
  {\iteration{\rattleAtomPos}{0}}
\newcommand*\rattleAtomVelUnconstrained         % v^{(0)}
  {\iteration{\rattleAtomVel}{0}}
\newcommand*\rattleBondPos{\rattlePos_{\rattleConstraintIndexSimple}}
\newcommand*\rattleBondPosUnit{\hat{\rattlePos}_{\rattleConstraintIndexSimple}}
\newcommand*\rattleBondVel{\rattleVel_{\rattleConstraintIndexSimple}}
\newcommand*\rattleBondVelUnit{\hat{\rattleVel}_{\rattleConstraintIndexSimple}}
\newcommand*\rattleBondPosUnconstrained
  {\iteration{\rattleBondPos}{0}}
\newcommand*\rattleBondVelUnconstrained
  {\iteration{\rattleBondVel}{0}}
\newcommand*\rattleAtomPosUnconstrainedFirst
    {\iteration{\rattleAtomPosFirst}{0}}
\newcommand*\rattleAtomPosUnconstrainedSecond
    {\iteration{\rattleAtomPosSecond}{0}}
\newcommand*\rattleAtomVelUnconstrainedFirst
    {\iteration{\rattleAtomVelFirst}{0}}
\newcommand*\rattleAtomVelUnconstrainedSecond
    {\iteration{\rattleAtomVelSecond}{0}}
%---------------------------------Constants-------------------------------------
\newcommand*\rattleBondlength                   % fixed bond length
  {d_{\rattleConstraintIndexSimple}}
\newcommand*\rattleTol{\xi}                     % d-r/d tolerance
\newcommand*\rattleRRTol{\varepsilon}           % r(t) * r(t+dt) tolerance
\newcommand*\rattleRVTol{\rattleTol}            % v(t)/d tolerance
%-------------------------------------------------------------------------------

%-------------------------------------------------------------------------------
%\setlength{\parindent}{0pt}						          % do not indent
\numberwithin{equation}{section}                % number eqs within sections
\begin{document}
\lstset{language=Python}
%%%%%%%%%%%%%%%%%%%%%%%%%%%%%%%%%%%%%%%%%%%%%%%%%%%%%%%%%%%%%%%%%%%%%%%%%%%%%%%%
%					 	                        TITLE							                         %
%%%%%%%%%%%%%%%%%%%%%%%%%%%%%%%%%%%%%%%%%%%%%%%%%%%%%%%%%%%%%%%%%%%%%%%%%%%%%%%%
\title{Group Seminar (revised: \revisiondate)}
\thanks{Professor Stratt, Vale Cofer-Shabica, Yan Zhao, Andrew Ton, Mansheej Paul, Evan Coleman}
\author{Artur Avkhadiev}
\email{artur\_avkhadiev@brown.edu}
\affiliation{Department of Physics, Brown University, Providence RI 02912, USA}
\date{\seminardate}
%%%%%%%%%%%%%%%%%%%%%%%%%%%%%%%%%%%%%%%%%%%%%%%%%%%%%%%%%%%%%%%%%%%%%%%%%%%%%%%%
%					 	                       ABSTRACT							                       %
%%%%%%%%%%%%%%%%%%%%%%%%%%%%%%%%%%%%%%%%%%%%%%%%%%%%%%%%%%%%%%%%%%%%%%%%%%%%%%%%
\begin{abstract}
  This report consist of two parts:
  \begin{enumerate}
      \item A discussion of the constraint dynamics algorithm based on velocity Verlet numerical integration scheme, \rattle.
      \item A calculation of the kinetic energy of a freely-jointed polymer chain using the bond-vector representation in the CM frame of the molecule.
    \end{enumerate}
\end{abstract}
\maketitle
\tableofcontents						 % hide table of contents
%%%%%%%%%%%%%%%%%%%%%%%%%%%%%%%%%%%%%%%%%%%%%%%%%%%%%%%%%%%%%%%%%%%%%%%%%%%%%%%%
%					 	                        BODY							                         %
%%%%%%%%%%%%%%%%%%%%%%%%%%%%%%%%%%%%%%%%%%%%%%%%%%%%%%%%%%%%%%%%%%%%%%%%%%%%%%%%
% Add include statements below. All the text should be in the folders.
\documentclass[%
    reprint,
    superscriptaddress,
    %groupedaddress,
    %unsortedaddress,
    %runinaddress,
    %frontmatterverbose,
    %preprint,
    longbibliography,
    bibnotes,
    %showpacs,preprintnumbers,
    %nofootinbib,
    %nobibnotes,
    %bibnotes,
     amsmath,amssymb,
     %aps,
     aip,
     jcp,                                       % J. Chem. Phys
    %pra,
    %prb,
    %rmp,
    %prstab,
    %prstper,
    %floatfix,
    ]{revtex4-1}
%%%%%%%%%%%%%%%%%%%%%%%%%%%%%%%%%%%%%%%%%%%%%%%%%%%%%%%%%%%%%%%%%%%%%%%%%%%%%%%%
%					 	                       LIBRARIES							                     %
%%%%%%%%%%%%%%%%%%%%%%%%%%%%%%%%%%%%%%%%%%%%%%%%%%%%%%%%%%%%%%%%%%%%%%%%%%%%%%%%
% Add whatever libraries you need here. Add a description if you want
\usepackage{graphicx}			 % Include figure files
\usepackage{dcolumn}			 % Align table columns on decimal point
\usepackage{tcolorbox}		 % use tcolorbox environment to box equations
\usepackage[USenglish]{babel}
\usepackage[useregional]{datetime2}
\usepackage[utf8]{inputenc}
\usepackage[T1]{fontenc}
\usepackage{enumerate}
\usepackage{caption}
\usepackage{subcaption}
\usepackage{physics}
\usepackage{hyperref}
\usepackage{chemformula}
\usepackage{booktabs}
\usepackage{makecell}
\usepackage{fullpage}
\usepackage{natbib}
\usepackage{setspace}
\usepackage{amsmath}
\usepackage{bm}              % enchanced bold math symbols
\usepackage{amssymb}
\usepackage{listings} 			 % For code citations
\usepackage{amsfonts}
\usepackage{mathtools}
\usepackage{commath}
\usepackage{fancyhdr}
\usepackage{lastpage}
\usepackage{tikz}
\usepackage{graphicx}
	\graphicspath{ {images/} }
\usepackage{feynmp-auto}
\usepackage[toc,page]{appendix}
%%%%%%%%%%%%%%%%%%%%%%%%%%%%%%%%%%%%%%%%%%%%%%%%%%%%%%%%%%%%%%%%%%%%%%%%%%%%%%%%
%					 	                      NEW COMMANDS							                   %
%%%%%%%%%%%%%%%%%%%%%%%%%%%%%%%%%%%%%%%%%%%%%%%%%%%%%%%%%%%%%%%%%%%%%%%%%%%%%%%%
% Redefine commands if necessary
\renewcommand\vec{\mathbf}                      % use boldface for vectors
% New commands
%------------------------------------Date---------------------------------------
\newcommand{\seminardate}{\DTMdisplaydate{2017}{7}{25}{-1}}
\newcommand{\revisiondate}{\DTMdisplaydate{2017}{7}{26}{-1}}
%-------------------------------------------------------------------------------
\newcommand*\Del{\vec{\nabla}}                  % del operator
\newcommand*\diff{\mathop{}\!\mathrm{d}}		    % differential d
\newcommand*\Diff[1]{\mathop{}\!\mathrm{d^#1}}	% d^(...)
\newcommand*{\scalarnorm}[1]                    % scalar norm |*|
  {\left\lvert#1\right\rvert}
\newcommand*{\vectornorm}[1]                    % vector norm ||*||
  {\left\lVert#1\right\rVert}
\newcommand*\onehalf{\frac{1}{2}}               % fraction 1/2
%------------------------------------Time---------------------------------------
\newcommand*\dlt[1]{\delta #1}
\newcommand*\timestep{\dlt{t}}              % timestep dt
\newcommand*\halfstep                       % half timestep 1/2 dt
  {\onehalf \timestep}
\newcommand*\timeInHalfStep{\left(t + \halfstep\right)}
\newcommand*\timeInFullStep{\left(t + \timestep\right)}
%-------------------------------------------------------------------------------
\input{rattle/definitions}
\input{ke_polymer/definitions}
%-------------------------------------------------------------------------------
%\setlength{\parindent}{0pt}						          % do not indent
\numberwithin{equation}{section}                % number eqs within sections
\begin{document}
\lstset{language=Python}
%%%%%%%%%%%%%%%%%%%%%%%%%%%%%%%%%%%%%%%%%%%%%%%%%%%%%%%%%%%%%%%%%%%%%%%%%%%%%%%%
%					 	                        TITLE							                         %
%%%%%%%%%%%%%%%%%%%%%%%%%%%%%%%%%%%%%%%%%%%%%%%%%%%%%%%%%%%%%%%%%%%%%%%%%%%%%%%%
\title{Group Seminar (revised: \revisiondate)}
\thanks{Professor Stratt, Vale Cofer-Shabica, Yan Zhao, Andrew Ton, Mansheej Paul, Evan Coleman}
\author{Artur Avkhadiev}
\email{artur\_avkhadiev@brown.edu}
\affiliation{Department of Physics, Brown University, Providence RI 02912, USA}
\date{\seminardate}
%%%%%%%%%%%%%%%%%%%%%%%%%%%%%%%%%%%%%%%%%%%%%%%%%%%%%%%%%%%%%%%%%%%%%%%%%%%%%%%%
%					 	                       ABSTRACT							                       %
%%%%%%%%%%%%%%%%%%%%%%%%%%%%%%%%%%%%%%%%%%%%%%%%%%%%%%%%%%%%%%%%%%%%%%%%%%%%%%%%
\begin{abstract}
  This report consist of two parts:
  \begin{enumerate}
      \item A discussion of the constraint dynamics algorithm based on velocity Verlet numerical integration scheme, \rattle.
      \item A calculation of the kinetic energy of a freely-jointed polymer chain using the bond-vector representation in the CM frame of the molecule.
    \end{enumerate}
\end{abstract}
\maketitle
\tableofcontents						 % hide table of contents
%%%%%%%%%%%%%%%%%%%%%%%%%%%%%%%%%%%%%%%%%%%%%%%%%%%%%%%%%%%%%%%%%%%%%%%%%%%%%%%%
%					 	                        BODY							                         %
%%%%%%%%%%%%%%%%%%%%%%%%%%%%%%%%%%%%%%%%%%%%%%%%%%%%%%%%%%%%%%%%%%%%%%%%%%%%%%%%
% Add include statements below. All the text should be in the folders.
\input{rattle/main}
\input{ke_polymer/main}
%-------------------------------------------------------------------------------
%%%%%%%%%%%%%%%%%%%%%%%%%%%%%%%%%%%%%%%%%%%%%%%%%%%%%%%%%%%%%%%%%%%%%%%%%%%%%%%%
%					                        BIBLIOGRAPHY							                   %
%%%%%%%%%%%%%%%%%%%%%%%%%%%%%%%%%%%%%%%%%%%%%%%%%%%%%%%%%%%%%%%%%%%%%%%%%%%%%%%%
% The bibliography file is reference,bib. If you need to add a
% reference, make sure all the entries look good (e.g., ensure
% proper capitalization by putting {} around the letters to be
% capitalized. Reference the books by chapters!
\nocite{*}
\bibliographystyle{apalike}
\bibliography{reference}
%-----------------------------
%%%%%%%%%%%%%%%%%%%%%%%%%%%%%%%%%%%%%%%%%%%%%%%%%%%%%%%%%%%%%%%%%%%%%%%%%%%%%%%%
%					 	                      APPENDICES							                     %
%%%%%%%%%%%%%%%%%%%%%%%%%%%%%%%%%%%%%%%%%%%%%%%%%%%%%%%%%%%%%%%%%%%%%%%%%%%%%%%%
% Add all appendices in appendices/main.tex with include statements.
% see appendices/sub-example-app.tex for an example.
\onecolumngrid
\input{appendices/main}
%-------------------------------------------------------------------------------
\end{document}

\documentclass[%
    reprint,
    superscriptaddress,
    %groupedaddress,
    %unsortedaddress,
    %runinaddress,
    %frontmatterverbose,
    %preprint,
    longbibliography,
    bibnotes,
    %showpacs,preprintnumbers,
    %nofootinbib,
    %nobibnotes,
    %bibnotes,
     amsmath,amssymb,
     %aps,
     aip,
     jcp,                                       % J. Chem. Phys
    %pra,
    %prb,
    %rmp,
    %prstab,
    %prstper,
    %floatfix,
    ]{revtex4-1}
%%%%%%%%%%%%%%%%%%%%%%%%%%%%%%%%%%%%%%%%%%%%%%%%%%%%%%%%%%%%%%%%%%%%%%%%%%%%%%%%
%					 	                       LIBRARIES							                     %
%%%%%%%%%%%%%%%%%%%%%%%%%%%%%%%%%%%%%%%%%%%%%%%%%%%%%%%%%%%%%%%%%%%%%%%%%%%%%%%%
% Add whatever libraries you need here. Add a description if you want
\usepackage{graphicx}			 % Include figure files
\usepackage{dcolumn}			 % Align table columns on decimal point
\usepackage{tcolorbox}		 % use tcolorbox environment to box equations
\usepackage[USenglish]{babel}
\usepackage[useregional]{datetime2}
\usepackage[utf8]{inputenc}
\usepackage[T1]{fontenc}
\usepackage{enumerate}
\usepackage{caption}
\usepackage{subcaption}
\usepackage{physics}
\usepackage{hyperref}
\usepackage{chemformula}
\usepackage{booktabs}
\usepackage{makecell}
\usepackage{fullpage}
\usepackage{natbib}
\usepackage{setspace}
\usepackage{amsmath}
\usepackage{bm}              % enchanced bold math symbols
\usepackage{amssymb}
\usepackage{listings} 			 % For code citations
\usepackage{amsfonts}
\usepackage{mathtools}
\usepackage{commath}
\usepackage{fancyhdr}
\usepackage{lastpage}
\usepackage{tikz}
\usepackage{graphicx}
	\graphicspath{ {images/} }
\usepackage{feynmp-auto}
\usepackage[toc,page]{appendix}
%%%%%%%%%%%%%%%%%%%%%%%%%%%%%%%%%%%%%%%%%%%%%%%%%%%%%%%%%%%%%%%%%%%%%%%%%%%%%%%%
%					 	                      NEW COMMANDS							                   %
%%%%%%%%%%%%%%%%%%%%%%%%%%%%%%%%%%%%%%%%%%%%%%%%%%%%%%%%%%%%%%%%%%%%%%%%%%%%%%%%
% Redefine commands if necessary
\renewcommand\vec{\mathbf}                      % use boldface for vectors
% New commands
%------------------------------------Date---------------------------------------
\newcommand{\seminardate}{\DTMdisplaydate{2017}{7}{25}{-1}}
\newcommand{\revisiondate}{\DTMdisplaydate{2017}{7}{26}{-1}}
%-------------------------------------------------------------------------------
\newcommand*\Del{\vec{\nabla}}                  % del operator
\newcommand*\diff{\mathop{}\!\mathrm{d}}		    % differential d
\newcommand*\Diff[1]{\mathop{}\!\mathrm{d^#1}}	% d^(...)
\newcommand*{\scalarnorm}[1]                    % scalar norm |*|
  {\left\lvert#1\right\rvert}
\newcommand*{\vectornorm}[1]                    % vector norm ||*||
  {\left\lVert#1\right\rVert}
\newcommand*\onehalf{\frac{1}{2}}               % fraction 1/2
%------------------------------------Time---------------------------------------
\newcommand*\dlt[1]{\delta #1}
\newcommand*\timestep{\dlt{t}}              % timestep dt
\newcommand*\halfstep                       % half timestep 1/2 dt
  {\onehalf \timestep}
\newcommand*\timeInHalfStep{\left(t + \halfstep\right)}
\newcommand*\timeInFullStep{\left(t + \timestep\right)}
%-------------------------------------------------------------------------------
\input{rattle/definitions}
\input{ke_polymer/definitions}
%-------------------------------------------------------------------------------
%\setlength{\parindent}{0pt}						          % do not indent
\numberwithin{equation}{section}                % number eqs within sections
\begin{document}
\lstset{language=Python}
%%%%%%%%%%%%%%%%%%%%%%%%%%%%%%%%%%%%%%%%%%%%%%%%%%%%%%%%%%%%%%%%%%%%%%%%%%%%%%%%
%					 	                        TITLE							                         %
%%%%%%%%%%%%%%%%%%%%%%%%%%%%%%%%%%%%%%%%%%%%%%%%%%%%%%%%%%%%%%%%%%%%%%%%%%%%%%%%
\title{Group Seminar (revised: \revisiondate)}
\thanks{Professor Stratt, Vale Cofer-Shabica, Yan Zhao, Andrew Ton, Mansheej Paul, Evan Coleman}
\author{Artur Avkhadiev}
\email{artur\_avkhadiev@brown.edu}
\affiliation{Department of Physics, Brown University, Providence RI 02912, USA}
\date{\seminardate}
%%%%%%%%%%%%%%%%%%%%%%%%%%%%%%%%%%%%%%%%%%%%%%%%%%%%%%%%%%%%%%%%%%%%%%%%%%%%%%%%
%					 	                       ABSTRACT							                       %
%%%%%%%%%%%%%%%%%%%%%%%%%%%%%%%%%%%%%%%%%%%%%%%%%%%%%%%%%%%%%%%%%%%%%%%%%%%%%%%%
\begin{abstract}
  This report consist of two parts:
  \begin{enumerate}
      \item A discussion of the constraint dynamics algorithm based on velocity Verlet numerical integration scheme, \rattle.
      \item A calculation of the kinetic energy of a freely-jointed polymer chain using the bond-vector representation in the CM frame of the molecule.
    \end{enumerate}
\end{abstract}
\maketitle
\tableofcontents						 % hide table of contents
%%%%%%%%%%%%%%%%%%%%%%%%%%%%%%%%%%%%%%%%%%%%%%%%%%%%%%%%%%%%%%%%%%%%%%%%%%%%%%%%
%					 	                        BODY							                         %
%%%%%%%%%%%%%%%%%%%%%%%%%%%%%%%%%%%%%%%%%%%%%%%%%%%%%%%%%%%%%%%%%%%%%%%%%%%%%%%%
% Add include statements below. All the text should be in the folders.
\input{rattle/main}
\input{ke_polymer/main}
%-------------------------------------------------------------------------------
%%%%%%%%%%%%%%%%%%%%%%%%%%%%%%%%%%%%%%%%%%%%%%%%%%%%%%%%%%%%%%%%%%%%%%%%%%%%%%%%
%					                        BIBLIOGRAPHY							                   %
%%%%%%%%%%%%%%%%%%%%%%%%%%%%%%%%%%%%%%%%%%%%%%%%%%%%%%%%%%%%%%%%%%%%%%%%%%%%%%%%
% The bibliography file is reference,bib. If you need to add a
% reference, make sure all the entries look good (e.g., ensure
% proper capitalization by putting {} around the letters to be
% capitalized. Reference the books by chapters!
\nocite{*}
\bibliographystyle{apalike}
\bibliography{reference}
%-----------------------------
%%%%%%%%%%%%%%%%%%%%%%%%%%%%%%%%%%%%%%%%%%%%%%%%%%%%%%%%%%%%%%%%%%%%%%%%%%%%%%%%
%					 	                      APPENDICES							                     %
%%%%%%%%%%%%%%%%%%%%%%%%%%%%%%%%%%%%%%%%%%%%%%%%%%%%%%%%%%%%%%%%%%%%%%%%%%%%%%%%
% Add all appendices in appendices/main.tex with include statements.
% see appendices/sub-example-app.tex for an example.
\onecolumngrid
\input{appendices/main}
%-------------------------------------------------------------------------------
\end{document}

%-------------------------------------------------------------------------------
%%%%%%%%%%%%%%%%%%%%%%%%%%%%%%%%%%%%%%%%%%%%%%%%%%%%%%%%%%%%%%%%%%%%%%%%%%%%%%%%
%					                        BIBLIOGRAPHY							                   %
%%%%%%%%%%%%%%%%%%%%%%%%%%%%%%%%%%%%%%%%%%%%%%%%%%%%%%%%%%%%%%%%%%%%%%%%%%%%%%%%
% The bibliography file is reference,bib. If you need to add a
% reference, make sure all the entries look good (e.g., ensure
% proper capitalization by putting {} around the letters to be
% capitalized. Reference the books by chapters!
\nocite{*}
\bibliographystyle{apalike}
\bibliography{reference}
%-----------------------------
%%%%%%%%%%%%%%%%%%%%%%%%%%%%%%%%%%%%%%%%%%%%%%%%%%%%%%%%%%%%%%%%%%%%%%%%%%%%%%%%
%					 	                      APPENDICES							                     %
%%%%%%%%%%%%%%%%%%%%%%%%%%%%%%%%%%%%%%%%%%%%%%%%%%%%%%%%%%%%%%%%%%%%%%%%%%%%%%%%
% Add all appendices in appendices/main.tex with include statements.
% see appendices/sub-example-app.tex for an example.
\onecolumngrid
\documentclass[%
    reprint,
    superscriptaddress,
    %groupedaddress,
    %unsortedaddress,
    %runinaddress,
    %frontmatterverbose,
    %preprint,
    longbibliography,
    bibnotes,
    %showpacs,preprintnumbers,
    %nofootinbib,
    %nobibnotes,
    %bibnotes,
     amsmath,amssymb,
     %aps,
     aip,
     jcp,                                       % J. Chem. Phys
    %pra,
    %prb,
    %rmp,
    %prstab,
    %prstper,
    %floatfix,
    ]{revtex4-1}
%%%%%%%%%%%%%%%%%%%%%%%%%%%%%%%%%%%%%%%%%%%%%%%%%%%%%%%%%%%%%%%%%%%%%%%%%%%%%%%%
%					 	                       LIBRARIES							                     %
%%%%%%%%%%%%%%%%%%%%%%%%%%%%%%%%%%%%%%%%%%%%%%%%%%%%%%%%%%%%%%%%%%%%%%%%%%%%%%%%
% Add whatever libraries you need here. Add a description if you want
\usepackage{graphicx}			 % Include figure files
\usepackage{dcolumn}			 % Align table columns on decimal point
\usepackage{tcolorbox}		 % use tcolorbox environment to box equations
\usepackage[USenglish]{babel}
\usepackage[useregional]{datetime2}
\usepackage[utf8]{inputenc}
\usepackage[T1]{fontenc}
\usepackage{enumerate}
\usepackage{caption}
\usepackage{subcaption}
\usepackage{physics}
\usepackage{hyperref}
\usepackage{chemformula}
\usepackage{booktabs}
\usepackage{makecell}
\usepackage{fullpage}
\usepackage{natbib}
\usepackage{setspace}
\usepackage{amsmath}
\usepackage{bm}              % enchanced bold math symbols
\usepackage{amssymb}
\usepackage{listings} 			 % For code citations
\usepackage{amsfonts}
\usepackage{mathtools}
\usepackage{commath}
\usepackage{fancyhdr}
\usepackage{lastpage}
\usepackage{tikz}
\usepackage{graphicx}
	\graphicspath{ {images/} }
\usepackage{feynmp-auto}
\usepackage[toc,page]{appendix}
%%%%%%%%%%%%%%%%%%%%%%%%%%%%%%%%%%%%%%%%%%%%%%%%%%%%%%%%%%%%%%%%%%%%%%%%%%%%%%%%
%					 	                      NEW COMMANDS							                   %
%%%%%%%%%%%%%%%%%%%%%%%%%%%%%%%%%%%%%%%%%%%%%%%%%%%%%%%%%%%%%%%%%%%%%%%%%%%%%%%%
% Redefine commands if necessary
\renewcommand\vec{\mathbf}                      % use boldface for vectors
% New commands
%------------------------------------Date---------------------------------------
\newcommand{\seminardate}{\DTMdisplaydate{2017}{7}{25}{-1}}
\newcommand{\revisiondate}{\DTMdisplaydate{2017}{7}{26}{-1}}
%-------------------------------------------------------------------------------
\newcommand*\Del{\vec{\nabla}}                  % del operator
\newcommand*\diff{\mathop{}\!\mathrm{d}}		    % differential d
\newcommand*\Diff[1]{\mathop{}\!\mathrm{d^#1}}	% d^(...)
\newcommand*{\scalarnorm}[1]                    % scalar norm |*|
  {\left\lvert#1\right\rvert}
\newcommand*{\vectornorm}[1]                    % vector norm ||*||
  {\left\lVert#1\right\rVert}
\newcommand*\onehalf{\frac{1}{2}}               % fraction 1/2
%------------------------------------Time---------------------------------------
\newcommand*\dlt[1]{\delta #1}
\newcommand*\timestep{\dlt{t}}              % timestep dt
\newcommand*\halfstep                       % half timestep 1/2 dt
  {\onehalf \timestep}
\newcommand*\timeInHalfStep{\left(t + \halfstep\right)}
\newcommand*\timeInFullStep{\left(t + \timestep\right)}
%-------------------------------------------------------------------------------
\input{rattle/definitions}
\input{ke_polymer/definitions}
%-------------------------------------------------------------------------------
%\setlength{\parindent}{0pt}						          % do not indent
\numberwithin{equation}{section}                % number eqs within sections
\begin{document}
\lstset{language=Python}
%%%%%%%%%%%%%%%%%%%%%%%%%%%%%%%%%%%%%%%%%%%%%%%%%%%%%%%%%%%%%%%%%%%%%%%%%%%%%%%%
%					 	                        TITLE							                         %
%%%%%%%%%%%%%%%%%%%%%%%%%%%%%%%%%%%%%%%%%%%%%%%%%%%%%%%%%%%%%%%%%%%%%%%%%%%%%%%%
\title{Group Seminar (revised: \revisiondate)}
\thanks{Professor Stratt, Vale Cofer-Shabica, Yan Zhao, Andrew Ton, Mansheej Paul, Evan Coleman}
\author{Artur Avkhadiev}
\email{artur\_avkhadiev@brown.edu}
\affiliation{Department of Physics, Brown University, Providence RI 02912, USA}
\date{\seminardate}
%%%%%%%%%%%%%%%%%%%%%%%%%%%%%%%%%%%%%%%%%%%%%%%%%%%%%%%%%%%%%%%%%%%%%%%%%%%%%%%%
%					 	                       ABSTRACT							                       %
%%%%%%%%%%%%%%%%%%%%%%%%%%%%%%%%%%%%%%%%%%%%%%%%%%%%%%%%%%%%%%%%%%%%%%%%%%%%%%%%
\begin{abstract}
  This report consist of two parts:
  \begin{enumerate}
      \item A discussion of the constraint dynamics algorithm based on velocity Verlet numerical integration scheme, \rattle.
      \item A calculation of the kinetic energy of a freely-jointed polymer chain using the bond-vector representation in the CM frame of the molecule.
    \end{enumerate}
\end{abstract}
\maketitle
\tableofcontents						 % hide table of contents
%%%%%%%%%%%%%%%%%%%%%%%%%%%%%%%%%%%%%%%%%%%%%%%%%%%%%%%%%%%%%%%%%%%%%%%%%%%%%%%%
%					 	                        BODY							                         %
%%%%%%%%%%%%%%%%%%%%%%%%%%%%%%%%%%%%%%%%%%%%%%%%%%%%%%%%%%%%%%%%%%%%%%%%%%%%%%%%
% Add include statements below. All the text should be in the folders.
\input{rattle/main}
\input{ke_polymer/main}
%-------------------------------------------------------------------------------
%%%%%%%%%%%%%%%%%%%%%%%%%%%%%%%%%%%%%%%%%%%%%%%%%%%%%%%%%%%%%%%%%%%%%%%%%%%%%%%%
%					                        BIBLIOGRAPHY							                   %
%%%%%%%%%%%%%%%%%%%%%%%%%%%%%%%%%%%%%%%%%%%%%%%%%%%%%%%%%%%%%%%%%%%%%%%%%%%%%%%%
% The bibliography file is reference,bib. If you need to add a
% reference, make sure all the entries look good (e.g., ensure
% proper capitalization by putting {} around the letters to be
% capitalized. Reference the books by chapters!
\nocite{*}
\bibliographystyle{apalike}
\bibliography{reference}
%-----------------------------
%%%%%%%%%%%%%%%%%%%%%%%%%%%%%%%%%%%%%%%%%%%%%%%%%%%%%%%%%%%%%%%%%%%%%%%%%%%%%%%%
%					 	                      APPENDICES							                     %
%%%%%%%%%%%%%%%%%%%%%%%%%%%%%%%%%%%%%%%%%%%%%%%%%%%%%%%%%%%%%%%%%%%%%%%%%%%%%%%%
% Add all appendices in appendices/main.tex with include statements.
% see appendices/sub-example-app.tex for an example.
\onecolumngrid
\input{appendices/main}
%-------------------------------------------------------------------------------
\end{document}

%-------------------------------------------------------------------------------
\end{document}

%-------------------------------------------------------------------------------
\end{document}

  %-----------------------------------------------------------------------------
  \par To summarize, one can calculate the total kinetic energy of a polymer molecule as $K\targetrep$, where
  \begin{tcolorbox}
  \begin{equation}
  \label{eq:ke-total}
  \begin{aligned}
    K &= \onehalf (N+1)m \vcm^2 \\
      &+ \onehalf m\bondlength^2 C_{ij} \left(\bondvel^{i} \bondvel^{j}\right)
  \end{aligned}
  \end{equation}
  \end{tcolorbox}
