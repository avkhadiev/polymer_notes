%%%%%%%%%%%%%%%%%%%%%%%%%%%%%%%%%%%%%%%%%%%%%%%%%%%%%%%%%%%%%%%%%%%%%%%%%%%%%%%%
%					 	             KINETIC ENERGY OF THE SYSTEM							             %
%%%%%%%%%%%%%%%%%%%%%%%%%%%%%%%%%%%%%%%%%%%%%%%%%%%%%%%%%%%%%%%%%%%%%%%%%%%%%%%%
% Subsection names and labels
\subsubsection{Kinetic Energy of a Classical System}
\label{sec:setup-kinetic-energy-of-system}
%-------------------------------------------------------------------------------
% Add text directly into the main file, or use \input statements
%-------------------------------------------------------------------------------
  \par Kinetic energy of a classical system can be conveniently written as:
  \begin{equation}
  \label{eq:kinetic-energy-of-system}
    K = \kecm + \kerel{a}
  \end{equation}
  \par Where we decompose the motion of the system into the motion of its center of mass and the relative motions about the center of mass. The former gives the first term in the equation above; the latter, the second term.
  \par This equation is written in the space-fixed frame --- $\vcm$ is to be computed from atoms' velocities and masses. 
  \par Here, $\mtot$ is the total mass of the system that consists of $N + 1$ sites, indexed from $0$ to $N$. In the case of a polymer molecule, each $a$th
  site corresponds to the $a$th atom in the chain, with mass $m_{a}$, position $\pos_{a}$, and velocity $\vel_{a}$.
%-------------------------------------------------------------------------------
