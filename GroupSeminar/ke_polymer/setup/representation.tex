%%%%%%%%%%%%%%%%%%%%%%%%%%%%%%%%%%%%%%%%%%%%%%%%%%%%%%%%%%%%%%%%%%%%%%%%%%%%%%%%
%					 	                  REPRESENATATION					                         %
%%%%%%%%%%%%%%%%%%%%%%%%%%%%%%%%%%%%%%%%%%%%%%%%%%%%%%%%%%%%%%%%%%%%%%%%%%%%%%%%
% Subsection names and labels
\subsubsection{Representation}
\label{sec:setup-representation}
%-------------------------------------------------------------------------------
% Add text directly into the main file, or use \input statements
%-------------------------------------------------------------------------------
    \par While the state of our polymer molecule model with $N + 1$ atoms can be represented, at any moment in time, by $m$, $\bondlength$, and $6(N + 1)$ position and velocity vectors of the constituent atoms (space-fixed frame), we would like to choose a different representation.
    \par Note: if we know $m$, $d$, and the position and velocity vectors of center of mass, $\rcm$ and $\vcm$, then all the additional information we need to specify a unique state at any point in time can be captured by $N$ \emph{bond vectors} $\bondpos_{i}$ and their time derivatives $\bondvel_{i}$.
    \par We call a bond vector $\bondpos_{i}$ the unit vector parallel to the bond between atoms $i-1$ and $i$ in the polymer molecule. The vector points towards the atom with the larger index. It makes sense to use unit vectors, since the bonds in our model are constrained to a constant length $\bondlength$.
    \par The bond vector $\bondpos_{i}$ can be calculated as:
    \begin{tcolorbox}
      \begin{equation}
      \label{eq:bond-vec}
        \bondpos_{i}
          = \frac{\pos_{i} - \pos_{i-1}}{\left| \pos_{i} - \pos_{i-1} \right|}
      \end{equation}
    \end{tcolorbox}
    \par Because of the constraint on the bond length, we can write down the time derivative of the bond vector \cite{allen}:
    \begin{tcolorbox}
      \begin{equation}
      \label{eq:bond-vel}
        \bondvel = \vec{\omega} \cross \bondpos
      \end{equation}
    \end{tcolorbox}
    \par Here, $\vec{\omega}$ is the angular velocity vector of the corresponding bond, and each vector is written in the same space-fixed frame coorindates. $\bondvel_{i}$ is not necessarily a unit vector.
    \par For a bond vector $\bondpos_{i}$, its angular velocity vector $\vec{\omega}_{i}$ can be calculated as
    %-------------------------------------------------------------------------------
% Add math environment directly below. Include a label.
%-------------------------------------------------------------------------------
\begin{equation}
\label{eq:angular-velocity-of-bond}
\begin{aligned}
  \vec{\omega}_{i}
   &= \frac
        {\vec{r}_{\textrm{bond CM}} \cross \vec{v}_{\textrm{bond CM}}}
        {\vectornorm{\vec{r}_{\textrm{bond CM}}}^2}
\end{aligned}
\end{equation}
%-------------------------------------------------------------------------------

%-------------------------------------------------------------------------------
