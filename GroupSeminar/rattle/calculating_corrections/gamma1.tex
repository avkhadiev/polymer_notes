%%%%%%%%%%%%%%%%%%%%%%%%%%%%%%%%%%%%%%%%%%%%%%%%%%%%%%%%%%%%%%%%%%%%%%%%%%%%%%%%
%					 	     RATTLE: CALCULATING CORRECTIONS: FIRST GAMMA						       %
%%%%%%%%%%%%%%%%%%%%%%%%%%%%%%%%%%%%%%%%%%%%%%%%%%%%%%%%%%%%%%%%%%%%%%%%%%%%%%%%
% Provides formulae for calculating the approximation of Lagrange multipliers at time t
%-------------------------------------------------------------------------------
% Section names and labels
\subsubsection{Calculating $\pmb{\iterationThis{\rattleLagrangeApprox}(t)}$}
\label{sec:rattle_iterativeCorrection_gamma1}
  %-----------------------------------------------------------------------------
  \par Let us begin with $\iterationThis{\rattleLagrangeApprox}(t)$. Suppose that at the $\rattleCorrectionIndexThis$th correction step, a bond vector $\iterationThis{\rattleBondPos}\timeInFullStep$ does not satisy  \ref{eq:rattle_constraints_posApprox} --- a correction is in order, so $\iterationThis{\rattleLagrangeApprox}(t)$ needs to be calculated. This quantity is an approximation to $\iterationThis{\rattleLagrange}(t)$. If we knew $\iterationThis{\rattleLagrange}(t)$, we could calculate $\iterationNext{\rattleBondPos}\timeInFullStep$ in such a way so as to satisfy the constraint exactly. That would mean:
  \begin{align*}
    \rattleBondlength^2
    &- \vectornorm{\iterationNext{\rattleBondPos}\timeInFullStep}^2 \\
    &= \rattleBondlength^2
      - \vectornorm{\iterationNext{\rattleBondPos}\timeInFullStep}^2 \\
    &= \rattleBondlength^2
      - \vectornorm{
          \iterationThis{\rattleBondPos}\timeInFullStep
          + \iterationThis{\dlt{\rattleBondPos}}\timeInFullStep
        }^2 \\
    &= 0
  \end{align*}
  \par Where
  \begin{align*}
    \iterationThis{\dlt{\rattleBondPos}}\timeInFullStep
      &= - \timestep^2
        \iterationThis{\rattleLagrange}(t)
        \left(
          \frac{1}{\rattleMassFirst} + \frac{1}{\rattleMassSecond}
        \right)
        \rattleBondPos(t)
  \end{align*}
  \par Expanding out, we obtain:
  \begin{widetext}
  \begin{equation}
  \label{eq:rattle_calculatingCorrections_firstEquation}
  \begin{aligned}
  0 = \left(
        \rattleBondlength^2 - \vectornorm{\iterationThis{\rattleBondPos}\timeInFullStep}^2
      \right)
      &- 2\timestep^2
        \iterationThis{\rattleLagrange}(t)
        \left(
          \frac{1}{\rattleMassFirst} + \frac{1}{\rattleMassSecond}
        \right)
        \left(
          \rattleBondPos(t) \bigcdot \iterationThis{\rattleBondPos}\timeInFullStep
        \right) \\
      &+ \left(
          \timestep^2
          \iterationThis{\rattleLagrange}(t)
          \left(
            \frac{1}{\rattleMassFirst} + \frac{1}{\rattleMassSecond}
          \right)
        \right)^2
        \vectornorm{\rattleBondPos(t)}^2 \\
  \end{aligned}
  \end{equation}
  \end{widetext}
  \par At this point, we will ignore the term quadratic in $\iterationThis{\rattleLagrange}\timeInFullStep$, and solve the resulting linear equation. The solution to this equation will be our approximation of $\iterationThis{\rattleLagrange}\timeInFullStep$: $\iterationThis{\rattleLagrangeApprox}\timeInFullStep$. Thus,
  \begin{tcolorbox}
  \begin{equation}
  \label{eq:rattle_calculatingCorrections_lambdaApproxFrist}
    \iterationThis{\rattleLagrangeApprox}(t)
      = \frac
          {\rattleBondlength^2
            - \vectornorm{\iterationThis{\rattleBondPos}\timeInFullStep}^2}
          {2\timestep^2
            \left(
              \frac{1}{\rattleMassFirst} + \frac{1}{\rattleMassSecond}
            \right)
            \left(
              \rattleBondPos(t) \bigcdot \iterationThis{\rattleBondPos}\timeInFullStep
            \right)}
  \end{equation}
  \end{tcolorbox}
  \par Assumming a reasonable timestep, at no point in the iterative correction should the position vector $\iterationThis{\rattleBondPos}\timeInFullStep$ turn away from $\rattleBondPos(t)$ so much as to be perpendicular to it and yield $\rattleBondPos(t) \bigcdot \iterationThis{\rattleBondPos}\timeInFullStep = 0$. However, to avoid division by zero and ensure the algorithm is implemented correctly, it is a good practice to check during the correction stage that this inner product is not too small.
  %-----------------------------------------------------------------------------
