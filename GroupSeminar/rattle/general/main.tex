%%%%%%%%%%%%%%%%%%%%%%%%%%%%%%%%%%%%%%%%%%%%%%%%%%%%%%%%%%%%%%%%%%%%%%%%%%%%%%%%
%					 	                       GENERAL								                     %
%%%%%%%%%%%%%%%%%%%%%%%%%%%%%%%%%%%%%%%%%%%%%%%%%%%%%%%%%%%%%%%%%%%%%%%%%%%%%%%%
% general formulation of a problem: solving equations of motions with holonomic constraints
%-------------------------------------------------------------------------------
% Section names and labels
\subsection{General Problem Formulation}
\label{sec:rattle_problemFormulation}
  %-----------------------------------------------------------------------------
  % Add text directly into the main file, or use \input statements
  %-----------------------------------------------------------------------------
  \par Generally, motion with (holonomic) constraints can be captured by the following two equations \cite{}:
  \begin{tcolorbox}
  \begin{equation}
  \label{eq:rattle_problemFormulation_eom}
  \rattleGeneralMass \rattleGeneralPosDDot
    = \rattleGeneralForce + \rattleGeneralConstraintForce
  \end{equation}
  %-----------------------------------------------------------------------------
  \begin{equation}
  \label{eq:rattle_problemFormulation_constraintForce}
  \rattleGeneralConstraintForce(t)
    = - \sum_{\rattleGeneralConstraintIndex = 1}^{\rattleGeneralNConstraints}
        \rattleGeneralLagrange(t)
        \Del_{\rattleGeneralPos}
          \rattleGeneralConstraint \left( t,\,\rattleGeneralSetPos \right)
  \end{equation}
  \end{tcolorbox}
  %-----------------------------------------------------------------------------
  \par Where:
  \begin{itemize}
    \item $\rattleGeneralPointIndex$ indexes $\rattleGeneralNPoints$ particles with masses $\rattleGeneralMass$.
    \item $\rattleGeneralConstraintIndex$ indexes $\rattleGeneralNConstraints$ constraints $\rattleGeneralConstraint$.
    \item $\rattleGeneralForce$ is the total force on the particle $\rattleGeneralPointIndex$.
    \item $\rattleGeneralConstraint$ depends only on time and particle positions $\rattleGeneralPos$; in particular, it does not depend on particle velocities (that is, the constraints are \emph{holonomic}).
    \item $\rattleGeneralConstraintForce$ can be regarded as a ``constraint force'' on the particle at $\rattleGeneralPos$. Its Lagrange multiplier $\rattleGeneralLagrange(t)$, proportional to the magnitude of the ``force'', assumes the necessary units depending on the units of the corresponding constraint $\rattleGeneralConstraint(t)$.
  \end{itemize}
  %-----------------------------------------------------------------------------
