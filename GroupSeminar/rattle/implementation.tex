%%%%%%%%%%%%%%%%%%%%%%%%%%%%%%%%%%%%%%%%%%%%%%%%%%%%%%%%%%%%%%%%%%%%%%%%%%%%%%%%
%					 	              RATTLE: IMPLEMENTATION DETAILS				              %
%%%%%%%%%%%%%%%%%%%%%%%%%%%%%%%%%%%%%%%%%%%%%%%%%%%%%%%%%%%%%%%%%%%%%%%%%%%%%%%%
% Provides formulae for calculating the approximations to coorective displacements
%-------------------------------------------------------------------------------
% Section names and labels
\subsubsection{Implementation Details}
\label{sec:rattle_implementationDetails}
  %-----------------------------------------------------------------------------
  \par Discussion Points:
  \begin{itemize}
      \item \emph{It is not too hard to debug \rattle}: if you missed an odd number of minus signs in calculating the approximations to constraint forces, your atoms in constrainted atomic pairs will be inexorably pushed away from each other.
      \item \emph{Bookkeeping to avoid unnecessary work:} keeping track of all atoms that were moved in the last correction step $\rattleCorrectionIndexThis - 1$ or are being moved in the current step $\rattleCorrectionIndexThis$. Implementation by Allen \& Tildesley\cite{}.
      \item \emph{How does the tolerance parameter $\rattleTol$ affect quantities computed in the simulation}? See plots from the diatomic exercise. It is a molecular dynamics simulation of a single freely rotating diatomic with its center of mass at the origin. Atoms in the molecule have equal mass. The diatomic is initialiazed given two parameters $\left(\hat{\vec{\Omega}},\,\dot{\hat{\vec{\Omega}}}\right)$ and allowed to rotate freely, subject to the constraint of the constant bond length.
      \item \emph{How should the intramolecular potential be modified if the simulation is using \rattle for numerical integration?}. How persisting interatomic interactions between pairs of constrained atoms affect numerical stability.
  \end{itemize}
  %-----------------------------------------------------------------------------
