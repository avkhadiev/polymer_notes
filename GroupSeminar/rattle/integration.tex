%%%%%%%%%%%%%%%%%%%%%%%%%%%%%%%%%%%%%%%%%%%%%%%%%%%%%%%%%%%%%%%%%%%%%%%%%%%%%%%%
%					 	             RATTLE: INTEGRATION SCHEME								             %
%%%%%%%%%%%%%%%%%%%%%%%%%%%%%%%%%%%%%%%%%%%%%%%%%%%%%%%%%%%%%%%%%%%%%%%%%%%%%%%%
% Describes how RATTLE builds off of the velocity Verlet integration scheme
%-------------------------------------------------------------------------------
% Section names and labels
\subsection{\rattle Integration Scheme}
\label{sec:rattle_constraints}
  %-----------------------------------------------------------------------------
  % Add text directly into the main file, or use \input statements
  %-----------------------------------------------------------------------------
  \par How does the modified velocity Verlet integration scheme look like in \rattle? The difference between the two is in constraint forces. If we include the constraint forces $\rattleConstraintForce$ in equations \ref{eq:rattle_velocityVerlet_velocityHalfStep}-\ref{eq:rattle_velocityVerlet_velocityFullStep}, we will obtain the new integration scheme:
  \begin{tcolorbox}
  %-----------------------------------------------------------------------------
  \begin{equation}
  \begin{split}
  \label{eq:rattle_integration_velocityHalfStep}
    \rattleVel \timeInHalfStep
      &= \rattleVel(t)
        + \halfstep
          \frac
            {\rattleForceNoIndex(t) + \rattleConstraintForceNoIndex(t)}
            {\rattleMassNoIndex} \\
      &= \rattleVelUnconstrained\timeInHalfStep
        + \halfstep \frac{\rattleConstraintForceNoIndex(t)}{\rattleMassNoIndex} \\
      &= \rattleVelUnconstrained\timeInHalfStep
        + \dlt{\rattleVel\timeInHalfStep}
  \end{split}
  \end{equation}
  \begin{equation}
  \begin{split}
  \label{eq:rattle_integration_positionFullStep}
    \rattlePos \timeInFullStep
      &= \rattlePos(t) + \timestep \rattleVel \timeInHalfStep \\
      &= \rattlePosUnconstrained\timeInFullStep
          + \dlt{\rattlePos\timeInFullStep}
  \end{split}
  \end{equation}
  %-----------------------------------------------------------------------------
  \begin{alignat}{4}
  \label{eq:rattle_integration_velocityFullStep}
    \rattleVel \timeInFullStep
      &= \rattleVel\timeInHalfStep
        &+&  \halfstep
          \frac{\rattleForceNoIndex\timeInFullStep}{\rattleMassNoIndex} \nonumber \\
        &&+& \halfstep
          \frac
            {\rattleConstraintForceNoIndex\timeInFullStep}
            {\rattleMassNoIndex} \nonumber \\
      &= \rattleVelUnconstrained\timeInFullStep
        &+& \halfstep
          \frac
            {\rattleConstraintForceNoIndex\timeInFullStep}
            {\rattleMassNoIndex} \nonumber \\
      &= \rattleVelUnconstrained\timeInFullStep
        &+& \dlt{\rattleVel\timeInFullStep}
  \end{alignat}
  %-----------------------------------------------------------------------------
  \end{tcolorbox}
  \par Where we have identified unconstrained velocity and position values, and defined the corrective displacements:
  \begin{tcolorbox}
  %-----------------------------------------------------------------------------
  \begin{align}
  \label{eq:rattle_integration_deltaVelocityHalfStep}
    \dlt{\rattleAtomVel\timeInHalfStep}
      &= \halfstep \frac{\rattleConstraintForce(t)}{\rattleMass} \\
  %-----------------------------------------------------------------------------
  \label{eq:rattle_integration_deltaPositionFullStep}
    \dlt{\rattleAtomPos\timeInFullStep}
      &= \timestep \dlt{\rattleAtomVel\timeInHalfStep} \\
  %-----------------------------------------------------------------------------
  \label{eq:rattle_integration_deltaVelocityFullStep}
    \dlt{\rattleAtomVel\timeInFullStep}
      &= \frac{\rattleConstraintForce\timeInFullStep}{\rattleMass}
  \end{align}
  %-----------------------------------------------------------------------------
  \end{tcolorbox}
  \par Since $\dlt{\rattleAtomPos\timeInFullStep}$ can be easily expressed via $\dlt{\rattleAtomVel\timeInHalfStep}$, we only need to concern ourselves with $\dlt{\rattleAtomVel\timeInHalfStep}$ and $\dlt{\rattleAtomVel\timeInFullStep}$. We can expand $\rattleConstraintForce$ as $\sum_{\rattleAtomIndexSecond}\rattleConstraintForceBond$ where, in turn, each summand can be expressed via Lagrange multipliers and bond vectors using equation \ref{eq:rattle_constraintForce}, yielding:
  \begin{align*}
    \dlt{\rattleAtomVel\timeInHalfStep}
      &= \sum_{\rattleConstraintIndex}
        \left(
            - \timestep \rattleLagrange(t)\rattleBondPos(t)
        \right) \\
      &= \sum_{\rattleConstraintIndex} \dlt{\rattleBondVel\timeInHalfStep} \\
    \dlt{\rattleAtomVel\timeInFullStep}
      &= \sum_{\rattleConstraintIndex}
      \left(
          - \timestep \rattleLagrange\timeInFullStep\rattleBondPos\timeInFullStep
      \right) \\
      &= \sum_{\rattleConstraintIndex} \dlt{\rattleBondVel\timeInFullStep} \\
  \end{align*}
  \par Where we defined $\dlt{\rattleBondVel\timeInHalfStep}$ and $\dlt{\rattleBondVel\timeInFullStep}$, and the definition of $\dlt{\rattleBondPos\timeInFullStep}$ follows from \ref{eq:rattle_integration_deltaPositionFullStep}:
  \begin{tcolorbox}
  %-----------------------------------------------------------------------------
    \begin{align}
    \label{eq:rattle_integration_deltaBondVelocityHalfStep}
      \dlt{\rattleBondVel\timeInHalfStep}
        &=  - \timestep \rattleLagrange(t)\rattleBondPos(t) \\
  %-----------------------------------------------------------------------------
    \label{eq:rattle_integration_deltaBondPositionFullStep}
      \dlt{\rattleBondPos\timeInFullStep}
        &=  - \timestep^2 \rattleLagrange(t)\rattleBondPos(t) \\
  %-----------------------------------------------------------------------------
    \label{eq:rattle_integration_deltaBondVelocityFullStep}
      \dlt{\rattleBondVel\timeInFullStep}
        &= - \timestep \rattleLagrange\timeInFullStep\rattleBondPos\timeInFullStep
    \end{align}
  %-----------------------------------------------------------------------------
  \end{tcolorbox}
  \par Since we can find the bond vectors and know the timestep, the problem of calculating the corrective steps is reduced to finding the Lagrange multipliers at times $t$ and $t + \timestep$.
  %-----------------------------------------------------------------------------
