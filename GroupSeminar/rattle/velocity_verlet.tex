%%%%%%%%%%%%%%%%%%%%%%%%%%%%%%%%%%%%%%%%%%%%%%%%%%%%%%%%%%%%%%%%%%%%%%%%%%%%%%%%
%					 	                    VELOCITY VERLET								                 %
%%%%%%%%%%%%%%%%%%%%%%%%%%%%%%%%%%%%%%%%%%%%%%%%%%%%%%%%%%%%%%%%%%%%%%%%%%%%%%%%
% Describes leap-frog velocity Verlet following Allend & Tildesley.
%-------------------------------------------------------------------------------
% Section names and labels
\subsection{Velocity Verlet}
\label{sec:rattle_velocityVerlet}
  %-----------------------------------------------------------------------------
  % Add text directly into the main file, or use \input statements
  %-----------------------------------------------------------------------------
  \par Given equations of motion:
  \begin{equation}
  \label{eq:rattle_velocityVerlet_eom}
  \rattleGeneralMass \rattleGeneralPosDDot = \rattleGeneralForce
  \end{equation}
  \par Velocity Verlet is a scheme for numerical integration of the above equations. Dropping the indices $\rattleGeneralPointIndex$ labeling the particles, it comprises the following three equations:
  \begin{tcolorbox}
  \begin{align}
  \label{eq:rattle_velocityVerlet_velocityHalfStep}
    \rattleVel \timeInHalfStep
      &= \rattleVel(t)
        + \halfstep \frac{\rattleForceNoIndex(t)}{\rattleMassNoIndex} \\
  \label{eq:rattle_velocityVerlet_positionFullStep}
    \rattlePos \timeInFullStep
      &= \rattlePos(t) + \timestep \rattleVel \timeInHalfStep \\
  \label{eq:rattle_velocityVerlet_velocityFullStep}
    \rattleVel \timeInFullStep
      &= \rattleVel \timeInHalfStep
        + \halfstep \frac{\rattleForceNoIndex\timeInFullStep}{\rattleMassNoIndex}
  \end{align}
  \end{tcolorbox}
  %-----------------------------------------------------------------------------
  \par The scheme comprises two stages, separated by the force evaluation loop:
  \begin{enumerate}
    \item
    \begin{itemize}
      \item Start with $\rattlePos(t)$, $\rattleVel(t)$, and $\rattleForceNoIndex(t)$.
      \item Calculate velocities at half timestep, $\rattleVel \timeInHalfStep$, according to equation \ref{eq:rattle_velocityVerlet_velocityHalfStep}.
      \item Using velocities at half timestep, $\rattleVel \timeInHalfStep$, calculate positions $\rattlePos \timeInFullStep$ according to equation \ref{eq:rattle_velocityVerlet_positionFullStep}; store the values.
    \end{itemize}
    \item Using $\rattlePos \timeInFullStep$, calculate the forces $\rattleForceNoIndex \timeInFullStep$; store the values.
    \begin{itemize}
      \item Using velocities at half timestep, $\rattleVel \timeInHalfStep$, and forces $\rattleForceNoIndex \timeInFullStep$, calculate velocities at full time step, $\rattleVel \timeInFullStep$, according to equation \ref{eq:rattle_velocityVerlet_velocityFullStep}; store the values.
    \end{itemize}
  \end{enumerate}
  \par This integration scheme stores positions, velocites and forces (or accelerations) at the same time, and minimizes the round-off error \cite{}.
